\begin{lstlisting}[caption=Eigenpaarberechnung mit Zeiterfassung, captionpos=b, label=alg:appAlgorithm:eigenPlot]
% Eingabe: -; Ausgabe: -
% Funktionsaufruf: eigenPlot

function eigenPlot

% Anpassung an Latexfont
set(0,'defaulttextinterpreter','latex');

% Beschriftung des Plots
figure; hold on;
xlabel('Dimension'); ylabel('Ben\"{o}tigte Rechenzeit in Sekunden');

% Initialisierung der x,y-Achsenwerte
xDimension = 10:10:1500; yTime = 10:10:1500;

  % Beginn der Eigenpaarberechnung
  for i=1:3
    for N=10:10:1500  % zehnstufiger Dimensionszuwachs
      tMatrix = tic; tEigenpairs = tic; % Timerinitialisierung
      A = rand(N); A = A'*A; % Erzeugung einer symmetrischen Matrix
      endMatrix = toc(tMatrix); % Dauer Matrixinitialisierung
      [X, D] = eig(A);  % Berechnung der Eigenpaare
      endEigenpairs = toc(tEigenpairs);
      yTime(N/10) = endEigenpairs-endMatrix;  % Dauer Eigenpaarberechnung
    end%for
    plot(xDimension, yTime);  % Erstellung des Diagramms
  end%for
  print -depsc eigLaufzeit;  % Umwandlung in .eps Datei

end%function
\end{lstlisting}

\newpage
\begin{lstlisting}[caption=Rayleigh-Ritz Verfahren mit Zeiterfassung, captionpos=b, label=alg:appAlgorithm:rayleighRitz]
% Eingabe: -; Ausgabe: -
% Funktionsaufruf: rayleighRitz

function rayleighRitz

% Anpassung an Latexfont
set(0,'defaulttextinterpreter','latex');

% Beschriftung des Plots
figure; hold on;
xlabel('Dimension $N$ der untersuchten Matrix');
ylabel({'Ben\"{o}tigte Rechenzeit in Sekunden'; ...
  'f\"{u}r $N/5$-dimensionalen Suchraum'});

% Initialisierung der x,y-Achsenwerte
xDimension = 10:10:1500; yTime = 10:10:1500;

  % Beginn der Eigenpaarberechnung
  for i=1:3
    for N=10:10:1500  % zehnstufiger Dimensionszuwachs
      tMatrix = tic; tRitzpairs = tic; % Timerinitialisierung
      A = rand(N); A = A'*A; % Erzeugung einer symmetrischen Matrix
      S = rand(N, N/5); % Suchraum der Dimension N/5
      endMatrix = toc(tMatrix); % Dauer Matrixinitialisierung

      % Rayleigh-Ritz Verfahren
      S = orth(S);  % Orthogonalisierung der Basisvektoren
      A = S'*A*S; % Verringerung der Dimension
      [X, D] = eig(A); S*X; % Berechnung der Ritz-Paare
      endRitzpairs = toc(tRitzpairs);
      yTime(N/10) = endRitzpairs-endMatrix;  % Dauer Eigenpaarberechnung
    end%for
    plot(xDimension, yTime);  % Erstellung des Diagramms
  end%for
  print -depsc rayleighRitzLaufzeit;  % Umwandlung in .eps Datei

end%function
\end{lstlisting}

\newpage
\begin{lstlisting}[caption=Berechnung des Winkels zwischen Ritz- und Eigenvektoren, captionpos=b, label=alg:appAlgorithm:ritzVecAngle]
% Eingabe: -; Ausgabe: -
% Funktionsaufruf: ritzPairAngle

function ritzPairAngle

% Anpassung an Latexfont
set(0,'defaulttextinterpreter','latex');

% Beschriftung des Plots
figure; hold on;
xlabel('Dimension der untersuchten Matrix');
ylabel({'Winkel in Bogenma{\ss}'});

% Initialisierung der x-Achse
xDimension = 10:10:500;

  % Beginn der Eigenpaarberechnung
  for i = 1:3
    for N = 10:10:500  % zehnstufiger Dimensionszuwachs
      %clearvars -except i N xDimension yMaxDist
      [i, N]
      A = rand(N); A = A'*A; % Erzeuge symmetrische Matrix
      [X,D] = eig(A);
      S = rand(N, 2); % Suchraum der Dimension 5

      % Rayleigh-Ritz Verfahren
      S = orth(S);  % Orthogonalisierung der Basisvektoren
      A2 = S'*A*S; % Verringerung der Dimension
      [X2, D2] = eig(A2); X2 = S*X2; % Berechnung der Ritz-Paare

      % Berechnung des minimalen Winkels
      p = 1:N; P = nchoosek(p,2);
      for k = 1:length(P) % = #Zeilen von P, wegen N >= 10
          Theta(k) = subspace(X2, X(:,P(k,:)));
      end
      yTheta(N/10) = min(Theta);
    end%for
    plot(xDimension, yTheta);  % Erstellung des Diagramms
  end%for
  print -depsc ritzPairAngle;  % Umwandlung in .eps Datei

end%function
\end{lstlisting}

\newpage

\begin{lstlisting}[caption=Berechnung des Winkels zwischen Ritz- und Eigenvektoren bei iterativem Rayleigh-Ritz Verfahren, captionpos=b, label=alg:appAlgorithm:iterRitzVecAngle]
% Eingabe: -; Ausgabe: -
% Funktionsaufruf: iterRayleighRitz

function iterRayleighRitz

% Anpassung an Latexfont
set(0,'defaulttextinterpreter','latex');

% Beschriftung des Plots
figure; hold on; xIteration = 1:50;
xlabel('Iteration'); ylabel('Winkel in Bogenma{\ss}');

%for i=1:3
A = rand(500); A=A'*A; [X, D] = eig(A);
Y1 = rand(500,2);
k1 = 1;

for j=1:100

    % Rayleigh-Ritz Prozedur
    S1 = A^k1*Y1; S1 = orth(S1);
    A1 = S1'*A*S1;
    [X1, D1] = eig(A1); Y1 = S1*X1;

    % Berechnung des Winkels
    p = 1:500; P = nchoosek(p,2);
    for k = 1:length(P) % = #Zeilen von P
        Theta(k) = subspace(Y1, X(:,P(k,:)));
    end
    yTheta(j) = min(Theta);
end%for
plot(xIteration, yTheta);

%print -depsc iterRitzAngle;

end%function
\end{lstlisting}

\newpage

\begin{lstlisting}[caption=FEAST, captionpos=b, label=alg:appAlgorithm:FEAST]
% Eingabe: -; % Ausgabe: -
% Funktionsaufruf: feast

function feast

% Initialisierungen
n = 500; iter = 10; lmin = 1; lmax = 2.5;

% Initialisierung des Plots
set(0,'defaulttextinterpreter','latex');
figure; hold on;
xlabel('Durchgef\"{u}hrte Iterationen');
ylabel('Winkel $\theta_{\min}(\mathcal{X},\mathcal{R})$');

% Erzeuge und transformiere die Filterfunktionen r auf
% das Intervall [lmin,lmax].
x = rkfun(); t = 2/(lmax-lmin)*x - (lmin+lmax)/(lmax-lmin);
s = rkfun('step', 5); r = s(t);

for j=1:3
    % Erzeuge HPD-Eigenwertproblem und skaliere die Matrizen
    A = 10*rand(n); A=A'*A - 1;
    B = 20*rand(n); B=B'*B;

    % Berechne Eigenpaare auf ]lmin,lmax[ mit eig(A,B) als Referenz
    [X, D] = eig(A,B);

    for i=1:n
        if lmin < D(i,i) < lmax
           eigVecInterval(:,i) = X(:,i);
           eigValInterval(i) = D(i,i);
        end%if
    end%for

    Yk = rand(n, length(eigValInterval));
    xAxis = 1:iter; M = B\A;

    for k=1:iter
        % Beschleunnigtes Rayleigh-Ritz Verfahren
        Pk = r(M, Yk); % Approximierung des Projektors
        Ak = Pk'*A*Pk; Bk = Pk'*B*Pk;
        [Xk, Dk] = eig(Ak, Bk);
        Yk = Pk*Xk;
        % Berechnung des Unterraumwinkels
        theta(k) = subspace(Yk, eigVecInterval);
    end%for
    semilogy(xAxis, theta);
end%for
end%function

\end{lstlisting}
