Das Lösen von Eigenwertproblemen ist eine Standarddisziplin in der
numerischen linearen Algebra. Gleichungen der Gestalt
\[
Ax = \lambda Bx
\]
begegnet man in ganz unterschiedlichen Kontexten. So sind sie beispielsweise
bei der Bestimmung von Eigenfrequenzen oder dem Ermitteln von Fixpunkten beim
Rotieren eines Fußballs\footnote{Hier wird auf den bekannten
\emph{Satz vom Fußball} angespielt. Dieser besagt, dass auf einem Fußball
zwei Punkte existieren, die zu Spielbeginn und zur Halbzeit
an der gleichen Stelle liegen -- informell formuliert.} ebenso wie beim
Untersuchen des PageRanks einer Website von
Bedeutung. Entsprechend strotz der Kanon von angebotenen numerischen
Lösungsmethoden von Vielfalt und Virtuosität.\\

Mit dem FEAST-Algorithmus, den \textsc{Eric Polizzi} in seinem Paper
\glqq\emph{Density-marix-based algorithm for solving eigenvalue problems}\grqq %~\cite{Pol}
vorstellt, wird ein Werkzeug zur Verfügung gestellt, um obiges Problem
für eine hermitesche Matrix $A$ und eine hermitesche, positiv definite
Matrix $B$ zu lösen.\\

Die vorliegende Arbeit wird die grundlegenden mathematischen Ideen dieses Algorithmus'
vorstellen. Neben einer naiven Implementation in MATLAB soll au"serdem
eine weitere Umsetzung besprochen werden, welche durch die von
\textsc{Mario Berljafa} und \textsc{Stefan G"uttel} entwickelte
\glqq\emph{Rational Krylov Toolbox for MATLAB}\grqq %~\cite{RKT}
ermöglicht wird.



\section{Grundlagen}

Um das Lesen dieser Arbeit mehr zu einer Freude denn zu einer Schikane zu
machen, soll dieser Abschnitt einige Grundlagen der linearen Algebra und
der Funktionentheorie bereitstellen. Obschon sich der Autor bem"uht hat,
in der Literatur g"angige Notation zu benutzen, bittet er den
verst"andnissvollen Leser bei Unklarheiten im Anhang \glqq Notationen\grqq\
nachzuschlagen.\\

Beginnen wir mit Definitionen und Resultaten aus der Matrizentheorie.
Eine Matrix $A\in\Cnn$ wird als \emph{hermitesch} bezeichnet, falls
sie die Identit"at $A=A^H$ erf"ullt. Diese l"asst sich nach dem
\emph{Spektralsatz} unit"ar diagonalisieren. Das hei"st,
unter Aufbringung einiger kognitiven Anstregungen lassen sich ... finden.
