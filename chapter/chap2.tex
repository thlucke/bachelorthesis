Dieses Kapitel widmet sich der Frage, ob und wie man aus der Menge aller Eigenpaare
des Problems
\[
Ax = \lambda Bx
\]
eine gew"unschte Teilmenge ausw"ahlen kann (das hei"st filtern). Dazu werden im Rahmen dieser Arbeit aus dem Katalog der Verfahren
die zwei Folgenden ausgew"ahlt und vorgestellt: Rayleigh-Ritz-Verfahren\footnote{Die im Folgenden pr"asentierte Vorgehensweise wird in der
Literatur nicht einheitlich bezeichnet. Anstelle von Rayleigh-Ritz-Verfahren sind auch die Bezeichnungen Galerkin-Methoden oder ... "ublich (siehe da und da).
In dieser Arbeit wird durchg"angig vom Rayleigh-Ritz-Verfahren oder der Rayleigh-Ritz-Methode
gesprochen.} und Konturintegration. An einem abschlie"senden Beispiel wird au"serdem die Kompatibilit"at beider Konzepte dargelegt.

\section{Rayleigh-Ritz-Verfahren}\label{sec:ritz}
%\footnote{Dieser Abschnitt orientiert sich in Formulierung und Dramaturgie an Abschnitt 4.3.1 aus Y. Saad Solving large evp. Hier verallgemeinern wir aber die Konzepte direkt auf allg. ewp.}
Die Idee dieser Klasse von Verfahren ist das Approximieren des von
den gesuchten Eigenvektoren aufgespannten Unterraums. %Dazu sollen -- wie der Name vermuten l"asst --
%Projektionen bem"uht werden.\\
Bevor wir das allgemeine Eigenwertproblem untersuchen, wenden wir unsere Aufmerksamkeit
dem gew"ohnlichen Eigenwertproblem
\[
Ax = \lambda x% \textcolor{red}{variablen wieder neu einf"uhren? oder in der einleitung: im folgenden sei stets...}
\]
zu und betrachten f"ur eine Zahl $m\in\N$ mit $m\le n$
einen $m$-dimensionalen Unterraum $\U_m\subseteq\Cn$. Dieser zun"achst nicht
n"aher bestimmte \emph{Suchraum} wird als Grundlage f"ur das Verfahren gew"ahlt.
Gesucht sind nun Paare $(\widetilde{\lambda}, \widetilde{x})\in\C\times \U_m \setminus\{0\}$ --
die wir als approximierte L"osungen des Eigenproblems verstehen wollen --
welche die Eigenschaft
\begin{equation}\label{eq:orthogonal}
\langle u, A\widetilde{x} - \widetilde{\lambda}\widetilde{x} \rangle=0
\end{equation}
f"ur alle $u \in \U_m $ erf"ullen. Das Residuum $(A\widetilde{x} - \widetilde{\lambda}\widetilde{x})$
soll also orthogonal auf dem Suchraum stehen. Paare, die diesem Anliegen nachkommen,
werden auch \emph{Ritz-Paare} bez"uglich des Suchraums $\U_m$ genannt.\footnote{Ritz-Paare m"ussen nicht notwendigerweise eine Orthogonalit"atsbedingung erf"ullen... oder doch?}\\

Wir wollen nun annehmen, dass mit der Menge von Vektoren $\{u_i\}_{i=1:m}\subseteq \Cn$ eine Orthonormalbasis
des Unterraums $\U_m$ gegeben ist. Definieren wir dann die Matrix $U_m :=[u_i]_{i=1:m}\in\C^{n,m}$, so muss wegen $\widetilde{x}\in\U_m$
ein Vektor $y\in\C^m$ existieren, mit $U_m y = \widetilde{x}$. Die Forderung \eqref{eq:orthogonal}
ist dann zu der Gleichung
\[
U_m^H(AU_m y - \widetilde{\lambda} U_m y) = 0
\]
"aquivalent. Als direkte Konsequenz dieser Umformulierung erhalten wir unter Ausnutzung
der Orthogonalit"at der Spalten von $U_m$ mit
\begin{equation}\label{eq:transform}
(U_m^H A U_m) y = \widetilde{\lambda}y
\end{equation}
ein neues Eigenwertproblem. Jedes Eigenpaar $(\widetilde{\lambda},y)$ von \eqref{eq:transform}
liefert dann nach Konstruktion mit $(\widetilde{\lambda}, U_m y)$ ein Ritz-Paar des gew"ohnlichen
Eigenwertproblems bez"uglich des Suchraums $\U_m$. In Abh"angigkeit von der Wahl
des Suchraumes variiert nat"urlich G"ute der Approximation. Ist etwa $\U_m$
durch Eigenvektoren aufgespannt, so ist jedes Ritz-Paar schon ein Eigenpaar.
Wir werden sp"ater aus den Untersuchungen des allgemeinen Eigenwertproblems folgern,
dass bereits im Falle der Invarianz von $\U_m$ unter $A$ jedes Ritz-Paar von $A$ bez"uglich $\U_m$ bereits ein Eigenpaar von $A$ ist.
An dieser Stelle begn"ugen wir uns vorerst mit dieser erfreulichen Botschaft.\\

%Fassen wir die eben geschilderte Vorgehensweise algorithmisch zusammen, so ist
%der folgende Pseudocode denkbar:
Die eben skizzierte Vorgehensweise zur Berechnung von Ritz-Paaren des gew"ohnlichen Eigenwertproblems
l"asst sich algorithmisch wie folgt zusammenfassen.

\begin{algorithm}\label{alg:rp}
\caption{Berechnung von Ritz-Paaren (Vgl. ~\cite[98]{saad})}
\begin{algorithmic}[1]
\State Berechne ONB $\{u_i\}_{i=1:m}$ von $\mathcal{U}_m$ und setze $U_m\gets[u_i]_{i=1:m}$.
\State Setze $\widetilde{A}\gets U_m^H A U_m$ und
l"ose $\widetilde{A}Y = Y \Lambda$ in $Y$ und
$\Lambda$.
\State Setze $\widetilde{X} \gets U_m Y$ und gib Ritz-Paare $(\lambda_i, \widetilde{x}_i)_{i=1:m}$ aus.
\end{algorithmic}
\end{algorithm}

Hierbei entsprechen die Spalten von $Y\in\C^{n,m}$ gewissen Eigenvektoren von \eqref{eq:transform}, w"ahrend auf der Diagonalen von $\Lambda\in\C^{m,m}$
die zugeh"origen Eigenwerte zu finden sind.\\

Die Idee dieser Methode ist also simpel: Transformiere das Eigenwertproblem mit
einer gewissen Matrix in ein anderes Eigenwertproblem und benutze dessen L"osungen,
um Ritz-Paare des urspr"unglichen Problems zu erhalten.
Doch wozu die M"uhe, das urspr"ungliche Eigenwertproblem in ein anderes
Eigenwertproblem zu "uberf"uhren? Zwar gelingt es, aus dem transformierten Problem
\eqref{eq:transform} Ritz-Paare zu extrahieren, aber w"are nicht auch denkbar,
s"amtliche Eigenpaare von $A$ zu approximieren und die zum Unterraum $\U_m$
korrespondierende Teilmenge direkt auszuw"ahlen?\\

Dies mag in Einzelf"allen in der Tat sinnvoller sein. Sprechen wir allerdings
von Matrixdimensionen jenseits der Vorstellungskraft, ist eine vollst"andige
Berechnung aller Eigenpaare mitunter ein sehr zeitintensives Vergn"ugen. Bei
genauerer Betrachtung der Gleichung \eqref{eq:transform} f"allt auf, dass die
Matrix $(U_m^H A U_m) \in \C^{m,m}$ im Falle $n \gg m$ ein mitunter deutlich kleineres
Format hat, als die Matrix $A$ im urspr"unglichen Problem. Man darf hier also erwarten,
dass die ben"otigte Laufzeit zur Bestimmung der Ritz-Paare mit dem Algorithmus \ref{alg:rp}
geringer ist, als beim Approximieren s"amtlicher Eigenpaare von $A$. Dies
wird im vierten Kapitel anhand ausgew"ahlter Beispiele vorgef"uhrt.\\

Nun, da wir der Existenz und Funktionsweise der Rayleigh-Ritz-Methode gewahr wurden, erweitern wie die obige Theorie auf das
verallgemeinerte Eigenwertproblem
\begin{equation}\label{chap2:gevp}
Ax = \lambda Bx.
\end{equation}
Dabei gehen wir ganz analog zum gew"ohnlichen Eigenwertproblem vor und betrachten wieder
einen $m$-dimensionaler Suchraum $\U_m\subseteq \Cn$.
Gefunden werden sollen dieses Mal Paare $ (\widetilde{\lambda}, \widetilde{x}) \in \C
\times \U_m \setminus\{ 0\}$ die der Orthogonalit"atsbedingung
\begin{equation}\label{eq:borthogonal}
A\widetilde{x} - \widetilde{\lambda}B\widetilde{x} \ \bot \ \U_m
\end{equation}
gen"ugen. Durch die Wahl eines geeigneten Vektors $y\in\C^m$ kann die N"aherungsl"osung $\widetilde{x}$ wie zuvor durch das Produkt $U_m y$ ersetzt werden, wobei die Spalten von $U_m$ erneut eine ONB des Suchraums $\U_m$ bilden. Folglich l"asst sich die zu \eqref{eq:borthogonal} "aquivalente Forderung
\[
U_m^H(AU_m y - \widetilde{\lambda} BU_m y) = 0.
\]
aufstellen. Wie bereits beim gew"ohnlichen Eigenwertproblem, l"asst sich nun aus jeder L"osung
$(\widetilde{\lambda}, y)$ von
\begin{equation}\label{eq:transformedevp}
(U_m^H AU_m) y = \widetilde{\lambda} (U_m^H B U_m) y.
\end{equation}
mit $(\widetilde{\lambda}, U_m y)$ ein Ritz-Paar f"ur das verallgemeinerte Eigenwertproblem
gewinnen.\\

Da mit der Matrix $B$ ein weiterer Darsteller auf der Eigenproblemb"uhne ber"ucksichtigt werden muss, zieht dies als Konsequenz eine Anpassung des Algorithmus' \ref{alg:rp} nach sich. Wir k"onnen diesen in der folgenden Manier abwandeln.

\begin{algorithm}\label{alg:grp}
\caption{Berechnung von Ritz-Paaren}
\begin{algorithmic}[1]
\State Berechne ONB $\{u_i\}_{i=1:m}$ von $\mathcal{U}_m$ und setze $U_m\gets[u_i]_{i=1:m}$.
\State Setze $\widetilde{A}\gets U_m^H A U_m$,
$\widetilde{B} \gets U_m^H BU_m$ und
l"ose $\widetilde{A}Y = \widetilde{B}Y \Lambda$ in $Y$ und $\Lambda$.
\State Setze $\widetilde{X} \gets U_m Y$ und gib Ritz-Paare $(\lambda_i, \widetilde{x}_i)_{i=1:m}$ aus.
\end{algorithmic}
\end{algorithm}

Nun mag der Fall eintreten, dass bei der Berechnung der Ritz-Paare gewisse Toleranzen ber"ucksichtigt werden m"ussen.
Es k"onnten etwa maximale euklidische Abweichungen der Ritz-Werte zu den Eigenwerten vorgegeben sein.
Auch Vorgaben bez"uglich des Winkels zwischen Eigenraum und dem Unterraum der Ritz-Vektoren sind denkbar. Dann ist es erforderlich die Wahl des Suchraums nicht dem Geschick alleine zu "uberlassen, sondern eine gegebenenfalls weniger geschickte Wahl algorithmisch kompensieren zu k"onnen.
Dies kann zum Beispiel durch das Einf"uhren einer Iterationsvorschrift gelingen, die in jedem Iterationsschritt den Suchraum "andert.
Wir werden diese Idee zu einem sp"ateren Zeitpunkt in dieser Arbeit weiterverfolgen.\\

Wir wollen uns nun vorerst von der Algorithmik verabschieden und einige Beobachtungen
festhalten. Eingangs wurde behauptet, dass die Invarianz des Suchraumes $\U_m$ unter
$A$ dazu f"uhrt, dass jedes Ritz-Paar bez"uglich $\U_m$ ein Eigenpaar von $A$ ist.
Doch warum ist das so? Dies zu beantworten verpflichtet sich der folgende Satz.

\begin{thm}\label{thm:invariant}
Neben zwei Matrizen $A,B\in\Cnn$ -- wobei $B$ eine HPD-Matrix ist -- sei f"ur
$m\in\N$ mit $m\le n$ ein $m$-dimensionaler Unterraum $\U_m \subseteq \C_n$ gegeben.
Ist dieser invariant unter $B^{-1}A$, so ist jedes Ritzpaar von $B^{-1}A$
bez"uglich $\U_m$ auch ein Eigenpaar von $(A,B)$.
\end{thm}

\begin{proof}
Beginnen wir mit einer ONB $\{u_i\}_{i=1:m}\subseteq\U_m$ des Unterraums $\U_m$
und setzen wie bisher $U_m := [u_i]_{i=1:m}\in\C^{n,m}$. Aufgrund der Invarianz
von $\U_m$ unter $B^{-1}A$ muss eine Matrix $V_m \in \C^{m,m}$ existieren, welche
die Gleichung
\begin{equation}\label{eq:thminvariant}
B^{-1}A U_m = U_m V_m
\end{equation}
erf"ullt. Insbesondere folgt unter Ausnutzung der Orthogonalit"at der Spalten
von $U_m$ die Identit"at $U_m^H B^{-1}A U_m = V_m$.
Sind nun $\lambda\in\C$ und $y\in\C^m\setminus\{0\}$ so gew"ahlt, dass $(\lambda, U_m y)$
ein Ritz-Paar von $B^{-1}A$ ist, so folgt aus
\[
U_m^H B^{-1}A U_m y = V_m y
\]
mit \eqref{eq:thminvariant} die Gleichung
\[
B^{-1}AU_m y = U_m V_m y = \lambda U_m y
\]
und schlie"slich auch die Behauptung durch Umstellen.
\end{proof}

Da das gew"ohnliche Eigenwertproblem ein Spezialfall des allgemeinen Eigenwertproblems
ist, k"onnen wir aus dem eben bewiesenen Resultat unmittelbar das folgende Korollar
ableiten.

\begin{kor}
Ist $A\in\Cnn$ und $\U_m\subseteq \Cn$ ein $m$-dimensionaler $A$-invarianter Unterraum, so ist
jedes Ritz-Paar von $A$ bez"uglich $\U_m$ ein Eigenpaar von $A$.
\end{kor}

\begin{proof}
Betrachte f"ur $B:=I_n$ das verallgemeinerte Eigenwertproblem
\[
Ax = \lambda Bx = \lambda x.
\]
Die Aussage folgt dann aus dem vorigen Satz.
\end{proof}

Es wird sich zu einem sp"ateren Zeitpunkt herausstellen, dass diese Erkenntnis aus algorithmischer Sicht h"ochst n"utzlich ist.
Die Rayleigh-Ritz-Methode l"asst sich nämlich in ein iteratives Verfahren umwandeln, welches in jedem Schritt den Suchraum "andert. Ist die Suchraumiterierte irgendwann einmal $A$-invariant, so kann der Algorithmus abgebrochen werden.\\

Wenden wir uns einer weiteren Kuriosit"at zu. Zwischen obigem Verfahren und dem Konzept der Projektion besteht ein enger Zusammenhang.
Um zu dieser Einsicht zu gelangen, bedarf es ein wenig
Vorbereitung.
%Das Rayleigh-Ritz-Verfahren steht in einem engen Zusammenhang zu Projektionsverfahren.
%Mit einer geeigneten Projektionsmatrix $P\in\Cnn$ ist ein Paar $(\lambda, x)\in\C\times\Co$
%\textcolor{red}{(oder $\in\C\times\U_m\setminus\{0\}$?)} genau dann eine L"osung
%der Gleichung \eqref{chap2:gevp}, wenn es eine L"osung von
%\[
%P^H APx = \lambda P^H BPx
%\]
%ist. Richtig definiert, besitzt $P$ noch weitere n"utzliche Eigenschaften, die
%der folgende Satz vorstellt.
\begin{thm}\label{thm:projektor}
Es sei $B\in\Cnn$ eine HPD-Matrix und f"ur $m\in\N$ mit $m\le n$ sei
ein $m$-dimensionaler Unterraum $\U_m \subseteq \Cn$ gegeben. Sei weiter $\{u_i\}_{i=1:m}\subseteq\U_m$ eine
Basis $B$-orthonormaler Vektoren, das hei"st, die Matrix $U_m := [u_i]_{i=1:m}
\in\C^{n,m}$ erf"ulle die Gleichung $U_m^H B U_m = I_m$. Dann ist die von der Matrix
 $P := U_m U_m^H B \in \Cnn$ induzierte lineare Abbildung
\[
p \colon \Cn \to \Cn, x\mapsto U_m U_m^H Bx
\]
eine $B$-orthogonale Projektion auf den Unterraum $\U_m$. Au"serdem gilt
f"ur alle $x\in\Cn$ die Identit"at
\[
\|x-p(x)\|_B = \min_{y\in\U_m} \|x - y\|_B.
\]
\end{thm}

\begin{proof}
Aus der $B$-Orthogonalit"at von $U_m$ folgt f"ur alle $x\in\Cn$
\[
p^2 (x) = U_m U_m^H B U_m U_m^H B x= U_m U_m^H Bx = p(x).
\]
%und damit gilt die Mengengleichheit $p(\Cn) = \U$ nach Konstruktion. %($\Rang{P_B} = \dim(\U)$).
Ist nun $y\in\U_m$ und $x\in\Cn$, dann folgt wegen $p(y) = y$ und $B=B^H$ auch
\begin{align*}
\langle y, x-p(x)\rangle_B &= y^H (Bx - B U_m U_m^H Bx) \\
&= y^H (Bx - B^H U_m U_m^H Bx) \\
&= y^H Bx - p(y)^H Bx = 0.
\end{align*}
Es gilt demnach $x-p(x) \ \bot_B \ \U_m$. Die Optimierungsaufgabe wird schlie"slich wegen
\begin{align*}
\|x-y\|_B^2 &= \|x-p(x) + p(x)-y\|_B^2 \\
&= \|x-p(x)\|_B^2 + \|p(x)-y\|_B^2\\
&\ge \|x-p(x)\|_B^2
\end{align*}
gel"ost. Dabei ist bei der zweiten Gleichheit zu ber"ucksichtigen, dass $x-p(x) \in \U_m^{\bot_B}$
und $p(x)-y \in \U_m$ gilt.
%denn aufgrund der positiven Definitheit von Normen, gilt Gleichheit genau dann,wenn $p(x)=x$ erf"ullt ist.
\end{proof}
Mit Hilfe der im Satz eingef"uhrten Projektionsmatrix $P$, l"asst sich die Gleichung
\eqref{eq:transformedevp} -- zu finden auf Seite \pageref{eq:transformedevp} -- umformulieren. Durch Linksmultiplikation
mit $B^H U_m$ und dem Einschub der Identit"at $U_m^H B U_m$ gilt n"amlich
\[
B^H U_m (U_m^H A U_m)(U_m^H B U_m)y = \widetilde{\lambda} B^H U_m (U_m^H B U_m)(U_m^H B U_m)y
\]
und folglich
\[
P^H A P U_m y = \widetilde{\lambda} P^H B P U_m y.
\]
Wenn wir uns an dieser Stelle erinnern, dass $\widetilde{x}$ durch $U_m y$ ersetzt wurde, so
erhalten wir
\[
P^H A P \widetilde{x} = \widetilde{\lambda} P^H B P \widetilde{x}.
\]
Im Fall der gew"ohnlichen Eigenwertgleichung erhalten wir speziell
\[
P A P \widetilde{x} = \widetilde{\lambda} \widetilde{x}.
\]
Wenden wir uns nun nochmal dem Rayleigh-Ritz-Verfahren zu und betrachten das
verallgemeinerte Eigenwertproblem f"ur eine hermitsche Matrix $A\in\Cnn$ sowie eine
HPD-Matrix $B\in\Cnn$. Angenommen, uns st"unde mit dem Suchraum $\U_m$ bereits ein von Eigenvektoren
aufgespannter Unterraum zur Verf"ugung. Dann existiert eine Matrix $U_m := [u_i]_{i=1:m}\in\C^{n,m}$ mit
$B$-orthonormalen Spalten aus Eigenvektoren von $(A,B)$ und $\Bild(U_m)=\U_m$, sowie eine reelle Diagonalmatrix
$\Lambda_m := \text{diag}(\lambda_i)_{i=1:m}\in\C^{m,m}$ mit
\[
AU_m = BU_m\Lambda_m.
\]
Dabei ensprechenden die Diagonaleintr"age von $\Lambda_m$ gerade den zu den Spalten
von $U_m$ korrespondierenden Eigenwerten von $(A,B)$. Folglich w"are das Bild
des Spektralprojektors $P=U_m U_m^H B$ nach Konstruktion wegen
\[
B^{-1}AP = B^{-1}(AU_m)U_m^H B = B^{-1}(BU_m \Lambda)U_m^H B
= U_m\Lambda U_m^H B = \left(\sum_{i=1}^m \lambda_i u_i u_i^H\right)\cdot B
\]
ein $(B^{-1}A)$-invarianter Unterraum. Daher erscheint es
wegen Satz \ref{thm:invariant} sinnvoll, den Spektralprojektor $P$ in das
Rayleigh-Ritz-Verfahren zu integrieren.
Da jedoch der Unterraum $\U_m$ im Allgemeinen nicht bekannt ist, stellt sich die Frage, ob es
"uberhaupt sinnvoll ist nach dem Projektor zu suchen.
Wir werden jedoch alsbald feststellen, dass %zumindest
die Matrix $U_m U_m^H$ analytisch berechnbar ist.


% nochmal ein wort verlieren, warum hier jetzt von filterung die rede sein kann
%folgender teil vielleicht eher was f"ur den anfang des kaptitels

%Im Allgemeinen ist es nicht ohne Weiteres m"oglich einen bestimmten Eigenraum gezielt
%zu approximieren, wenn keine weiteren Informationen bekannt sind. G"angige
%Wahl f"ur Suchr"aume sind Krylow-R"aume. Deren Struktur f"uhrt dazu, dass
%Unterr"aume approximiert werden, die von den Eigenvektoren mit den betragsm"a"sig
%gr"o"sten Eigenwerten aufgespannt werden. Es obliegt daher dem pers"onlichen
%Geschick einen passenden Suchraum aufzustellen.\\



%\textcolor{red}{anstelle einer basis von U kann auch beliebige matrix Q gew"ahlt
%werden und dann diese auf Um projeziert werden mit $U:=U_m^H U_m B Q$ Dann
%ist man fertig, aber Projektor ist i.A.nicht bekannt, daher approximieren. siehe
%kapitel 3}

\section{Konturintegration}\label{sec:kontur}

Die folgenden Zeilen pr"asentieren eine Vorgehensweise, die es gestattet, den im vorigen
Abschnitt eingef"uhrten Spektralprojektor analytisch zu konstruieren. Wir werden
zu einem sp"ateren Zeitpunkt untersuchen, in wie weit die Numerik in der Lage ist, dieses
Verfahren algorithmisch umzusetzen.\\

Zun"achst ist es erforderlich einige Begriffe und Notationen einzuf"uhren. F"ur zwei Zahlen $z_1, z_2 \in\C$ sei mit der Menge
\[
[z_1, z_2] := \{(1-t)z_1 + tz_2 \in\C \mid t\in[0,1]\}
\]
die direkte Verbindungsstrecke zwischen $z_1$ und $z_2$ bezeichnet. Im Spezialfall $z_1,z_2 \in\R$
stellen wir uns daher unter $[z_1,z_2]$ ein reellwertiges Intervall eingebettet in die komplexe Zahlenebene vor.
Des Weiteren ben"otigen wir sogenannte \emph{Jordan-Kurven}. Diese definieren wir im Kontext
dieser Arbeit wie folgt.

\begin{defn} \label{def:jordankurve}
  Es sei $S:=\{z\in\C : |z| = 1\}$. Eine Menge
  $\Gamma\subset\C$ hei"st \emph{Jordan-Kurve}, falls
  ein Hom"oomorphismus $\gamma\colon S\to \C$ existiert mit $\Gamma = \Bild(\gamma)$. Dabei ist
  ein \emph{Hom"oomorphismus} eine stetige Bijektion mit stetiger Umkehrabbildung.
\end{defn}

Insbesondere sind Jordan-Kurven also geschlossene und "uberschneidungsfreie Kurven.
Ausgestattet mit diesen neuen Vokabeln kann die nun Pr"asentation des angek"undigten Verfahrens beginnen. Im weiteren Verlauf seien $A, B \in\Cnn$ hermitesche Matrizen und $B$ zus"atzlich
positiv definit.\\

Wir wollen annehmen, dass zwei Zahlen $\lambda_1, \lambda_2 \in \R$ mit $\lambda_1 < \lambda_2$
vorgegeben sind, die das abgeschlossene Intervall $I:=[\lambda_1, \lambda_2]$ definieren.
Wie bisher sollen Paare der Gestalt $(\lambda, x) \in I \times \Co$ ermittelt werden,
welche der verallgemeinerten Eigenwertgleichung
  \begin{equation}\label{eq:eigen} %eventuell einfaches ewp
  Ax = \lambda Bx
  \end{equation}

gen"ugen. Der Ausgangspunkt zur Bestimmung der gesuchten Eigenpaare %\footnote{Um der besseren Lesbarkeit Willen werden
%im Folgenden die verallgemeinerten Eigenvektoren und deren verallgemeinerte
%Eigenwerte kurz als Eigenvektor und Eigenwert bezeichnet.}
ist die durch%wird das Problem mit Hilfe einer
%gewissen Matrix $Q\in\Cnn$ in das "aquivalente Problem
%  \begin{equation}\label{eq:Qeigen}
%  A_Q \phi = \mu B_Q \phi
%  \end{equation}
%"uberf"uhrt. Hierbei sind $A_Q = Q^TAQ$, $B_Q = Q^TBQ$ und $(\mu, \phi)
%\in I\times\C^n$.
%Wird die Matrix $Q$ richtig gew"ahlt, so ist dann jeder zul"assige Eigenwert von
%\eqref{eq:Qeigen} auch ein zul"assiger Eigenwert von \eqref{eq:eigen} und
%umgekehrt.\footnote{Daher ist die Angabe $\mu\in I$ gerechtfertigt.}
%F"ur die Ermittlung der gesuchten Eigenvektoren bedarf es hingegen zus"atzlicher Arbeit.\\
%Zur Bestimmung der Transformationsmatrix $Q$ wird eine Konturintegration bem"uht.
%Ausgangspunkt dieser Integration ist die durch
  \begin{align*}
  G\colon\Omega &\to\Cnn\\
  \omega &\mapsto (\omega B - A)^{-1}
  \end{align*}

definierte \emph{Green-Funktion}, wobei $\Omega \subseteq \C$ eine vom Spektrum
von $B^{-1}A$ disjunkte Teilmenge der komplexen Zahlen ist.\footnote{W"are $\Omega$ nicht disjunkt vom Spektrum von $B^{-1}A$, so g"abe es ein $\omega\in\Omega$, welches zur Singularit"at von $(\omega B-A)$ f"uhrte.}
Diese Funktion $G$ wird nun "uber eine Jordan-Kurve $\Gamma$,
die um das vorgegebene Intervall $I$ \glqq heruml"auft\grqq, %Pr"aziser!%
in der Gestalt
\begin{equation}\label{eq:intgreen}
\frac{1}{2\pi\iota}\int_\Gamma G(\omega)\text{ d}\omega
\end{equation}
integriert.\footnote{Mit $\iota$ ist im Folgenden stets die imagin"are Einheit bezeichnet,
also $\iota = \sqrt{-1}$.} Das Integral ist hierbei eintragsweise zu verstehen.\\

Wir wollen nun annehmen, dass es genau $m\in\N$ Eigenwerte gibt,
die im Inneren von $I$ liegen. %\footnote{Diese Forderung ist wohldefiniert, da
%der Eigenschaften von $A$ und $B$ wegen alle %Eigenwerte von $B^{-1}A$ reell sind.}
%F"ur zu diesen Eigenwerten passende Eigenvektoren $\{x_i\}_{i=1:k} \subseteq \Cn$ sei au"serdem die Matrix
%$X_k := [x_i]_{i=1:k}$ gegeben.
Dann kann man nachweisen, dass sich \eqref{eq:intgreen} durch eine Matrix
$U_m = [u_i]_{i=1:k}\in\C^{n,m}$ mit $B$-orthonormalen Spalten in
\[
\frac{1}{2\pi\iota}\int_\gamma G(\omega)\text{ d}\omega = U_m U_m^H
\]
faktorisieren l"asst, wobei $u_i$ f"ur jedes $i=1:m$ ein zu einem in $I$ liegenden
Eigenwert korrespondierender Eigenvektor ist.\footnote{F"ur einen Beweis des Spezialfalls $B=I$ sei auf \cite[64f]{gohberg} und \cite[222-226]{liesen} verwiesen.}
Als direkte Konsequenz erhalten wir folglich mit dem Produkt
\[
\left( \frac{1}{2\pi\iota}\int_\gamma G(\omega)\text{ d}\omega \right) \cdot B
\]
den Spektralprojektor $U_m U_m^H B$. %Mit diesem Projektor
%k"onnen wir nun vorgehen wie im vorigen Kapitel und alle Eigenpaare des Problems
%finden.% Bedauerlicherweise bleibt die Dimension des Problems unver"andert. Daher
%begn"ugt man sich anstelle der Matrix $B$ mit einer vollrangigen Matrix $\widetilde{B}\in\C^{n,k}$
%um das Ausgangsproblem auf die Dimension $(k\times k)$ zu reduzieren.\\

%MEHR KOMPANA!
%B-Orthogonalit"at der x-vektoren
\newpage
\section{Illustration}\label{sec:bsp}

Bevor wir uns vom zweiten Kapitel verabschieden, sollen die Erkenntnisse der vorangegangenen
Abschnitte an einem Beispiel vorgef"uhrt werden. Dabei werden wir ausgehend von einem
verallgemeinerten Eigenwertproblem gewisse Eigenwerte filtern und die Konturintegration
mit der Ritz-Methode kombinieren.\\

Daf"ur wenden wir uns den beiden Matrizen
%Beipsiel: einmal mit irgendner matrix $\widetilde{B}$ und dann einmal $P^H AP$
%mit dem Spektralprojektor l"osen. (hei"st: einmal gro"ses Problem und einmal
%kleines Problem)
$A,B \in \C^{3,3}$ zu, welche durch
\[
A:= \text{diag}(3,1,4)\text{ und }
B:= \text{diag}(1,5,9)
\]
gegeben sind. Dann ist $A$ hermitesch und $B$ eine HPD Matrix.
Durch einfaches Nachrechnen "uberpr"uft man, dass das
verallgemeinerte Eigenwertproblem \eqref{eq:eigen} durch Vektoren
$x_1 \in \spn_\C\{e_1\}$ mit zugeh"origem Eigenwert $\lambda_1 = 3$, Vektoren
$x_2 \in \spn_\C\{e_2\}$ mit zugeh"origem Eigenwert $\lambda_2 = 1/5$, sowie
Vektoren $x_3 \in\spn_\C\{e_3\}$ mit zugeh"origem Eigenwert $\lambda_3 = 4/9$
gel"ost wird.\footnote{Hier bezeichnen $e_1, e_2, e_3 \in\Cn$ die kanonischen Einheitsvektoren.}\\

Vergessen wir f"ur den Moment, dass uns die Eigenpaare bekannt sind und versuchen
mit den Methoden, die in den vorangegangenen Abschnitten vorgestellt wurden, die Eigenpaare auf $I = [-1,1]$ zu bestimmen.
Gem"a"s Abschnitt \ref{sec:kontur} w"ahlen wir daher als Integrationskontur $\Gamma$ das Bild der Funktion
\begin{equation}\label{kurve}
\gamma\colon [0,2\pi]\to\Cn\text{, }\varphi\mapsto e^{\iota \varphi}
\end{equation}
und integrieren die Green-Funktion
\[
G(\omega) = \text{diag}(\omega-3, 5\omega-1, 9\omega-4)^{-1}
=\text{diag}\left(\frac{1}{\omega-3}, \frac{1}{5\omega-1}, \frac{1}{9\omega-4}\right)
\]
dar"uber.
Nach Definition 2.4 ist $\Gamma$ eine Jordan-Kurve.
Da nach Konstruktion keiner der Eigenwerte im Bild von
$\gamma$ liegt, ist $G$ auf der gesamten Kontur wohldefiniert.
%\begin{figure}[h!]
%	\center
%	\begin{tikzpicture}
%	\draw[->] (-3.5cm,0cm) -- (3.5cm,0cm) node[right,fill=white] {Re};
%    \draw[->] (0cm,-2.5cm) -- (0cm,2.5cm) node[above,fill=white] {Im};
%    \draw[->] (0cm, 0cm) -- (.7, .7);
%	\draw[red](0cm,0cm)circle(1cm);
%	\foreach \x in {0,0.66,2.33} {
%	\filldraw[black] (\x cm,0) circle(2pt);
%	}
%	\draw (0,-0.3) node{$\lambda_1$};
%	\draw (0.66,-0.3) node{$\lambda_2$};
%	\draw (2.33,-0.3) node{$\lambda_3$};
	%\node[rotate=45] at (0.5, 1) {$r=1$};
	%\node at (2.3,-2.3) {$\C$};
	%\end{tikzpicture}
	%\caption{Skizze der Kurve \textcolor{red}{$\gamma$} in der komplexen Ebene.}
%\end{figure}
Folgen wir also weiter dem Abschnitt \ref{sec:kontur}, so erhalten wir
\[
\frac{1}{2\pi\iota} \int_\Gamma G \text{ d}s =
\frac{1}{2\pi\iota}\int_0^{2\pi} G(\gamma(\omega))\cdot \gamma'(\omega)
\text{ d}\omega
= \text{diag}\left( 0, \frac{1}{5}, \frac{1}{9} \right)
\]

Diese Matrix l"asst sich mit den $B$-orthonormalen Vektoren $u_1 = 1/\sqrt5 \cdot e_2$
und $u_2 = 1/3\cdot e_3$ wie gew"unscht in
\[
\begin{bmatrix} 0 & 0 & 0 \\ 0 & 1/6 & 0 \\ 0 & 0 & 1/9 \end{bmatrix}
= \begin{bmatrix} 0 & 0  \\ 1/\sqrt6 & 0  \\ 0 & 1/3  \end{bmatrix}
\begin{bmatrix} 0 & 1/\sqrt6 & 0 \\ 0 & 0 & 1/3 \end{bmatrix} =: U_2 U_2^H
\]
faktorisieren. Nachdem dies geschafft ist, lassen wir uns von der in Abschnitt \ref{sec:ritz} diskutierten Vorgehensweise inspirieren und reduzieren das Problem zun"achst mit
Hilfe einer vollrangigen Matrix $V\in\C^{3,2}$ auf ein Problem kleinerer Dimension.
Zu diesem Zwecke w"ahlen wir
\[
V:=\begin{bmatrix} 0 & 1 \\ 1 & 0 \\ 0 & 1\end{bmatrix}
\]
und benutzen die Matrix $Q:=U_2 U_2^H V$ um das Ausgangsproblem vem"oge
$\widetilde{A}:= Q^H A Q$ und $\widetilde{B}:=Q^H B Q$ auf das Problem
\[
\widetilde{A}y = \begin{bmatrix}1/36 & 0\\0 & 1/9 \end{bmatrix} y = \mu \begin{bmatrix}5/36 & 0\\0 & 1/4 \end{bmatrix} y = \mu \widetilde{B}y
\]
der Dimension $(2\times 2)$ zu transformieren. Die Eigenwerte $1/5$ und $4/9$ lassen sich leicht ablesen
und stimmen wie erwartet mit den sich auf $[-1,1]$ befindlichen Eigenwerten des eingangs formulierten Problems "uberein. Passende Eigenvektoren sind
Elemente aus $\spn_\C \{[1 \ 0]^T\}$ und $\spn_\C \{[0 \ 1]^T\}$.\\

Es bleibt die R"ucktransformation der Eigenvektoren. Dazu w"ahlen wir aus kosmetischen Gr"unden die Eigenvektoren
\[
y_1 := \begin{bmatrix}
\sqrt6 \\ 0
\end{bmatrix}
\text{ und }
y_2 := \begin{bmatrix}
0 \\ 3
\end{bmatrix}
\]
und setzen $Y = [y_1 \ y_2]$. Wir erhalten dann
\[
QY = \begin{bmatrix} 0 & 0 \\ 1/6 & 0 \\ 0 & 1/9 \end{bmatrix}
\begin{bmatrix}
\sqrt6 & 0 \\ 0 & 3
\end{bmatrix}
=
\begin{bmatrix} 0 & 0  \\ 1/\sqrt6 & 0  \\ 0 & 1/3  \end{bmatrix}
= U_2.
\]

Eine Bemerkung zum Abschluss: Erinnern wir und an Satz \ref{thm:invariant}, so ist es nicht verwunderlich, dass diese Rechnung
exakte Eigenpaare liefert. Schlie"slich hat $Q$
vollen Rang und das Bild der von $Q$ induzierten linearen Abbildung ist invariant unter $B^{-1}A$.


%dem Projektor $P:= X_2 X_2^T B$ transformieren.\\
%Zu l"osen ist daher das Eigenwertproblem
%\[
%P^T A P x = \text{diag}(0,4,0)x = \lambda \cdot \text{diag}(0,6,1)x = \lambda P^T B P x.
%\]


%Da wir bereits wissen, dass wir zwei Eigenpaare finden wollen, w"ahlen wir eine
%beliebige Matrix $\widetilde{B}\in\C^{3,2}$ vollen Ranges -- die wir in diesem
%Beispiel mit $\widetilde{B} = \begin{bmatrix}[5]$
