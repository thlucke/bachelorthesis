Ein M"oglicheit das eingangs geschilderte Eigenwertproblem zu l"osen, ist
das Anwenden einer Klasse von Verfahren, die sich \emph{Rayleigh-Ritz-Verfahren}
nennen. Die Idee besteht darin die Eigenr"aume/ den Eigenraum \textcolor{red}{(was nun genau? bekomme ich alle
Eigenr"aume? nur einen? was genau approximiere ich?)} durch eine Folge von Unterr"aumen
zu approximieren, die durch orthogonale Projektionen konstruiert werden.\textcolor{red}{was ist genau
mit approximieren gemeint? in welchem sinne? und ist das "uberhaupt korrekt?} In diesem
Abschnitt wird die mathematische Idee dieser Projektionsverfahren vorgestellt und
die Verwendung von Filtern motiviert \textcolor{red}{kam die motivation nicht schon
in der einleitung?}

\section{Orthogonale Projektionsverfahren}
\footnote{Dieser Abschnitt orientiert sich in Formulierung und Dramaturgie an Abschnitt 4.3.1 aus Y. Saad Solving large evp. Hier verallgemeinern wir aber die Konzepte direkt auf allg. ewp.}

Wir konzentrieren uns zun"achst auf das spezielle Eigenproblem $Ax = \lambda x$ \textcolor{red}{variablen wieder neu einf"uhren? oder in der einleitung: im folgenden sei stets...} und betrachten f"ur ein $m\in\N$ mit $m\le n$
einen $m$-dimensionalen Unterraum $V\subset\Cn$. Dieser zun"achst nicht
n"aher bestimmte \emph{Suchraum} wird als Grundlage f"ur das Projektionsverfahren gew"ahlt.
Gesucht sind nun Paare $(\widetilde{\lambda}, \widetilde{x})\in\C\times V$ -- die wir als approximierte L"osungen des Eigenproblems verstehen wollen --
welche die Eigenschaft
\begin{equation}\label{eq:orthogonal}
\langle A\widetilde{x} - \widetilde{\lambda}\widetilde{x}, v\rangle=0
\end{equation}
f"ur alle $v\in V$ erf"ullen. Das Residuum $A\widetilde{x} - \widetilde{\lambda}\widetilde{x}$
soll also orthogonal auf dem Suchraum stehen. \textcolor{red}{residuum bzgl was?}\\

Angenommen, eine Orthonormalbasis $\{v_i\}_{i=1:m}\subset V$ ist gegeben.
Dann l"asst sich die Forderung \eqref{eq:orthogonal} mit Hilfe der Matrix $V_m :=[v_i]_{i=1:m}\in\C^{n,m}$, einem geeigneten Vektor
$y\in\C^m$ und der Substitution $V_m y=\widetilde{x}$ in das Gleichungssystem
\[
V_m^H(AV_m y - \widetilde{\lambda} V_m y) = 0
\]
"uberf"uhren. Als direkte Konsequenz dieser Substitution erhalten wir unter Ausnutzung der Orthogonalit"at der Spalten von $V_m$ die Bedingung
\[
V_m^H A V_m y = \widetilde{\lambda}y.
\]
\textcolor{red}{WIE DAS ALLGEMEINE PROBLEM EINBRINGEN? WO IST DANN B?}
