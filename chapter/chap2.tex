Ein M"oglicheit das eingangs geschilderte Eigenwertproblem zu l"osen, ist
das Anwenden einer Klasse von Verfahren, die sich \emph{Rayleigh-Ritz-Verfahren}
nennen. Die Idee besteht darin die Eigenr"aume/ den Eigenraum \textcolor{red}{(was nun genau? bekomme ich alle
Eigenr"aume? nur einen? was genau approximiere ich?)} durch eine Folge von Unterr"aumen
zu approximieren, die durch orthogonale Projektionen konstruiert werden.\textcolor{red}{was ist genau
mit approximieren gemeint? in welchem sinne? und ist das "uberhaupt korrekt?} In diesem
Abschnitt wird die mathematische Idee dieser Projektionsverfahren vorgestellt und
die Verwendung von Filtern motiviert \textcolor{red}{kam die motivation nicht schon
in der einleitung?}

\section{Orthogonale Projektionsverfahren}
\footnote{Dieser Abschnitt orientiert sich in Formulierung und Dramaturgie an Abschnitt 4.3.1 aus Y. Saad Solving large evp. Hier verallgemeinern wir aber die Konzepte direkt auf allg. ewp.}
Bevor wir das allgemeine Eigenproblem untersuchen, konzentrieren wir uns zun"achst auf das gew"ohnliche Eigenproblem
\[
Ax = \lambda x% \textcolor{red}{variablen wieder neu einf"uhren? oder in der einleitung: im folgenden sei stets...}
\]
und betrachten f"ur eine Zahl $m\in\N$ mit $m\le n$
einen $m$-dimensionalen Unterraum $V\subset\Cn$. Dieser zun"achst nicht
n"aher bestimmte \emph{Suchraum} wird als Grundlage f"ur das Projektionsverfahren gew"ahlt.
Gesucht sind nun Paare $(\widetilde{\lambda}, \widetilde{x})\in\C\times V$ -- die wir als approximierte L"osungen des Eigenproblems verstehen wollen --
welche die Eigenschaft
\begin{equation}\label{eq:orthogonal}
\langle A\widetilde{x} - \widetilde{\lambda}\widetilde{x}, v\rangle=0
\end{equation}
f"ur alle $v\in V$ erf"ullen.\footnote{Paare $(\widetilde{\lambda}, \widetilde{x})$ der
eben beschriebenen Art werden auch \emph{Ritz-Paare} genannt.}
Das Residuum $A\widetilde{x} - \widetilde{\lambda}\widetilde{x}$
soll also orthogonal auf dem Suchraum stehen.\\%residuum bzgl was???

Angenommen, eine Orthonormalbasis $\{v_i\}_{i=1:m}\subset V$ ist gegeben.
Dann l"asst sich die Forderung \eqref{eq:orthogonal} mit Hilfe der Matrix $\pmb{V_m} :=[v_i]_{i=1:m}\in\C^{n,m}$, einem geeigneten Vektor
$y\in\C^m$ und der Substitution $V_m y=\widetilde{x}$ in das Gleichungssystem
\[
V_m^H(AV_m y - \widetilde{\lambda} V_m y) = 0
\]
"uberf"uhren. Als direkte Konsequenz dieser Substitution erhalten wir unter Ausnutzung der Orthogonalit"at der Spalten von $V_m$ mit
\begin{equation}\label{eq:transform}
V_m^H A V_m y = \widetilde{\lambda}y.
\end{equation}
ein neues Eigenproblem. Jedes Eigenpaar $(\widetilde{\lambda},y)$ von \eqref{eq:transform}
liefert dann ein Ritz-Paar $(\widetilde{\lambda}, V_m y)$ des gew"ohnlichen
Eigenproblems bez"uglich des Suchraums $V$. Man kann zeigen, dass im Falle der Invarianz
von $V$ unter $A$ jedes Ritz-Paar von $A$ sogar ein Eigenpaar ist. \textcolor{red}{beweis eventuell dazu nehmen}\\

Ausgehend von diesen "Uberlegungen l"asst sich das Rayleigh-Ritz-Verfahren wie folgt
beschreiben:

\begin{lstlisting}[caption = Rayleigh-Ritz-Verfahren (Vgl. Saad Algo. 4.5), captionpos=b]
Berechne eine ONB {v1,...,vm} von V und setze Vm = [v1,...,vm]
Berechne Eigenwerte von Vm^H * A * Vm und wähle die gewünschten aus
Berechne die zu den eben ermittelten Eigenwerte zugehörigen EVektoren
Berechne die Ritz Paare
\end{lstlisting}

Die folgende Proposition weist nach, dass der eben vorgestellte Algorithmus tats"achlich
ein Projektionsverfahren ist.

\begin{prop}
Die durch die Matrix $V_m V_m^H \in \Cnn$ induzierte lineare Abbildung
\[
P\colon \Cn \to \Cn, x\mapsto V_m V_m^H x
\]
ist eine orthogonale Projektion auf den Unterraum V und l"ost das Minimierungsproblem
\[
\min_{y\in V}\|x-y\|.
\]
\end{prop}

\begin{proof}
Sei $x\in\Cn$ vorgegeben. Dann gilt $P(x) = V_m V_m^H x \in \Bild(V_m) = V$ nach Konstruktion.
Da aus der Orthogonalit"at der Spalten von $V_m$
\[
P^2 = V_m V_m^H V_m V_m^H = V_m V_m^H = P
\]
folgt, ist $P$ somit eine Projektion auf $V$. Dar"uber hinaus gilt f"ur alle
Vektoren $y\in V$
\[
\langle x-P(x), y\rangle = y^H (x - V_m V_m^H x) = y^H (I - V_m V_m^H)x
= y^H x - y^H V_m V_m^H x = y^H x - y^H x = 0
\]
\textcolor{red}{Bem.: $y^H V_m V_m^H = P(y)^H = y^H$, da $y \in V$.}
Also ist $x-P(x) \in V^{\bot}$ und $P$ tats"achlich eine orthogonale Projektion.
Wegen $P(x)-y \in V$ folgt die Minimalit"at aus der Absch"atzung
\begin{align*}
\|x-y\|^2 &= \|x-P(x) + P(x)-y\|^2 \\
&= \langle x-P(x) + P(x)-y, x-P(x) + P(x)-y \rangle \\
&= \|x-P(x)\|^2 + \langle x-P(x), P(x)-y \rangle + \langle P(x)-y, x-P(x)\rangle + \|P(x)-y\|^2 \\
&= \|x - P(x)\|^2 + \| P(x)-y \|^2\\
&\ge \|x-P(x)\|^2.
\end{align*}
Damit ist die Behauptung bewiesen, da Gleichheit nur mit $y=P(x)$ folgen kann.
\end{proof}
Projektio $P=V$... siehe skript liesen

Auf diesen "Uberlegungen aufbauend, l"asst sich nun das folgende Verfahren konstruieren.
blabla saad... s. 98.

\textcolor{red}{WIE DAS ALLGEMEINE PROBLEM EINBRINGEN? WO IST DANN B? Direkt allgemein machen?}
