Dieses Kapitel widmet sich der Frage, ob und wie man aus der Menge aller Eigenpaare
des Problems
\[
Ax = \lambda Bx
\]
eine gew"unschte Teilmenge ausw"ahlen kann. Dazu werden im Rahmen dieser Arbeit aus dem Katalog der Verfahren
die zwei Folgenden ausgew"ahlt und vorgestellt: das Rayleigh-Ritz-Verfahren / Galerkin-Methoden
\footnote{Galerkin und Rayleigh ritz literatur blabla} und Konturintegration.

Ob ein Zusammenhang zwischen beiden Konzepten besteht, wird im n"achsten Kapitel gekl"art.\\

%Ein M"oglicheit das eingangs geschilderte Eigenwertproblem zu l"osen, ist
%das Anwenden einer Klasse von Verfahren, die sich \emph{Rayleigh-Ritz-Verfahren}
%nennen. Die Idee besteht darin die Eigenr"aume/ den Eigenraum \textcolor{red}{(was nun genau? bekomme ich alle
%Eigenr"aume? nur einegn? was genau approximiere ich?)} durch eine Folge von Unterr"aumen
%zu approximieren, die durch orthogonale Projektionen konstruiert werden.\textcolor{red}{was ist genau
%mit approximieren gemeint? in welchem sinne? und ist das "uberhaupt korrekt?} In diesem
%Abschnitt wird die mathematische Idee dieser Projektionsverfahren vorgestellt und
%die Verwendung von Filtern motiviert \textcolor{red}{kam die motivation nicht schon
%in der einleitung?}

\section{Rayleigh-Ritz-Verfahren / Galerkin}
%\footnote{Dieser Abschnitt orientiert sich in Formulierung und Dramaturgie an Abschnitt 4.3.1 aus Y. Saad Solving large evp. Hier verallgemeinern wir aber die Konzepte direkt auf allg. ewp.}
Eine \textcolor{red}{Die?} Idee dieser Klasse von Verfahren ist das Approximieren des von
den gesuchten Eigenvektoren aufgespannten Unterraums. %Dazu sollen -- wie der Name vermuten l"asst --
%Projektionen bem"uht werden.\\
Bevor wir das allgemeine Eigenwertproblem untersuchen, wenden wir unsere Aufmerksamkeit
dem gew"ohnlichen Eigenwertproblem
\[
Ax = \lambda x% \textcolor{red}{variablen wieder neu einf"uhren? oder in der einleitung: im folgenden sei stets...}
\]
zu und betrachten f"ur eine Zahl $m\in\N$ mit $m\le n$
einen $m$-dimensionalen Unterraum $\U_m\subseteq\Cn$. Dieser zun"achst nicht
n"aher bestimmte \emph{Suchraum} wird als Grundlage f"ur das Verfahren gew"ahlt.
Gesucht sind nun Paare $(\widetilde{\lambda}, \widetilde{x})\in\C\times \U_m \setminus\{0\}$ --
die wir als approximierte L"osungen des Eigenproblems verstehen wollen --
welche die Eigenschaft
\begin{equation}\label{eq:orthogonal}
\langle u, A\widetilde{x} - \widetilde{\lambda}\widetilde{x} \rangle=0
\end{equation}
f"ur alle $u \in \U_m $ erf"ullen. Das Residuum $A\widetilde{x} - \widetilde{\lambda}\widetilde{x}$
soll also orthogonal auf dem Suchraum stehen. Paare, die diesem Anliegen nachkommen,
werden auch \emph{Ritz-Paare} bez"uglich des Suchraums $\U_m$ genannt.\\

Wir wollen nun annehmen, dass mit der Menge von Vektoren $\{u_i\}_{i=1:m}\subseteq \Cn$ eine Orthonormalbasis
des Unterraums $\U_m$ gegeben ist. Definieren wir dann die Matrix $U_m :=[u_i]_{i=1:m}\in\C^{n,m}$, so muss wegen $\widetilde{x}\in\U_m$
ein Vektor $y\in\C^m$ existieren, mit $U_m y = \widetilde{x}$. Die Forderung \eqref{eq:orthogonal}
ist dann zu der Gleichung
\[
U_m^H(AU_m y - \widetilde{\lambda} U_m y) = 0
\]
"aquivalent. Als direkte Konsequenz dieser Umformulierung erhalten wir unter Ausnutzung
der Orthogonalit"at der Spalten von $U_m$ mit
\begin{equation}\label{eq:transform}
(U_m^H A U_m) y = \widetilde{\lambda}y
\end{equation}
ein neues Eigenwertproblem. Jedes Eigenpaar $(\widetilde{\lambda},y)$ von \eqref{eq:transform}
liefert dann nach Konstruktion mit $(\widetilde{\lambda}, U_m y)$ ein Ritz-Paar des gew"ohnlichen
Eigenwertproblems bez"uglich des Suchraums $\U_m$. In Abh"angigkeit von der Wahl
des Suchraumes variiert nat"urlich G"ute der Approximation. Ist etwa $\U_m$
durch Eigenvektoren aufgespannt, so ist jedes Ritz-Paar schon ein Eigenpaar.
Wir werden sp"ater aus den Untersuchungen des allgemeinen Eigenwertproblems folgern,
dass bereits im Falle der Invarianz von $\U_m$ unter $A$ jedes Ritz-Paar von $A$ bereits ein Eigenpaar ist.
An dieser Stelle begn"ugen wir uns vorerst mit dieser erfreulichen Botschaft.\\

%Fassen wir die eben geschilderte Vorgehensweise algorithmisch zusammen, so ist
%der folgende Pseudocode denkbar:
Zusammengefasst ergibt sich mit den obigen "Uberlegungen f"ur die Berechnung von
Ritz-Paaren folgender Algorithmus \textcolor{red}{(XXX)}:
\begin{itemize}
\item berechne ONB $\{u_i\}_{i=1:m}$ von $\U_m$, setze $U_m := [u_i]_{i=1:m}$
\item setze $\widetilde{A} := U_m^H AU_m$ und l"ose $\widetilde{A} y_i = \lambda_i y_i$, $y_i \in \Cn, i=1:m$
\item $x_i := U_m y_i$
\item gib Ritz-Paare $(\lambda_i, x_i)$ aus
\end{itemize}

Die Idee dieser Methode ist also simpel: Transformiere das Eigenwertproblem mit
einer gewissen Matrix in ein anderes Eigenwertproblem und benutze dessen L"osungen,
um Ritz-Paare des urspr"unglichen Problems zu erhalten.
Doch wozu die M"uhe, das urspr"ungliche Eigenwertproblem in ein anderes
Eigenwertproblem zu "uberf"uhren? Zwar gelingt es, aus dem transformierten Problem
\eqref{eq:transform} Ritz-Paare zu extrahieren, aber w"are nicht auch denkbar,
s"amtliche Eigenpaare von $A$ zu approximieren und die zum Unterraum $\U_m$
korrespondierende Teilmenge direkt auszuw"ahlen?\\

Dies mag in Einzelf"allen in der Tat sinnvoller sein. Sprechen wir allerdings
von Matrixdimensionen jenseits der Vorstellungskraft, ist eine vollst"andige
Berechnung aller Eigenpaare mitunter ein sehr zeitintensives Vergn"ugen. Bei
genauerer Betrachtung der Gleichung \eqref{eq:transform} f"allt auf, dass die
Matrix $(U_m^H A U_m) \in \C^{m,m}$ im Falle $n \gg m$ ein mitunter deutlich kleineres
Format hat, als die Matrix $A$ im urspr"unglichen Problem. Man darf hier also erwarten,
dass die ben"otigte Laufzeit zur Bestimmung der Ritz-Paare mit dem Algorithmus \textcolor{red}{(XXX)}
geringer ist, als beim Approximieren s"amtlicher Eigenpaare von $A$. Dies
wird im vierten Kapitel anhand ausgew"ahlter Beispiele illustriert.\\

Wir gehen nun einen Schritt weiter und erweitern die obige Theorie auf das
verallgemeinerte Eigenwertproblem
\begin{equation}\label{chap2:gevp}
Ax = \lambda Bx.
\end{equation}
Dabei gehen wir ganz analog zum gew"ohnlichen Eigenwertproblem vor.
Es sei daher wieder $\U_m\subseteq \Cn$ ein $m$-dimensionaler Suchraum.
Gefunden werden sollen nun Paare $ (\widetilde{\lambda}, \widetilde{x}) \in \C
\times \U_m \setminus\{ 0\}$ die der Orthogonalit"atsbedingung
\begin{equation}\label{eq:borthogonal}
A\widetilde{x} - \widetilde{\lambda}B\widetilde{x} \ \bot \ \U_m
\end{equation}
gen"ugen. Wieder substituieren wir $U_m y=\widetilde{x}$ und erhalten die
zu \eqref{eq:borthogonal} "aquivalente Forderung
\[
U_m^H(AU_m y - \widetilde{\lambda} BU_m y) = 0.
\]
Schlie"slich l"osen wir das transformierte Eigenwertproblem
\begin{equation}\label{eq:transformedevp}
(U_m^H AU_m) y = \widetilde{\lambda} (U_m^H B U_m) y.
\end{equation}
Wie bereits beim gew"ohnlichen Eigenwertproblem kann aus jeder L"osung $(\widetilde{\lambda}, y)$ von \eqref{eq:transformedevp}
mit $(\widetilde{\lambda}, U_m y)$ ein Ritz-Paar f"ur das verallgemeinerte Eigenwertproblem
gewonnen werden.\\

Mit dieser Konstruktion l"asst sich der Algorithmus \textcolor{red}{(XXX)}
wie folgt abwandeln:

\begin{itemize}
\item berechne ONB $\{u_i\}_{i=1:m}$ von $\U_m$, setze $U_m := [u_i]_{i=1:m}$
\item setze $\widetilde{A} := U_m^H AU_m$, $\widetilde{B}:= U_m^H BU_m$ und l"ose
$\widetilde{A} y_i = \lambda_i \widetilde{B}y_i$, $y_i \in \Cn, i=1:m$
\item $x_i := U_m y_i$
\item gib Ritz-Paare $(\lambda_i, x_i)$ aus
\end{itemize}

Nun, da die Idee der Verfahren vorgestellt wurde, wollen wir einige Beobachtungen
festhalten. Eingangs wurde behauptet, dass die Invarianz des Suchraumes $\U_m$ unter
$A$ dazu f"uhrt, dass jedes Ritz-Paar bez"uglich $\U_m$ ein Eigenpaar von $A$ ist.
Doch warum ist das so? Dies zu beantworten verpflichtet sich der folgende Satz.

\begin{thm}\label{thm:invariant}
Neben zwei Matrizen $A,B\in\Cnn$ -- wobei $B$ eine HPD-Matrix ist -- sei f"ur
$m\in\N$ mit $m\le n$ ein $m$-dimensionaler Unterraum $\U_m \subseteq \C_n$ gegeben.
Ist dieser invariant unter $B^{-1}A$, so ist jedes Ritzpaar von $B^{-1}A$
bez"uglich $\U_m$ auch ein verallgemeinertes Eigenpaar von $A$.
\end{thm}

\begin{proof}
Beginnen wir mit einer ONB $\{u_i\}_{i=1:m}\subseteq\U_m$ des Unterraums $\U_m$
und setzen wie bisher $U_m := [u_i]_{i=1:m}\in\C^{n,m}$. Aufgrund der Invarianz
von $\U_m$ unter $B^{-1}A$ muss eine Matrix $V_m \in \C^{m,m}$ existieren, welche
die Gleichung
\begin{equation}\label{eq:thminvariant}
B^{-1}A U_m = U_m V_m
\end{equation}
erf"ullt. Insbesondere folgt unter Ausnutzung der Orthogonalit"at der Spalten
von $U_m$ die Identit"at $U_m^H B^{-1}A U_m = V_m$.
Sind nun $\lambda\in\C$ und $y\in\C^m\setminus\{0\}$ so gew"ahlt, dass $(\lambda, U_m y)$
ein Ritz-Paar von $B^{-1}A$ ist, so folgt aus
\[
U_m^H B^{-1}A U_m y = V_m y
\]
mit \eqref{eq:thminvariant} die Gleichung
\[
B^{-1}AU_m y = U_m V_m y = \lambda U_m y
\]
und schlie"slich auch die Behauptung durch Umstellen.
\end{proof}

Da das gew"ohnliche Eigenwertproblem ein Spezialfall des allgemeinen Eigenwertproblems
ist, k"onnen wir aus dem eben bewiesenen Resultat unmittelbar das folgende Korollar
ableiten.

\begin{kor}
Ist $A\in\Cnn$ und $\U_m\subseteq \Cn$ ein $m$-dimensionaler Unterraum, so ist
jedes Ritz-Paar von $A$ bez"uglich $\U_m$ ein Eigenpaar von $A$.
\end{kor}

\begin{proof}
Betrachte f"ur $B:=I_n$ das verallgemeinerte Eigenwertproblem
\[
Ax = \lambda Bx = \lambda x.
\]
Die Aussage folgt dann aus dem vorigen Satz.
\end{proof}

Diese Erkenntnis ist vor allem aus algorithmischer Sicht h"ochst n"utzlich. Implementiert
man das Rayleigh-Ritz-Verfahren f"ur das gew"ohnliche Eigenwertproblem iterativ \textcolor{red}{(siehe Quelle blablabla)}, so
kann der Iterationsprozess abgebrochen werden, sobald die Unterraumiterierte $\Bild(U_m^(i))$
invariant unter $A$ ist.\\

Das Rayleigh-Ritz-Verfahren steht in einem engen Zusammenhang zu Projektionsverfahren.
Mit einer geeigneten Projektionsmatrix $P\in\Cnn$ ist ein Paar $(\lambda, x)\in\C\times\Co$
\textcolor{red}{(oder $\in\C\times\U_m\setminus\{0\}$?)} genau dann eine L"osung
der Gleichung \eqref{chap2:gevp}, wenn es eine L"osung von
\[
P^H APx = \lambda P^H BPx
\]
ist. Richtig definiert, besitzt $P$ noch weitere n"utzliche Eigenschaften, die
der folgende Satz vorstellt.

\begin{thm}\label{thm:projektor}
Es sei $B\in\Cnn$ eine HPD-Matrix und f"ur $m\in\N$ mit $m\le n$ sei
ein $m$-dimensionaler Unterraum $\U_m \subseteq \Cn$ gegeben. Sei weiter $\{u_i\}_{i=1:m}\subseteq\U_m$ eine
Basis $B$-orthogonaler Vektoren, das hei"st, die Matrix $U_m := [u_i]_{i=1:m}
\in\C^{n,m}$ erf"ulle die Gleichung $U_m^H B U_m = I_m$. Dann ist die von der Matrix
 $P := U_m U_m^H B \in \Cnn$ induzierte lineare Abbildung
\[
p \colon \Cn \to \Cn, x\mapsto U_m U_m^H Bx
\]
eine $B$-orthogonale Projektion auf den Unterraum $\U_m$. Au"serdem gilt
f"ur alle $x\in\Cn$ die Identit"at
\[
\|x-p(x)\|_B = \min_{y\in\U} \|x - y\|_B, x\in\Cn.
\]
\end{thm}

\begin{proof}
Aus der $B$-Orthogonalit"at von $U_m$ folgt f"ur alle $x\in\Cn$
\[
p^2 (x) = U_m U_m^H B U_m U_m^H B x= U_m U_m^H Bx = p(x)
\]
und damit gilt die Mengengleichheit $p(\Cn) = \U$ nach Konstruktion. %($\Rang{P_B} = \dim(\U)$).
Ist nun $y\in\U$ und $x\in\Cn$, dann folgt wegen $p(y) = y$ und $B=B^H$ auch
\begin{align*}
\langle y, x-p(x)\rangle_B &= y^H (Bx - B U_m U_m^H Bx) \\
&= y^H (Bx - B^H U_m U_m^H Bx) \\
&= y^H Bx - p(y)^H Bx = 0.
\end{align*}
Es gilt demnach $x-p(x) \ \bot_B \ \U$. Die Optimierungsaufgabe wird wegen
\begin{align*}
\|x-y\|_B^2 &= \|x-p(x) + p(x)-y\|_B^2 \\
&= \|x-p(x)\|_B^2 + \|p(x)-y\|_B^2\\
&\ge \|x-p(x)\|_B^2
\end{align*}
gel"ost, denn aufgrund der positiven Definitheit von Normen, gilt Gleichheit genau dann,
wenn $p(x)=x$ erf"ullt ist.
\end{proof}

Mit Hilfe der im Satz eingef"uhrten Matrix $P$, l"asst sich die Gleichung
\eqref{eq:transformedevp} elegant umformulieren. Durch Linksmultiplikation
mit $B^H U_m$ und dem Einschub der Identit"at $U_m^H B U_m$ gilt n"amlich
\[
B^H U_m (U_m^H A U_m)(U_m^H B U_m)y = \lambda B^H U_m (U_m^H B U_m)(U_m^H B U_m)y
\]
und folglich
\[
P^H A P U_m y = \lambda P^H B P U_m y.
\]
Wenn wir uns an dieser Stelle erinnern, dass $x$ durch $U_m y$ ersetzt wurde, dann
erhalten wir tats"achlich
\[
P^H A P x = \lambda P^H B P x.
\]
Im Fall der gew"ohnlichen Eigenwertgleichung erhalten wir speziell
\[
P A P x = \lambda x.
\]

%\section{Analytische Konstruktion des Spektralprojektors}

Es ist leider nicht ohne Weiteres m"oglich einen bestimmten Eigenraum gezielt
zu approximieren, wenn keine weiteren Informationen bekannt sind. G"angige
Wahl f"ur Suchr"aume sind Krylow-R"aume. Deren Struktur f"uhrt dazu, dass
Unterr"aume approximiert werden, die von den Eigenvektoren mit den betragsm"a"sig
gr"o"sten Eigenwerten aufgespannt werden. Es obliegt daher dem pers"onlichen
Geschick einen passenden Suchraum aufzustellen.\\

Im kommenden Abschnitt wird daher eine Methode vorgestellt, mit dessen Hilfe gezielt
Eigenpaare gefunden werden k"onnen.

\textcolor{red}{anstelle einer basis von U kann auch beliebige matrix Q gew"ahlt
werden und dann diese auf Um projeziert werden mit $U:=U_m^H U_m B Q$ Dann
ist man fertig, aber Projektor ist i.A.nicht bekannt, daher approximieren. siehe
kapitel 3}

\section{Konturintegration}\label{sec:kontur}

Wir wollen nun anehmen, dass zwei Zahlen $\lambda_1, \lambda_2 \in \R$ vorgegeben sind, die
das abgeschlossene Intervall $I:=[\lambda_1, \lambda_2] \subseteq \R$ definieren.
F"ur zwei hermitesche Matrizen $A, B \in \Cnn$ mit der zus"atzlichen
Forderung, dass $B$ positiv definit ist, sollen Paare der Gestalt
$(\lambda, x) \in I \times \C^n$ ermittelt werden, welche der verallgemeinerten
Eigenwertgleichung

  \begin{equation}\label{eq:eigen} %eventuell einfaches ewp
  Ax = \lambda Bx
  \end{equation}

gen"ugen. Um diese Eigenpaare\footnote{Um der besseren Lesbarkeit Willen werden
im Folgenden die verallgemeinerten Eigenvektoren und deren verallgemeinerte
Eigenwerte kurz als Eigenvektor und Eigenwert bezeichnet.}
bestimmen zu k"onnen, f"uhren wir die durch%wird das Problem mit Hilfe einer
%gewissen Matrix $Q\in\Cnn$ in das "aquivalente Problem

%  \begin{equation}\label{eq:Qeigen}
%  A_Q \phi = \mu B_Q \phi
%  \end{equation}

%"uberf"uhrt. Hierbei sind $A_Q = Q^TAQ$, $B_Q = Q^TBQ$ und $(\mu, \phi)
%\in I\times\C^n$.
%Wird die Matrix $Q$ richtig gew"ahlt, so ist dann jeder zul"assige Eigenwert von
%\eqref{eq:Qeigen} auch ein zul"assiger Eigenwert von \eqref{eq:eigen} und
%umgekehrt.\footnote{Daher ist die Angabe $\mu\in I$ gerechtfertigt.}
%F"ur die Ermittlung der gesuchten Eigenvektoren bedarf es hingegen zus"atzlicher Arbeit.\\

%Zur Bestimmung der Transformationsmatrix $Q$ wird eine Konturintegration bem"uht.
%Ausgangspunkt dieser Integration ist die durch
  \begin{align*}
  G\colon\Omega &\to\Cnn\\
  \omega &\mapsto (\omega B - A)^{-1}
  \end{align*}

definierte \textsc{Green}-Funktion ein, wobei $\Omega \subseteq \C$ eine passende
Teilmenge der komplexen Zahlen ist. Insbesondere m"ussen $\Omega$ und das
Spektrum von $B^{-1}A$ disjunkt sein, da $G$ andernfalls nicht wohldefiniert ist.\\

Die Funktion $G$ wird nun "uber eine geschlossene Jordan-Kurve $\gamma$,
die um das vorgegebene Intervall $I$ heruml"auft, in der Gestalt
\[
-\frac{1}{2\pi\iota}\int_\gamma G(\omega)\text{ d}\omega
\]
integriert.\footnote{Mit $\iota$ ist im Folgenden stets die imagin"are Einheit bezeichnet,
also $\iota = \sqrt{-1}$.}\\

Wir wollen nun annehmen, dass es genau $k\in\N$ Eigenwerte gibt,
die im Inneren des Intervalls $I$ liegen.\footnote{Diese Forderung ist wohldefiniert, da
der Eigenschaften von $A$ und $B$ wegen alle Eigenwerte von $B^{-1}A$ reell sind.}
F"ur zu diesen Eigenwerten passende Eigenvektoren $\{x_i\}_{i=1:k} \subseteq \Cn$ sei au"serdem die Matrix
$X_k := [x_i]_{i=1:k}$ gegeben.
Dann l"asst die Konturintegration R"uckschl"usse auf die Eigenvektoren zu, denn mit
Mitteln der komplexen Analysis kann man
zeigen, dass die Identit"at
\begin{equation}\label{eq:integral}
X_k X_k^T = \sum_{i=1}^k x_i x_i^T =
-\frac{1}{2\pi\iota}\int_\gamma G(\omega)\text{ d}\omega
\end{equation}
gilt. \textcolor{red}{genauer? Wie sieht X genau aus?}\\

Wir w"ahlen nun $k$ linear unabh"angige Vektoren $\{y_i\}_{i=1:k} \subseteq \Cn$
aus, die wir in der Matrix $Y_k := [ y_i ]_{i=1:k}$ zusammenfassen.
Unsere gew"unschte Matrix $Q$ definieren wir nun durch Matrixmultiplikation
verm"oge
\begin{equation}\label{eq:Q}
Q := X_k X_k^T Y_k.
\end{equation}
Die Spalten von $Q$ sind also gerade Linearkombinationen aus den Eigenvektoren.
Sind dann $\{\phi_i\}_{i=1:k} \subseteq \Cn$ Vektoren die das transformierte
Eigenwertproblem \eqref{eq:Qeigen} l"osen
und ist $\Phi_k:=[\phi_i]_{i=1:k}$ gegeben,
so gilt $X_k = Q \Phi_k$.\\

In der Praxis sind nat"urlich weder die Anzahl der Eigenwerte noch die dazugeh"origen Eigenvektoren
bekannt. W"ahrend die Anzahl mit Hilfe einer passenden Filterfunktion bestimmt werden kann,
verhilft die wegen \eqref{eq:integral} und \eqref{eq:Q} g"ultige Gleichheit
\begin{equation}\label{eq:solution}
Q = -\frac{1}{2\pi\iota}\left(\int_\gamma G(\omega)\text{ d}\omega\right)\cdot Y
\end{equation}
zur Berechnung der Eigenvektoren.

%B-Orthogonalit"at der x-vektoren
\newpage
\section{Beispiel}\label{sec:bsp}

In diesem Abschnitt soll mit Hilfe eines einfachen Beispiels die eben erarbeitete
Theorie verifiziert werden. Daf"ur betrachten wir die beiden Matrizen
$A,B \in \Cnn$, welche durch
\[
A:= \begin{bmatrix} 0 & 0 & 0 \\ 0 & 1 & 0 \\ 0 & 0 & -1 \end{bmatrix} \text{ und }
B:= \begin{bmatrix} 1/2 & 0 & 0\\ 0 & 1/2 & 0 \\ 0 & 0 & 1/4 \end{bmatrix}
\]
gegeben sind. Durch einfaches Nachrechnen "uberpr"uft man, dass das
verallgemeinerte Eigenwertproblem \eqref{eq:eigen} durch Vektoren
$x_1 \in \spn_\C\{e_1\}$ mit zugeh"origem Eigenwert $\lambda_1 = 0$, Vektoren
$x_2 \in \spn_\C\{e_2\}$ mit zugeh"origem Eigenwert $\lambda_2 = 2$, sowie
Vektoren $x_3 \in\spn_\C\{e_3\}$ mit zugeh"origem Eigenwert $\lambda_3 = -4$
gel"ost wird.\footnote{Hier bezeichnen $e_1, e_2, e_3 \in\Cn$ die kanonischen Einheitsvektoren.}\\

Wir wollen nun auf dem reellen Intervall $[-3,3]$ die Eigenpaare bestimmen.
Als Integrationskontur w"ahlen wir daher den Kreis mit einem Radius von drei
L"angeneinheiten und dem Mittelpunkt im Ursprung. Das hei"st, wir werden die
\textsc{Green}-Funkion $G$ "uber die Kurve
\begin{equation}\label{kurve}
\gamma\colon [0,2\pi]\to\Cn\text{, }\varphi\mapsto 3e^{\iota \varphi}
\end{equation}
integrieren. Da nach Konstruktion keiner der Eigenwerte auf dem
Graphen von $\gamma$ liegt, ist $G$ auf der gesamten Kontur wohldefiniert.

\begin{figure}[h!]
	\center
	\begin{tikzpicture}
	\draw[->] (-3.5cm,0cm) -- (3.5cm,0cm) node[right,fill=white] {Re};
    \draw[->] (0cm,-2.5cm) -- (0cm,2.5cm) node[above,fill=white] {Im};
    \draw[->] (0cm, 0cm) -- (1.41, 1.41);
	\draw[red](0cm,0cm)circle(2cm);
	\foreach \x in {-3,0,1.33} {
	\filldraw[black] (\x cm,0) circle(2pt);
	}
	\draw (0.3,-0.3) node{$\lambda_1$};
	\draw (1.33,-0.3) node{$\lambda_2$};
	\draw (-3,-0.3) node{$\lambda_3$};
	\node[rotate=45] at (0.5, 1) {$r=3$};
	%\node at (2.3,-2.3) {$\C$};
	\end{tikzpicture}
	\caption{Skizze der Kurve \textcolor{red}{$\gamma$} in der komplexen Ebene.}
\end{figure}

Zun"achst "uberpr"ufen wir
die Identit"at \eqref{eq:integral}. Mit $x_1 := \begin{bmatrix} \sqrt2 \iota & 0 & 0
\end{bmatrix}^T$
und $x_2 := \begin{bmatrix} 0 & \sqrt2 \iota &0\end{bmatrix}^T$ sind zwei passende
Eigenvektoren f"ur unser Eigenwertproblem gegeben, und in der Tat gilt
\begin{align*}
-\frac{1}{2\pi\iota} \int_\gamma G(\omega) \text{ d}\omega &=
-\frac{1}{2\pi\iota}\int_0^{2\pi} G(\gamma(\omega))\cdot \gamma'(\omega)
\text{ d}\omega \\
&= -\frac{1}{2\pi\iota}\int_0^{2\pi} \begin{bmatrix} \frac{2}{3e^{\iota\omega}}&0&0\\
0 & \frac{2}{3e^{\iota\omega}-2}&0\\
0&0&\frac{4}{3e^{\iota\omega}+4}
 \end{bmatrix}\cdot 3\iota e^{\iota\omega}
\text{ d}\omega \\
&= \text{diag}(-2,-2,0) \\
&= x_1 x_1^{T} + x_2 x_2^T.
\end{align*}

Als N"achstes "uberpr"ufen wir, ob die Matrix $X_2 := \begin{bmatrix} x_1 & x_2\end{bmatrix}$ mit Hilfe der in Abschnitt \ref{sec:math} beschriebenen Methode rekonstruierbar ist.\\

Dort wird von uns zun"achst die Konstruktion der Matrix $Q$ aus \eqref{eq:Q} verlangt.
Da wir bereits wissen, dass wir zwei Eigenvektoren finden wollen, w"ahlen wir also beliebig
eine Matrix $Y_2 \in \C^{3,2}$ vollen Ranges -- die wir in diesem
Beispiel mit $Y_2 := \begin{bmatrix} e_1 & e_1 + e_2 \end{bmatrix}$ festlegen wollen --
und multiplizieren diese von rechts an das Integral der \textsc{Green}-Funktion "uber
die durch \eqref{kurve} definierte Kontur.\footnote{Es mag den Leser verwundern,
wie bei Unkenntnis "uber die Anzahl $k\in\N$ der zu findenden Eigenvektoren die
Spaltenzahl von $Y_k$ korrekt zu ermitteln ist. Eine g"angige Methode ist zum Beispiel das Konstruieren einer
passenden Filterfunktion, welche die nicht erw"unschten Eigenwerte aussortiert (Vgl. Abschnitt \ref{sec:rkt}).}
Wir erhalten somit
\[
Q = -\frac{1}{2\pi\iota}\left(\int_{\gamma}G(\omega)\text{ d}\omega \right)\cdot Y_2
= \begin{bmatrix} -2 & 0&0 \\ 0 & -2& 0\\ 0&0&0\end{bmatrix}
\begin{bmatrix} 1 & 1 \\ 0 & 1\\0&0 \end{bmatrix}
= \begin{bmatrix} -2 & -2 \\ 0 & -2\\0&0 \end{bmatrix}.
\]
Nun l"osen wir das in \eqref{eq:Qeigen} beschriebene
Eigenwertproblem. Wir suchen also Paare $(\mu,\phi) \in \C\times\Cn$ die
der Gleichung
\[
A_Q \phi = \mu B_Q \phi
\]
gen"ugen. Durch Nachrechnen "uberzeugt man sich davon, dass dieses von allen
Vektoren $\phi_1 \in \spn_\C\{e_1\}$ mit dem Eigenwert $\mu_1 = 0$ sowie
allen Vektoren $\phi_2 \in \spn_\C\{-e_1 + e_2\}$ mit dem Eigenwert
$\mu_2 = 2$ gel"ost wird.\\

Die soeben ermittelten Eigenwerte stimmen also schon mit den Eigenwerten
unseres betrachteten Problems "uberein. W"ahlen wir schlie"slich die Matrix
\[
\Phi_2 := \begin{bmatrix} \phi_1 & \phi_2 \end{bmatrix} :=
\begin{bmatrix} -\iota/\sqrt2  & \iota/\sqrt2 \\ 0 & -\iota/\sqrt2 \end{bmatrix},
\]
mit den Eigenvektoren $\phi_1$ und $\phi_2$, so erhalten wir wie gew"unscht
\[
Q\Phi_2 = \begin{bmatrix} -2 & -2 \\ 0 & -2\\0&0 \end{bmatrix}
\begin{bmatrix} -\iota/\sqrt2  & \iota/\sqrt2 \\ 0 & -\iota/\sqrt2 \end{bmatrix} =
\begin{bmatrix} \sqrt2 \iota & 0 \\ 0 & \sqrt2 \iota \\ 0 &0\end{bmatrix} = X_2.
\]

\textbf{Bemerkung.} An dieser Stelle sei darauf hingewiesen,
dass die Matrix $\Phi_2$ aus kosmetischen Gr"unden gerade so
gew"ahlt wurde, dass die Kalkulation am Ende $X_2$ ergibt und somit alles
h"ubsch anzusehen ist.
Der Wahl einer anderen Matrix $\Phi'_2$ mit skalaren Vielfachen der hier
gew"ahlten Eigenvektoren steht selbstverst"andlich nichts im Wege und f"uhrt freilich zu einer anderen
Matrix $X_2'$, deren Spalten ebenfalls zul"assige Eigenvektoren enth"alt.\\
