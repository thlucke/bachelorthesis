Nachdem vorgestellt wurde, wie mit Polynomen Eigenwerte extrahiert werden k"onnen,
verpflichtet sich diese Kapitel der Vorstellung einer anderen Klasse von Methoden, die
das Filtern von Eigenwerten erm"oglichen. Die Protagonisten dieser Methoden werden
durch sogenannte \emph{rationale Funktionen} verk"orpert. Ausgehend
von zwei Polynomfunktionen $p, q\colon{\R}\to\R$ mit
\[
p(t) := \sum_{k=0}^n p_k t^k \text{ \ und\ } q(t) := \sum_{k=0}^n q_k t^k
\]
f"ur reelle/rationale? Koeffizienten $(p_i)_{i=1:n}$ und $(q_i)_{i=1:n}$
definieren wir eine \emph{rationale Funktion} $f\colon\R\to\R$ verm"oge
\[
f(t) := \frac{p(t)}{q(t)}
\]
wobei freilich darauf geachtet werden muss, dass zu keiner Zeit durch Null
geteilt wird.
