Nachdem das vorangegangene Kapitel verschiedene Ideen zum Filtern von Eigenpaaren theoretisch beleuchtet hat, werden wir uns nun mit der Frage der praktischen Umsetzbarkeit besch"aftigen.
Im Mittelpunkt wird dabei die Konstruktion geeigneter Suchr"aume stehen, welche im Rayleigh-Ritz-Verfahren zum Einsatz kommen sollen.
Es wird sich zeigen, dass die Konturintegration hierbei ein n"utzliches Hilfsmittel darstellt.


\section{Beschleunigte Rayleigh-Ritz Iteration}\label{chap3:beschrr}

Zu Beginn des zweiten Kapitels wurde mit dem Rayleigh-Ritz-Verfahren eine Methode vorgestellt, mit dessen Hilfe Ritz-Paare bez"uglich eines gewissen Suchraums bestimmt werden k"onnen.
Es wurde angedeutet, dass das Einbinden einer geeigneten Iterationsvorschrift zur Verringerung der Abweichungen von Ritz-Paaren zu tats"achlichen Eigenpaaren f"uhren kann.
Dies soll im Folgenden genauer ausgef"uhrt werden.\\

\newpage
Ausgangspunkt f"ur unsere Betrachtungen ist nachwievor das hermitesche Eigenwertproblem
\begin{equation}\label{eq:chap3ewp}
Ax = \lambda Bx.
\end{equation}
In seinen Abhandlungen "uber Unterraumiterationen
schl"agt Y. Saad in ~\cite[118 f.]{saad} folgende Methode vor.\footnote{Der Algorithmus wurde vom Autor frei in die deutsche Sprache "ubersetzt und die Notation im Sinne der Konsistenz dieser Arbeit angepasst.}

\begin{algorithm}\label{alg:chap3rriter}
\caption{Rayleigh-Ritz-Iteration}\label{euclid}
\begin{algorithmic}[1]
\State W"ahle $m$ Zufallsvektoren $V_{(0)} \gets [v_i]_{i=1:m} \in\C^{n,m}$.
Setze $k \gets 1$ und w"ahle initialen Exponenten $p_{(1)} \in\N$.
\State \textbf{repeat}
\State \ \ \ \ Setze $\hat{Q}_{(k)} \gets A^{p_{(k)}} V_{(k-1)}$.
\State \ \ \ \ Orthonormalisiere die Spalten von $\hat{Q}_{(k)}$ zu $Q_{(k)}$.
\State \ \ \ \ Setze $\widetilde{A}_{(k)} \gets Q_{(k)}^H A Q_{(k)}$.
\State \ \ \ \ Berechne die Schurvektoren $\widetilde{X}_{(k)} \gets [x_i]_{i=1:m}$ von $\widetilde{A}$ mit Hilfe des QR-Algorithmus'.
\State \ \ \ \ Setze $V_{(k)} \gets Q_{(k)}\widetilde{X}_{(k)}$.
\State \ \ \ \ $k \gets k+1$.
\State \textbf{until} Verfahren konvergiert.
\end{algorithmic}
\end{algorithm}

Hierbei ist $(p_{(k)})_{k\in\N}$ eine Folge von nat"urlichzahligen Exponenten auf die wir sp"ater eingehen werden. Zuvor
begutachten wir die Zeilen vier bis sieben. Hier ist eine deutliche N"ahe zu dem Rayleigh-Ritz-Verfahren zu erkennen.
Als Suchraum wird die lineare H"ulle der Spaltenvektoren von $\hat{Q}_{(k)}$ benutzt und die Berechnung von Ritz-Vektoren erfolgt in Schritt sechs -- schlie"slich stimmen unserer Voraussetzungen wegen die Schurvektoren gerade mit den Eigenvektoren von $\widetilde{A}_{(k)}$ "uberein.
Wir verzichten an dieser Stelle auf Ausf"uhrungen "uber die Funktionsweise des \emph{QR-Algorithmus'}. Der interessierte Leser wende sich zum Beispiel an ~\cite[55 ff.]{stewart}.\\

Nun zur Folge der Exponenten. Im Spezialfall, dass $p_{(k)} = k$ f"ur alle $k\in\N$ gilt, ist obiger Algorithmus stark mit der \emph{Potenzmethode} zur Bestimmung von Eigenpaaren verwandt, ja sogar eine Verallgemeinerung davon.\footnote{Vgl. ~\cite[115 ff.]{saad} und ~\cite[1 f.]{kpt}.}
W"urden wir die Exponenten tats"achlich so w"ahlen, kann die Anzahl der Orthonormalisierungen im vierten Schritt sehr gro"s werden.
Dies l"asst sich aber vermeiden.
\textcolor{red}{man will nicht dauern orthogonalisieren, daher blabla s. 116 in saad}


%Diese Methode l"asst sich nicht nur als direktes Verfahren implementieren, sondern ist auch iterativ umsetzbar. Doch wie genau geht diese Iteration vonstatten? Wor"uber wird iteriert?\\

%Die folgenden Abs"atze
%an den Ausf"uhrungen
%von Ping Tak Peter Tang und Eric Polizzi in ~\cite{ptep}. Allerdings wird
%die Notation im Sinne der Konsistenz dieser Arbeit an einigen Stellen abweichen.\\

%Betrachten wir zur Beantwortung dieser Fragen wie bisher das verallgemeinerte Eigenwertproblem mit zwei
%komplexwertigen, hermiteschen $(n\times n)$-Matrizen $A$ und $B$ und fordern
%zus"atzlich die positive Definitheit von $B$. Zu diesem Duo gesellt sich nun
%mit $\p(B^{-1}A)$ ein Polynom in $B^{-1}A$, welches wir benutzen, um den Algorithmus \ref{alg:grp} aus dem vorigen Kapitel gem"a"s dem
%oben zitierten Paper wie folgt zu "andern.

\begin{algorithm}\label{alg:beschlrr}
\caption{Beschleunigtes iteratives Rayleigh-Ritz-Verfahren}\label{euclid}
\begin{algorithmic}[1]
\State W"ahle $m$ Zufallsvektoren $V_{(0)} \gets [v_i]_{i=1:m} \in\C^{n,m}$.
Setze $k \gets 1$.
\State \textbf{repeat}
\State \ \ \ \ Approximiere den Unterraumprojektor: $Q_{(k)} \gets \p(B^{-1}A)V_{(k-1)}$
\State \ \ \ \ Reduziere die Dimension: $\widetilde{A}_{(k)} \gets Q_{(k)}^H A Q_{(k)}$,
$\widetilde{B}_{(k)} \gets Q_{(k)}^H B Q_{(k)}$.
\State \ \ \ \ L"ose das transformierte Problem $\widetilde{A}_{(k)}\widetilde{X}_{(k)}
= \widetilde{B}_{(k)}\widetilde{X}_{(k)}\widetilde{\Lambda}_{(k)}$ in
$\widetilde{X}_{(k)}$ und $\widetilde{\Lambda}_{(k)}$.
\State \ \ \ \ Setze $V_{(k)} \gets Q_{(k)}\widetilde{X}_{(k)}$.
\State \ \ \ \ $k \gets k+1$.
\State \textbf{until} Abbruchkriterium ist erf"ullt.
\end{algorithmic}
\end{algorithm}



Die obigen Fragen sind also leicht beantwortet: In dieser Variante des Rayleigh-Ritz-Verfahrens wird solange "uber den Suchraum iteriert, bis die ermittelten Ritz-Paare gewisse Anforderungen erf"ullen.
Es dr"angt sich jedoch eine weitere Frage auf. Welche Funktion erf"ullt das Polynom $\p$?\\

Im Kontext dieses Algorithmus' wird $\p$ auch als \emph{Filter} oder \emph{Beschleuniger}
bezeichnet. Von dessen Wahl h"angt n"amlich ab, ob und wie gut Eigenpaare approximiert
werden: Sei $[\lambda_1,\lambda_2]$ ein reelles Intervall, auf dem $l\in\N$ Eigenwerte und die korrespondierenden Eigenvektoren gefunden werden k"onnen. Ist nun $\p(B^{-1}A)$ der Spektralprojektor,
$m=l$ und hat die Matrix $Q_{(1)} = \p(B^{-1}A) V_{(0)}$ vollen Rang, so konvergiert der Algorithmus \ref{alg:beschlrr} in einer Iteration
(Vgl. ~\cite[356]{ptep}).
Dies folgt unter Ausnutzung der Invarianz des Bildes von $Q_{(1)}$ unter $B^{-1}A$
aus dem Satz $\ref{thm:invariant}$.\\

Da der Spektralprojektor in den meisten F"allen unbekannt sein d"urfte, liegt
die Idee nahe, ihn wenigstens zu approximieren. Tang und Polizzi ~\cite[356]{ptep} merken an, dass dies gut funktioniert, falls $\p$ eine durch \emph{Gau"s-Legendre-Quadratur} konstruierte rationale Funktion ist.
Um den Gedankengang der Autoren nachvollziehen zu k"onnen, wird sich das folgende Intermezzo mit der Auffrischung des Konzeptes von Quadraturformeln und rationalen Funktionen besch"aftigen. Im Anschluss fahren wir mit der
Konstruktion des Projektors fortfahren.

\section{Gau"s'sche Quadratur}

Um ein Integral numerisch zu approximieren, bedient man sich sogenannter \emph{Quadraturformeln}. Dazu betrachten wir eine stetige Funktion $f\colon\R\to\R$, welche wir auf einem gegebenen Intervall $I:=[a,b]\subset\R$ integrieren wollen.\footnote{Im Allgemeinen ist die Stetigkeit von $f$ nicht zwingend erforderlich. Wir werden uns hier der Einfachheit halber auf stetige Funktionen einschr"anken.}
Zu gegebenen St"utzpunkten $(x_i, f(x_i))_{i=0:n}$ auf $I\times\R$ sei $p_n$ das zugeh"orige \emph{Interpolationspolynom} vom Grad $n$, also ein Polynom, welches $p(x_i) = f(x_i)$ f"ur alle $i$ mit $0\le i\le n$ erf"ullt.\\

Dann bezeichnen wir die N"aherung
\begin{equation}\label{eq:quadratur}
Q_n(f) := \int_a^b p_n (x)\text{ d}x =
(b-a)\sum_{i=0}^n \omega_i f(x_i)
\end{equation}
als \emph{interpolatorische Quadraturformel}. Dabei gilt
\[
\omega_k = \int_0^1 \prod_{j=0,j\neq k}^n
\frac{t-t_j}{t_k - t_j} \text{ d}t, \ t_j
= \frac{x_j-a}{b-a}.
\]
Die Qualit"at der Approximation, also die Abweichung vom exakten Integral, h"angt ma"sgeblich von der Wahl und Anzahl der St"utzpunkte ab. Wollten wir
beispielsweise das Integral einer konstanten Funktion berechnen, so erschiene es wenig plausibel, anstelle der direkten Berechnung ein Polynom vom Grad 69 auf 70 St"utzstellen f"ur die Approximation zu bem"uhen.\\

Bei der Anwendung von Gau"s-Legendre-Quadraturen ergibt sich die Wahl der St"utzpunkte durch die Berechnung von Nullstellen von Polynomen, die in einer Orthogonalit"atsbeziehung zueinander stehen.
Wir werden gleich formal formulieren, wie dies zu verstehen ist.\\

Ausgangspunkt f"ur die Integration ist nun eine stetige Funktion $f$, die eine Faktorisierung in zwei stetige Funktionen $g$ und $\omega$ der Art
\[
f = g\cdot \omega
\]
besitzt, wobei $\omega$ auf dem Integrationsintervall $[a,b]$ positiv sein soll.\footnote{Im unstetigen Fall darf die Funktion $\omega$ in h"ochstens endlich vielen Punkten negativ sein.} Ziel ist daher die Berechnung von
\begin{equation}\label{eq:gintegral}
\int_a^b g(x)\omega(x) \text{ d}x.
\end{equation}
Wir fordern nun, dass \eqref{eq:quadratur} mit \eqref{eq:gintegral} f"ur alle Polynome bis zum Grad $(2n-1)$ "ubereinstimmt.
Dazu betrachten wir die Standardbasis $\{x^i\}_{i=1:(2n-1)}$ auf dem Raum der Polynome vom
Grad $2n-1$. Dann landen wir unweigerlich bei dem
Gleichungssystem
\[
\sum_{j=1}^n \omega_j x_j^k = \int_a^b x^k \omega(x) \text{ d}x \text{ mit } k = 0,\ldots,2n-1.
\]
Man kann zeigen, dass die L"osung dieses Systems durch Nullstellen eines Polynoms gegeben ist, welches durch
ein Gram-Schmidt-Orthogonalisierungsverfahren bez"uglich des Skalarproduktes
\[
\langle p,q\rangle_\omega := \int_a^b p(t) q(t)\omega(t) \text{ d}t
\]
konstruiert wurde. Das hei"st konkret: Ausgehend vom Polynom $p_0 \equiv 1$ ist
\[
p_n(x) := x^n - \sum_{j=0}^{n-1} \frac{\langle x^n, p_j \rangle_\omega}{\langle p_j, p_j\rangle_\omega} p_j (x)
\]
gerade dasjenige Polynom, durch dessen Nullstellen das obige Gleichungssystem gel"ost wird. Sind nun
$x_1,\ldots,x_n$ die Nullstellen dieses $n$-ten Orthogonalit"atspolynoms, so he"ist die numerische
Integrationsformel
\[
Q_n(f) = \sum_{j=1}^n \omega_j f(x_j) \text{ mit }
\omega_j = \langle L_j, 1 \rangle_\omega
= \int_a^b L_j(x)\p(x\text{ d}x
\]
\emph{Gau"s'sche Quadraturformel der $n$-ten Ordnung}.
Dabei ist
\[
L_j(x) = \prod_{k\neq j=1}^n \frac{x-x_j}{x_k - x_j}.
\]

\section{Approximation des Spektralprojektors}

Nachdem wir diese Kurzzusammenfassung des Gau"s-Quadratur-Verfahrens diskutiert haben,
widmen wir uns wieder der Approximation des Spektralprojektors
$P = X_k X_k^H B$. Dabei "ubernehmen wir die am Anfang dieses Kapitels eingef"uhrte Notation, sowie die an die Matrizen $A$ und $B$ gestellten Voraussetzungen.\\

Wenden wir uns daher wieder dem reellen Intervall $I := [\lambda_1, \lambda_2]$ zu. Das Ziel ist
die Konstruktion einer rationalen Funktion $\r\colon\C\to\C$ mit $\r(\R) \subseteq \R$,
die auf $I$ n"aherungsweise der Indikatorfunktion von $I$ entspricht. Dazu
bem"uhen wir die Cauchy'sche Integraldarstellung der Indikatorfunktion und
wandeln diese mit Hilfe numerischer Quadraturformeln in die gew"unschte
rationale Funktion $\r$ um.\\

Zu"achst zur Indikatorfunktion: Ist $c\in\R$ der Mittelpunkt des Intervalls $I$ und
$r$ der Abstand des Mittelpunktes zum Rand des Intervalls, dann entspricht die Menge
\[
\mathcal{C} := \{z\in\C : |z-c| = r\}
\]
gerade einer Sph"are mit Radius $r$ um $c$. Mit dem Cauchy'schen Integralsatz
l"asst sich zeigen, dass im Falle $z\notin \mathcal{C}$
\[
\frac{1}{2\pi\iota}\int_{ \mathcal{C}}\frac{1}{\omega-z}\text{ d}\omega
= \begin{cases}1 &\text{ falls }|z-c| < r \\ 0 &\text{ falls }|z-c| > 0 \end{cases}
\]
gilt. Dieses Integral gilt es nun zu approximieren.\textcolor{red}{wie genau?}(Vgl. ~\cite[20]{jaenich}).\\

Wir werden das eben diskutierte Integral mit einer Gau"s-Legendre Quadraturformel ann"ahern.
Dazu ist es n"otig die Kontur $\mathcal{C}$ so zu parametrisieren, dass eine Transformation
der Integrationsgrenzen auf das Intervall $[-1,1]$ m"oglichst einfach m"oglich ist. Zu diesem Zweck sei
\begin{align*}
\gamma\colon[-1,3]&\to\C \\
t&\mapsto c+re^{\iota \frac{\pi}{2}(1+t)}
\end{align*}
als Parametrisierung der Sph"are gew"ahlt. Die
Ableitung von $\gamma$ ist dann f"ur jedes $t\in[-1,3]$ durch
\[
\gamma'(t)=\iota \frac{\pi}{2}re^{\iota \frac{\pi}{2}(1+t)}
\]
gegeben. Mit den bekannten Regeln der Integration erhalten wir somit f"ur alle $z\notin\mathcal{C}$
\begin{align*}
\frac{1}{2\pi\iota}\int_{ \mathcal{C}}\frac{1}{\omega-z}\text{ d}\omega
&= \frac{1}{2\pi\iota} \int_{-1}^3 \frac{\gamma'(t)}{\gamma(t)-z}\text{ d}t \\
&= \frac{1}{2\pi\iota} \left( \int_{-1}^1 \frac{\gamma'(t)}{\gamma(t)-z} \text{ d}t +
\int_{1}^3\frac{\gamma'(t)}{\gamma(t)-z}\text{ d}t \right) \\
&= \frac{1}{2\pi\iota} \left( \int_{-1}^1 \frac{\gamma'(t)}{\gamma(t)-z} \text{ d}t +
\int_{-1}^1\frac{\gamma'(2-t)}{\gamma(2-t)-z}\text{ d}t \right) \\
&= \frac{1}{2\pi\iota} \int_{-1}^1 \left( \frac{\gamma'(t)}{\gamma(t)-z} +
\frac{\overline{\gamma'(t)}}{\overline{\gamma(t)}-z}\right)\text{d}t
\end{align*}
wobei $\overline{\gamma(t)}$ und $\overline{\gamma'(t)}$ die komplexen Konjugationen
von $\gamma(t)$ beziehungsweise $\gamma'(t)$ bezeichnen.\\

F"ur $q\in\N$ mit \textcolor{red}{wie gro"s ist $q$ genau?} seien $(w_j, t_j)_{j=1:q}$
die f"ur die Gau"s-Legendre-Quadratur ben"otigten Gewichte und Diskretisierungspunkte.
Dann setzen wir
\[
\rho(z) := \frac{1}{2\pi\iota}\sum_{j=1}^q \left(
\frac{w_j \cdot \gamma'(t_j)}{\gamma(t_j)-z} - \frac{w_j \cdot \overline{\gamma'(t_j)}}{\overline{\gamma(t_j)}-z}
\right)
\]
und erhalten nach der Substitution $\gamma(t_j) := \gamma_j$ und
$\sigma_j := w_j \gamma'(t_j) / (2\pi\iota)$ die gew"unschte rationale
Funktion
\[
\r\colon\C\to\C, z\mapsto\sum_{j=1}^q\left(\frac{\sigma_j}{\gamma_j - z} +
\frac{\overline{\sigma_j}}{\overline{\gamma_j} - z}\right)
\]
zur Approximation der Indikatorfunktion. Hierbei ist bemerkenswert, dass die
rationale Funktion bereits in Partialbruchzerlegung vorliegt. Setzen wir schlie"slich $B^{-1}A$ in die
rationale Funktion ein, so erhalten wir
\begin{align*}
\r(B^{-1}A) &= \sum_{k=1}^q \sigma_k (\gamma_k I - B^{-1}A)^{-1} +
\sum_{k=1}^q \overline{\sigma_k} (\overline{\gamma_k} I - B^{-1}A)^{-1}\\
&= \sum_{k=1}^q \sigma_k (\gamma_k B - A)^{-1} B +
\sum_{k=1}^q \overline{\sigma_k} (\overline{\gamma_k} B - A)^{-1} B
\end{align*}
und folglich
\[
\r(B^{-1}A)V =
\sum_{k=1}^q \sigma_k (\gamma_k B - A)^{-1} BV +
\sum_{k=1}^q \overline{\sigma_k} (\overline{\gamma_k} B - A)^{-1} BV
\]
f"ur eine Matrix $V\in\C^{n,q}$.

%Ausgehend von zwei Polynomen $p, q\in\C [t]$ mit
%\[
%p := \sum_{k=0}^n p_k t^k \text{ \ und\ } q := \sum_{k=0}^n q_k t^k
%\]
%definieren wir eine rationale Funktion $\rho\colon\C\setminus{N_q}\to\C$ verm"oge
%\[
%\r(t) := \frac{p(t)}{q(t)}
%\]
%und identifizieren wie "ublich die Unbestimmte $t$ mit den Argumenten von $p$ und $q$.
%Dabei ist $N_q := \{t\in\C \mid q(t) = 0\}$.
%\section{Der FEAST-Algorithmus}

%f"urderhin
