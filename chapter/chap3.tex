In diesem Kapitel soll demonstriert werden, dass Konturintegration und
Rayleigh-Ritz-Verfahren in einem engen Zusammenhang stehen \textcolor{red}{wird sich noch zeigen}.
Um dies zu Untermauern, orientieren sich die folgenden Abs"atze an den Ausf"uhrungen
von Ping Tak Peter Tang und Eric Polizzi in ~\cite{ptep}. Allerdings wird
die Notation im Sinne der Konsistenz an einigen Stellen abweichen.

\section{Beschleunigtes Rayleigh-Ritz Verfahren}\label{chap3:beschrr}

Betrachten wir also wie bisher das verallgemeinerte Eigenwertproblem mit zwei
komplexwertigen, hermiteschen $(n\times n)$-Matrizen $A$ und $B$ und fordern
zu"satzlich die positive Definitheit von $B$. Zu diesem Duo gesellt sich nun
mit $p(B^{-1}A)$ ein Polynom in $B^{-1}A$, welches wir benutzen um gem"a"s dem
oben zitierten Paper den Algorithmus () aus dem vorigen Kapitel wie folgt zu "andern.

\begin{algorithm}\label{alg:beschlrr}
\caption{Beschleunigtes iteratives Rayleigh-Ritz-Verfahren}\label{euclid}
\begin{algorithmic}[1]
\State W"ahle $m$ Zufallsvektoren $Q_{(0)} \gets [q_i]_{i=1:m} \in\C^{n,m}$.
Setze $k \gets 1$.
\State \textbf{repeat}
\State \ \ \ \ Approximiere den Unterraumprojektor: $Y_{(k)} \gets p(B^{-1}A)Q_{(k-1)}$
\State \ \ \ \ Reduziere die Dimension: $\widetilde{A}_{(k)} \gets Y_{(k)}^H A Y_{(k)}$,
$\widetilde{B}_{(k)} \gets Y_{(k)}^H B Y_{(k)}$.
\State \ \ \ \ L"ose das transformierte Problem $\widetilde{A}_{(k)}\widetilde{X}_{(k)}
= \widetilde{B}_{(k)}\widetilde{X}_{(k)}\widetilde{\Lambda}_{(k)}$ in
$\widetilde{X}_{(k)}$ und $\widetilde{\Lambda}_{(k)}$.
\State \ \ \ \ Setze $Q_{(k)} \gets Y_{(k)}\widetilde{X}_{(k)}$.
\State \ \ \ \ $k \gets k+1$.
\State \textbf{until} Abbruchkriterium ist erf"ullt.
\end{algorithmic}
\end{algorithm}

Das Polynom $p$ wird in diesem Kontext auch als \emph{Filter} oder \emph{Beschleuniger}
bezeichnet. Von dessen Wahl h"angt n"amlich ab, ob und wie gut Eigenpaare approximiert
werden. Es ist sogar m"oglich, gezielt solche Eigenpaare zu finden, wie sie im
Abschnitt \ref{sec:kontur} gesucht waren.\\

Um dies einzusehen, greifen wir erneut die Notationen aus besagtem Kapitel auf:
Es sei $[\lambda_1, \lambda_2]$ dasjenige Intervall, auf dem die Eigenwerte und
korresponierenden Eigenvektoren gefunden werden sollen und $X_k$ sei diejenige
Matrix dessen Spalten aus gerade diesen Eigenvektoren besteht. Es wurde bereits
diskutiert, dass die durch Konturintegration ermittelten Eigenvektoren
$B$-orthogonal sind. Damit l"asst sich also -- wie im Satz \ref{thm:projektor}
bewiesen -- der Spektralprojektor $P = X_k X_k^H B$ konstruieren. Falls nun
$p(B^{-1}A)$ mit diesem Projektor "ubereinstimmt, dann terminiert Algorithmus
\ref{alg:beschlrr} im Falle der Vollrangigkeit von $Y_{(1)}$ nach einer Iteration.\footnote{
Hierbei ist entscheidend, dass die Anzahl der Spalten von $Q$ mit der Anzahl der
Eigenpaare "ubereinstimmt, die auf dem Intervall zu finden sind (Vgl. ~\cite[356]{ptep}).}
Dies folgt unter Ausnutzung der Invarianz des Bildes von $PQ_{(0)}$ unter $B^{-1}A$
aus dem Satz $\ref{thm:invariant}$.\\

Da der Spektralprojektor in den meisten F"allen unbekannt sein d"urfte, liegt
die Idee nahe, ihn zumindest zu approximieren. Da in ~\cite[356]{ptep} bemerkt wird,
dass dies gut funktioniert, wenn $p$ eine durch \emph{Gau"s-Legendre-Quadratur}
konstruierte rationale Funktion ist, wird sich das folgende Intermezzo mit eben dieser
Klasse von Funktionen besch"aftigen, bevor wir mit der Konstruktion des Projektors
fortfahren.


\section{Rationale Funktionen}

Ausgehend von zwei Polynomen $p, q\in\C [t]$ mit
\[
p := \sum_{k=0}^n p_k t^k \text{ \ und\ } q := \sum_{k=0}^n q_k t^k
\]
definieren wir eine rationale Funktion $\rho\colon\C\setminus{N_q}\to\C$ verm"oge
\[
\rho(t) := \frac{p(t)}{q(t)}
\]
und identifizieren wie "ublich die Unbestimmte $t$ mit den Argumenten von $p$ und $q$.
Dabei ist $N_q := \{t\in\C \mid q(t) = 0\}$. Eine rationale Funktion wird als \emph{echt gebrochen} bezeichnet,
falls die Bedingung $\Grad(p) < \Grad(q)$ erf"ullt ist.



\section{Approximation des Spektralprojektors}

Nun da die f"ur den weiteren Verlauf der Arbeit wichtigen Eigenschaften rationaler
Funktionen wiederholt wurden, widmen wir uns der Approximation des Spektralprojektors
$P = X_k X_k^H B$. Daf"ur setzen wir den in Abschnitt \ref{chap3:beschrr} bereits begonnen Gedankengang aus ~\cite{ptep}
fort und "ubernehmen die zu letzt vereinbarten Voraussetzungen und Notationen.\\

Wenden wir uns also wieder dem reellen Intervall $I := [\lambda_1, \lambda_2]$ zu. Das Ziel ist
die Konstruktion einer rationalen Funktion $\rho\colon\C\to\C$ mit $\rho|_\R \subseteq \R$,
die auf $I$ n"aherungsweise der Indikatorfunktion von $I$ entspricht. Dazu
bem"uhen wir die Cauchy'sche Integraldarstellung der Indikatorfunktion und
wandeln diese mit Hilfe numerischer Quadraturformeln in die gew"unschte
rationale Funktion $\rho$ um.\\

Zu"achst zur Indikatorfunktion: Ist $c\in\R$ der Mittelpunkt des Intervalls $I$ und
$r$ der Abstand des Mittelpunktes zum Rand des Intervalls, dann entspricht die Menge
\[
\mathcal{C} := \{z\in\C : |z-c| = r\}
\]
gerade einer Sph"are mit Radius $r$ um $c$. Mit dem Cauchy'schen Integralsatz
l"asst sich zeigen, dass im Falle $z\notin \mathcal{C}$
\[
\frac{1}{2\pi\iota}\int_{\partial \mathcal{C}}\frac{1}{\omega-z}\text{ d}\omega
= \begin{cases}1 &\text{ falls }|z-c| < r \\ 0 &\text{ falls }|z-c| > 0 \end{cases}
\]
gilt. Dieses Integral gilt es nun zu approximieren.\textcolor{red}{wie genau?}(Vgl. ~\cite[20]{jaenich}).\\



\section{Der FEAST-Algorithmus}
