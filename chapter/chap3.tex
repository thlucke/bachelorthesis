Nachdem vorgestellt wurde, wie mit Polynomen Eigenwerte extrahiert werden k"onnen,
verpflichtet sich diese Kapitel der Vorstellung einer anderen Klasse von Methoden, die
das Filtern von Eigenwerten erm"oglichen. Die Protagonisten dieser Methoden sind
die sogenannten \emph{rationale Funktionen}. Ausgehend
von zwei Polynomen $p, q\in\C [t]$ mit
\[
p := \sum_{k=0}^n p_k t^k \text{ \ und\ } q := \sum_{k=0}^n q_k t^k
\]
definieren wir eine rationale Funktion $r\colon\C\setminus{N_q}\to\C$ verm"oge
\[
r(t) := \frac{p(t)}{q(t)}
\]
und identifizieren wie "ublich die Argumente von $p$ und $q$ mit der Unbestimmten $t$.
Eine rationale Funktion wird als \emph{echt gebrochen} bezeichnet, falls die
Bedingung $\Grad(p) < \Grad(q)$ erf"ullt ist.
\textcolor{red}{Nullstellen von $q$ m"ussen aus dem Definitionsbereich verschwinden.}\\

Wir erinnern uns an das Rayleigh-Ritz-Verfahren aus Kapitel 2. Die Konvergenz (?)
h"angt davon ab, mit welchem Startvektor das Verfahren beginnt...\\

Ein optimaler Filter ist der Spektralprojektor $\rho(B^{-1}A) = X_k X_k^H B$... warum, siehe PTEP S. 356.

\section{Konturintegration}
