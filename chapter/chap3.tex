In diesem Kapitel soll demonstriert werden, dass Konturintegration und
Rayleigh-Ritz-Verfahren in einem engen Zusammenhang stehen \textcolor{red}{wird sich noch zeigen}.
Um dies zu Untermauern, orientieren sich die folgenden Abs"atze an den Ausf"uhrungen
von Ping Tak Peter Tang und Eric Polizzi in ~\cite{ptep}. Allerdings wird
die Notation im Sinne der Konsistenz an einigen Stellen abweichen.

\section{Beschleunigtes Rayleigh-Ritz Verfahren}\label{chap3:beschrr}

Betrachten wir also wie bisher das verallgemeinerte Eigenwertproblem mit zwei
komplexwertigen, hermiteschen $(n\times n)$-Matrizen $A$ und $B$ und fordern
zu"satzlich die positive Definitheit von $B$. Zu diesem Duo gesellt sich nun
mit $p(B^{-1}A)$ ein Polynom in $B^{-1}A$, welches wir benutzen um gem"a"s dem
oben zitierten Paper den Algorithmus () aus dem vorigen Kapitel wie folgt zu "andern.

\begin{algorithm}\label{alg:beschlrr}
\caption{Beschleunigtes iteratives Rayleigh-Ritz-Verfahren}\label{euclid}
\begin{algorithmic}[1]
\State W"ahle $m$ Zufallsvektoren $Q_{(0)} \gets [q_i]_{i=1:m} \in\C^{n,m}$.
Setze $k \gets 1$.
\State \textbf{repeat}
\State \ \ \ \ Approximiere den Unterraumprojektor: $Y_{(k)} \gets p(B^{-1}A)Q_{(k-1)}$
\State \ \ \ \ Reduziere die Dimension: $\widetilde{A}_{(k)} \gets Y_{(k)}^H A Y_{(k)}$,
$\widetilde{B}_{(k)} \gets Y_{(k)}^H B Y_{(k)}$.
\State \ \ \ \ L"ose das transformierte Problem $\widetilde{A}_{(k)}\widetilde{X}_{(k)}
= \widetilde{B}_{(k)}\widetilde{X}_{(k)}\widetilde{\Lambda}_{(k)}$ in
$\widetilde{X}_{(k)}$ und $\widetilde{\Lambda}_{(k)}$.
\State \ \ \ \ Setze $Q_{(k)} \gets Y_{(k)}\widetilde{X}_{(k)}$.
\State \ \ \ \ $k \gets k+1$.
\State \textbf{until} Abbruchkriterium ist erf"ullt.
\end{algorithmic}
\end{algorithm}

Das Polynom $p$ wird in diesem Kontext auch als \emph{Filter} oder \emph{Beschleuniger}
bezeichnet. Von dessen Wahl h"angt n"amlich ab, ob und wie gut Eigenpaare approximiert
werden. Es ist sogar m"oglich, gezielt solche Eigenpaare zu finden, wie sie im
Abschnitt \ref{sec:kontur} gesucht waren.\\

Um dies einzusehen, greifen wir erneut die Notationen aus besagtem Kapitel auf:
Es sei $[\lambda_1, \lambda_2]$ dasjenige Intervall, auf dem die Eigenwerte und
korresponierenden Eigenvektoren gefunden werden sollen und $X_k$ sei diejenige
Matrix dessen Spalten aus gerade diesen Eigenvektoren besteht. Es wurde bereits
diskutiert, dass die durch Konturintegration ermittelten Eigenvektoren
$B$-orthogonal sind. Damit l"asst sich also -- wie im Satz \ref{thm:projektor}
bewiesen -- der Spektralprojektor $P = X_k X_k^H B$ konstruieren. Falls nun
$p(B^{-1}A)$ mit diesem Projektor "ubereinstimmt, dann terminiert Algorithmus
\ref{alg:beschlrr} im Falle der Vollrangigkeit von $Y_{(1)}$ nach einer Iteration.\footnote{
Hierbei ist entscheidend, dass die Anzahl der Spalten von $Q$ mit der Anzahl der
Eigenpaare "ubereinstimmt, die auf dem Intervall zu finden sind (Vgl. ~\cite[356]{ptep}).}
Dies folgt unter Ausnutzung der Invarianz des Bildes von $PQ_{(0)}$ unter $B^{-1}A$
aus dem Satz $\ref{thm:invariant}$.\\

Da der Spektralprojektor in den meisten F"allen unbekannt sein d"urfte, liegt
die Idee nahe, ihn zumindest zu approximieren. Da in ~\cite[356]{ptep} bemerkt wird,
dass dies gut funktioniert, wenn $p$ eine durch \emph{Gau"s-Legendre-Quadratur}
konstruierte rationale Funktion ist, wird sich das folgende Intermezzo mit eben dieser
Klasse von Funktionen besch"aftigen, bevor wir mit der Konstruktion des Projektors
fortfahren.


\section{Rationale Funktionen}

Ausgehend von zwei Polynomen $p, q\in\C [t]$ mit
\[
p := \sum_{k=0}^n p_k t^k \text{ \ und\ } q := \sum_{k=0}^n q_k t^k
\]
definieren wir eine rationale Funktion $\rho\colon\C\setminus{N_q}\to\C$ verm"oge
\[
\rho(t) := \frac{p(t)}{q(t)}
\]
und identifizieren wie "ublich die Unbestimmte $t$ mit den Argumenten von $p$ und $q$.
Dabei ist $N_q := \{t\in\C \mid q(t) = 0\}$. Eine rationale Funktion wird als \emph{echt gebrochen} bezeichnet,
falls die Bedingung $\Grad(p) < \Grad(q)$ erf"ullt ist.



\section{Approximation des Spektralprojektors}

Nun da die f"ur den weiteren Verlauf der Arbeit wichtigen Eigenschaften rationaler
Funktionen wiederholt wurden, widmen wir uns der Approximation des Spektralprojektors
$P = X_k X_k^H B$. Daf"ur setzen wir den in Abschnitt \ref{chap3:beschrr} bereits begonnen Gedankengang aus ~\cite{ptep}
fort und "ubernehmen die zu letzt vereinbarten Voraussetzungen und Notationen.\\

Wenden wir uns also wieder dem reellen Intervall $I := [\lambda_1, \lambda_2]$ zu. Das Ziel ist
die Konstruktion einer rationalen Funktion $\rho\colon\C\to\C$ mit $\rho|_\R \subseteq \R$,
die auf $I$ n"aherungsweise der Indikatorfunktion von $I$ entspricht. Dazu
bem"uhen wir die Cauchy'sche Integraldarstellung der Indikatorfunktion und
wandeln diese mit Hilfe numerischer Quadraturformeln in die gew"unschte
rationale Funktion $\rho$ um.\\

Zu"achst zur Indikatorfunktion: Ist $c\in\R$ der Mittelpunkt des Intervalls $I$ und
$r$ der Abstand des Mittelpunktes zum Rand des Intervalls, dann entspricht die Menge
\[
\mathcal{C} := \{z\in\C : |z-c| = r\}
\]
gerade einer Sph"are mit Radius $r$ um $c$. Mit dem Cauchy'schen Integralsatz
l"asst sich zeigen, dass im Falle $z\notin \mathcal{C}$
\[
\frac{1}{2\pi\iota}\oint_{ \mathcal{C}}\frac{1}{\omega-z}\text{ d}\omega
= \begin{cases}1 &\text{ falls }|z-c| < r \\ 0 &\text{ falls }|z-c| > 0 \end{cases}
\]
gilt. Dieses Integral gilt es nun zu approximieren.\textcolor{red}{wie genau?}(Vgl. ~\cite[20]{jaenich}).\\

Wir werden das eben diskutierte Integral mit einer Gau"s-Legendre Quadraturformel ann"ahern.
Dazu ist es n"otig die Kontur $\mathcal{C}$ so zu parametrisieren, dass eine Transformation
der Integrationsgrenzen auf das Intervall $[-1,1]$ m"oglichst einfach m"oglich ist. Zu diesem Zweck sei
\begin{align*}
\gamma\colon[-1,3]&\to\C \\
t&\mapsto c+re^{\iota \frac{\pi}{2}(1+t)}
\end{align*}
als Parametrisierung der Sph"are gew"ahlt. Die
Ableitung von $\gamma$ ist dann f"ur jedes $t\in[-1,3]$ durch
\[
\gamma'(t)=\iota \frac{\pi}{2}re^{\iota \frac{\pi}{2}(1+t)}
\]
gegeben. Mit den bekannten Regeln der Integration erhalten wir somit f"ur alle $z\notin\mathcal{C}$
\begin{align*}
\frac{1}{2\pi\iota}\oint_{ \mathcal{C}}\frac{1}{\omega-z}\text{ d}\omega
&= \frac{1}{2\pi\iota} \int_{-1}^3 \frac{\gamma'(t)}{\gamma(t)-z}\text{ d}t \\
&= \frac{1}{2\pi\iota} \left( \int_{-1}^1 \frac{\gamma'(t)}{\gamma(t)-z} \text{ d}t +
\int_{1}^3\frac{\gamma'(t)}{\gamma(t)-z}\text{ d}t \right) \\
&= \frac{1}{2\pi\iota} \left( \int_{-1}^1 \frac{\gamma'(t)}{\gamma(t)-z} \text{ d}t +
\int_{-1}^1\frac{\gamma'(2-t)}{\gamma(2-t)-z}\text{ d}t \right) \\
&= \frac{1}{2\pi\iota} \int_{-1}^1 \left( \frac{\gamma'(t)}{\gamma(t)-z} \text{ d}t +
\frac{\overline{\gamma'(t)}}{\overline{\gamma(t)}-z}\text{ d}t \right)
\end{align*}
wobei $\overline{\gamma(t)}$ und $\overline{\gamma'(t)}$ die komplexen Konjugationen
von $\gamma(t)$ beziehungsweise $\gamma'(t)$ bezeichnen.\\

F"ur $q\in\N$ mit \textcolor{red}{wie gro"s ist $q$ genau?} seien $(w_j, t_j)_{j=1:q}$
die f"ur die Gau"s-Legendre-Quadratur ben"otigten Gewichte und Diskretisierungspunkte.
Dann setzen wir
\[
\rho(z) := \frac{1}{2\pi\iota}\sum_{j=1}^q \left(
\frac{w_j \cdot \gamma'(t_j)}{\gamma(t_j)-z} - \frac{w_j \cdot \overline{\gamma'(t_j)}}{\overline{\gamma(t_j)}-z}
\right)
\]
und erhalten nach der Substitution $\gamma(t_j) := \gamma_j$ und
$\sigma_j := w_j \gamma'(t_j) / (2\pi\iota)$ die gew"unschte rationale
Funktion
\[
\rho\colon\C\to\C, z\mapsto\sum_{j=1}^q\left(\frac{\sigma_j}{\gamma_j - z} +
\frac{\overline{\sigma_j}}{\overline{\gamma_j} - z}\right)
\]
zur Approximation der Indikatorfunktion. Hierbei ist bemerkenswert, dass die
rationale Funktion bereits in Partialbruchzerlegung vorliegt.\\

Wir wollen uns nun davon "uberzeugen, dass der Gebrauch der eben erarbeiteten
Quadratur eine sinnvolle Wahl ist.

Erinnerung an Konturkapitel: ... weiter auf Seite 358
\section{Der FEAST-Algorithmus}
