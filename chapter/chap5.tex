Die folgende Proposition weist nach, dass der eben vorgestellte Algorithmus tats"achlich
ein Projektionsverfahren ist.

\begin{prop}\label{prop:projektor}
Die durch die Matrix $V_m V_m^H \in \Cnn$ induzierte lineare Abbildung
\[
P\colon \Cn \to \Cn, x\mapsto V_m V_m^H x
\]
ist eine orthogonale Projektion auf den Unterraum V und l"ost das Minimierungsproblem
\[
\min_{y\in V}\|x-y\|, x\in\Cn (\|x-P(x)\| = ...)
\]
\end{prop}

\begin{proof}
Sei $x\in\Cn$ vorgegeben. Dann gilt $P(x) = V_m V_m^H x \in \Bild(V_m) = V$ nach Konstruktion.
Da aus der Orthogonalit"at der Spalten von $V_m$
\[
P^2 = V_m V_m^H V_m V_m^H = V_m V_m^H = P
\]
folgt, ist $P$ somit eine Projektion auf $V$. Dar"uber hinaus gilt f"ur alle
Vektoren $y\in V$
\[
\langle y, x-P(x)\rangle = y^H (x - V_m V_m^H x) = y^H x - P(y)^H x = 0
\]
Also ist $x-P(x) \in V^{\bot}$ und $P$ tats"achlich eine orthogonale Projektion.
Wegen $P(x)-y \in V$ folgt die Minimalit"at aus der Absch"atzung
\begin{align*}
\|x-y\|^2 &= \|x-P(x) + P(x)-y\|^2 \\
&= \langle x-P(x) + P(x)-y, x-P(x) + P(x)-y \rangle \\
&= \|x-P(x)\|^2 + \langle x-P(x), P(x)-y \rangle + \langle P(x)-y, x-P(x)\rangle + \|P(x)-y\|^2 \\
&= \|x - P(x)\|^2 + \| P(x)-y \|^2\\
&\ge \|x-P(x)\|^2.
\end{align*}
Damit ist die Behauptung bewiesen, da Gleichheit nur mit $y=P(x)$ folgen kann.
\end{proof}

Wir wollen nun die eben erarbeitete Theorie auf das verallgemeinerte Eigenwertproblem
\[
Ax = \lambda Bx
\]
"ubertragen.
