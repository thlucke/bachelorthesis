Nachdem nun der Hauptteil der vorliegenden Arbeit beendet ist, sollen zum Abschluss die wichtigen Punkte dieser Arbeit zusammengefasst und bewertet werden. Neben dem Formulieren von offen gebliebenen Fragen sollen hier nochmal einige Kerngedanken aufbereitet werden und damit zur weiteren Besch"aftigung mit der Thematik einladen.\\

\glqq \emph{Weniger ist mehr}.\grqq\ Mit diesen Worten lockt das Titelblatt dieser Arbeit den Leser. Sieht man sich mit der Aufgabe konfrontiert, eine kleine Teilmenge aus der Gesamtheit aller Eigenpaare eines Eigenwertproblems herauszufiltern, so haben wir gesehen, dass es helfen kann, die Dimension des Eigenwertproblems zu verringern. In diesem Sinne bringt \emph{weniger} also in der Tat \emph{mehr}, da die Reduktion des Problems Zeitersparnis herbei f"uhren kann. Dies wurde im vorigen Kapitel illustriert.\\

Als ein Werkzeug zur Verkleinerung des Problems haben wir das Rayleigh-Ritz Verfahren kennengelernt, bei dem durch Multiplikation mit geeigneten Matrizen, welche sich durch das Erf"ullen einer gewissen Orthogonalit"atsbedingung ergeben, ein Herabsetzen der Dimension auf ganz nat"urliche Weise geschieht. Ist dies erledigt, so werden s"amtliche Eigenpaare des niedrigdimensionalen Problems bestimmt und in Ritz-Paare transformiert.\\

Es zeigte sich, dass die Wahl eines geeigneten Suchraums bei dieser Methode wesentlich zur Verwertbarkeit der Rechenergebnisse beitr"agt. Unter gewissen Voraussetzungen konvergierte beispielsweise das iterative Rayleigh-Ritz Verfahren in einem Schritt und lieferte direkt Eigenpaare. Es stellte sich als logische Konsequenz die Frage, ob g"unstige Suchr"aume im Vorfeld ermittelt werden k"onnen.\\

Hier eilten die Idee der Konturintegration und die Verwendung rationaler Funktionen zu Hilfe. Das Konturintegral erm"oglichte die Konstruktion des Spektralprojektors, mit dessen Hilfe invariante Suchr"aume generiert werden k"onnen. Rationale Funktionen wiederum spielten insbesondere f"ur die Implementation eine wichtige Rolle. Durch sie wurde es m"oglich, die Indikatorfunktion zu einem gegebenen Intervall zu approximieren. Zur Konstruktion besagter Funktionen wurde die RKToolbox herangezogen, die auf Grundlage von Erkenntnissen Zolotarevs entprechende Interpolanten erzeugt.

\newpage

Bei der Implementation der Algorithmen sowie der Auswertung der von diesen erzeugten Rechenergebnisse konnte eine Konsistenz zu den Resultaten aus der zitierten Literatur festgestellt werden. Dabei sei erneut daran erinnert, dass die geringen Stichprobengr"o"sen, die in dieser Arbeit verwendet wurden, verminderte Aussagekraft haben.\\

Es wurde nicht diskutiert, welche Verfahren konkret zur Bestimmung des Spektrums verwendet werden. Beim L"osen des niedrigdimensionalen Eigenwertproblems wurde stets auf die \mcode{eig}-Funktion MATLABs zur"uckgegriffen. Hier blieben Analysen unter Anwendung g"angiger Methoden aus.\\

Zum Schluss sei noch auf weiterf"uhrende Literatur hingewiesen. Wie bereits erw"ahnt, arbeiten Polizzi et al. an der Weiterentwicklung des FEAST-Algorithmus. LITERATUR::: Siehe sonst noch hier und da.
