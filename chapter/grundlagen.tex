\documentclass[a4paper, 11pt]{article}

\usepackage[ngerman]{babel}
\usepackage{amsmath, amsfonts, amssymb, amsthm}
\usepackage{geometry}
\usepackage[onehalfspacing]{setspace}

\usepackage{fontspec}
\setmainfont[
	BoldFont=MinionPro-Bold.otf,
	ItalicFont=Minion Pro Italic.ttf
	]{MinionPro-Regular.otf}

\setlength{\parindent}{0pt}

\newcommand{\Cnn}{\mathbb{C}^{n,n}}
\newcommand{\Cn}{\mathbb{C}^n}
\newcommand{\C}{\mathbb{C}}
\newcommand{\Co}{\mathbb{\C}\setminus\{0\}}
\newcommand{\R}{\mathbb{R}}
\newcommand{\GL}{\text{GL}_n (\C)}

\newtheorem{prop}{Proposition}

\begin{document}
Um das Lesen dieser Arbeit mehr zu einer Freude denn zu einer Schikane zu
machen, soll dieser Abschnitt einige Grundlagen der linearen Algebra und
der Funktionentheorie bereitstellen. Obschon sich der Autor bem"uht hat,
in der Literatur g"angige Notation zu benutzen, bittet er den
verst"andnissvollen Leser bei Unklarheiten im Anhang \glqq Notationen\grqq\
nachzuschlagen.\\

Beginnen wir mit Definitionen und Resultaten aus der Matrizentheorie.
Eine Matrix $A\in\Cnn$ wird als \emph{hermitesch} bezeichnet, falls
sie die Identit"at $A=A^H$ erf"ullt. Diese l"asst sich nach dem
\emph{Spektralsatz} unit"ar diagonalisieren. Das hei"st,
unter Aufbringung einiger kognitiven Anstregungen lassen sich ... finden.
\end{document}
