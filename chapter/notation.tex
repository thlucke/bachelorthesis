\begin{tabular}{ll}
$\N$ & Menge der nat"urlichen Zahlen \{1,2,3,\ldots\}\\
$\N_{0}$ & $\N\cup\{0\}$\\
$\R$ & Menge der reellen Zahlen\\
$\C$ & Menge der komplexen Zahlen\\
$\R^{m,n}$ & Menge der reellen $(m\times n)$-Matrizen\\
$\C^{m,n}$ & Menge der komplexen $(m\times n)$-Matrizen\\
$\R^n$ & Elemente aus $\R^{n,1}$\\
$\Cn$ & Elemente aus $\C^{n,1}$\\
$\delta_{i,j}$ & Kronecker-Delta $\delta_{i,j} = 1$, f"ur $i=j$ und $\delta_{i,j} = 0$, f"ur $i\neq j$\\
$I_n$ & Einheitsmatrix $[\delta_{i,j}]_{i,j=1:n}$\\
$A^H$ & Komplex konjugiert Transponierte der Matrix $A$\\
$\gg$ & Erheblich gr"o"ser als\\
$\bot$ & Steht orthogonal / unit"ar auf\\
$\bot_B$ & Steht $B$-orthogonal / $B$-unit"ar auf\\
$\U^{\bot_B}$ & $B$-orthogonales Komplement des Unterraums $\U$\\
$\langle x,y\rangle_B$ & Schreibweise f"ur $x^H B y$, f"ur eine HPD-Matrix $B$\\
$\|\cdot\|_B$ & Von der Matrix $B$ induzierte Norm: $\|x\|_B := \sqrt{x^H B x}$\\
$\|x\|_2$ & Die euklidische Norm $\sqrt{x^H x}$\\
$\mathfrak{Re}$ & Realteil komplexer Zahlen
\end{tabular}
