\begin{thm}[Singul"arwertzerlegung]\label{thm:appTheorems:svd}
Ist $A\in\C^{m,n}$ eine $r$-rangige Matrix, so existieren unit"are Matrizen $U\in\C^{m,m}, V\in\Cnn$
und eine Diagonalmatrix $\Sigma^2_+ =\text{diag}(\sigma^2_1,\ldots,\sigma^2_r)\in\C^{r,r}$ mit der Eigenschaft $\sigma_1 \ge \ldots \ge \sigma_r > 0$, die
eine Faktorisierung der Art
\[
A = U \Sigma V^H
\]
erm"oglichen. Dabei ist
\[
\Sigma = \begin{bmatrix} \Sigma_+ & 0_{r,n-r} \\
0_{m-r,r} & 0_{m-r,n-r} \end{bmatrix}.
\]
\end{thm}

\begin{thm}[Beste Rang-$k$ Approximierung]\label{thm:appTheorems:rang}
Ist $A\in\C^{m,n}$ eine $r$-rangige Matrix und $A = U\Sigma V^H$ mit $U=[u_i]_{i=1:m}$ und $V=[v_i]_{i=1:n}$ die Singul"arwertzerlegung von $A$ wie in Satz \ref{thm:appTheorems:svd}, dann gilt
\[
\left\|A-\sum_{i=1}^k \sigma_i u_i v_i^H\right\|_2 = \min_{\substack{B\in\C^{m,n} \\ \text{Rang}(B)=k}} \|A-B\|_2
\]
f"ur alle $k=1:r$.
\end{thm}

\begin{thm}[Spektralsatz f"ur hermitesche Matrizen]\label{thm:appTheorems:Spektralsatz}
Eine Matrix $A\in\Cnn$ ist genau dann hermitesch, wenn sie unit"ar diagonalisierbar ist und alle Eigenwerte von $A$ reell sind.
\end{thm}

\begin{thm}[Existenz der Cholesky-Zerlegung]\label{thm:appTheorems:Cholesky}
Ist $A\in\Cnn$ eine HPD-Matrix, so existiert eine eindeutig bestimmte untere Dreiecksmatrix $L\in\Cnn$ mit positiven Diagonalelementen, mit $A=LL^H$.
\end{thm}

%\begin{thm}[Jordan'sche Normalform]\label{thm:appTheorems:Jordan}
%blubb
%\end{thm}

\begin{thm}[Cauchy'scher Integralsatz]\label{thm:appTheorems:Cauchy}
Ist $\Omega\subseteq\C$ ein einfach zusammenh"angendes Gebiet, $f\colon\Omega\to\C$ eine holomorphe Funktion, sowie $\Gamma$ eine Jordan-Kurve, so gilt
\[
\oint_\Gamma f(z) \text{ d}z = 0.
\]
\end{thm}

\newpage

\begin{thm}[Potenzmethode]\label{thm:appTheorems:Potenzmethode} Es sei $A\in\Cnn$ eine hermitesche Matrix mit Eigenpaaren $(\lambda_i,x_i)_{i=1:n} \in \R\times\Co$ und $n\ge 2$.
Es gelte weiter $|\lambda_1| > |\lambda_j|$ f"ur alle
$j > 1$. Ist $y_0 \in\Co$, so konvergiert die Folge
$(y_k)_{k\in\N}$ mit
\[
y_{(k+1)} = \frac{1}{\|A^{k+1} y_{(0)}\|} A^{k+1}y_{(0)}
\]
gegen einen normierten Eigenvektor zum Eigenwert $\lambda_1$
\end{thm}

\begin{thm}[QR-Zerlegung]\label{thm:appTheorems:QR}
Sei $A\in\C^{n,m}$ mit $n\ge m$. Dann existieren eine unit"are Matrix $Q\in\Cnn$ und eine obere Dreiecksmatrix $\widetilde{R} = [r_{ij}]_{i,j=1:m}\in\C^{m,m}$ mit $r_{ii}\ge 0$ f"ur $i=1:m$ mit
\[
A = Q\begin{bmatrix} \widetilde{R} \\ 0_{n-m,m} \end{bmatrix} =: QR.
\]
\end{thm}
