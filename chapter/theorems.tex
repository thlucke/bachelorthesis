\begin{thm}[Singul"arwertzerlegung]\label{thm:svd}
Ist $A\in\C^{m,n}$ eine $r$-rangige Matrix, so existieren unit"are Matrizen $U\in\C^{m,m}, V\in\Cnn$
und $\Sigma^2_+ =\text{diag}(\sigma^2_1,\ldots,\sigma^2_r)\in\C^{r,r}$ mit der Eigenschaft $\sigma_1 \ge \ldots \ge \sigma_r > 0$, die
eine Faktorisierung der Art
\[
A = U \Sigma V^H
\]
erm"oglichen. Dabei ist
\[
\Sigma = \begin{bmatrix} \Sigma_+ & 0_{r,n-r} \\
0_{m-r,r} & 0_{m-r,n-r} \end{bmatrix}.
\]
\textcolor{red}{nochmal bearbeiten mit den quadraten}

\end{thm}

\begin{thm}[Spektralsatz f"ur hermitesche Matrizen]\label{thm:appTheorems:Spektralsatz}
bla
\end{thm}

\begin{thm}[Existenz der Cholesky-Zerlegung]\label{thm:appTheorems:Cholesky}
bla
\end{thm}

\begin{thm}[Jordan'sche Normalform]\label{thm:appTheorems:Jordan}
blubb
\end{thm}

\begin{thm}[Cauchy'scher Integralsatz]\label{thm:appTheorems:Cauchy}
bla
\end{thm}

\begin{thm}[Potenzmethode]\label{thm:appTheorems:Potenzmethode} Es sei $A\in\Cnn$ eine hermitesche Matrix mit Eigenpaaren $(\lambda_i,x_i)_{i=1:n} \in \R\times\Co$ und $n\ge 2$.
Es gelte weiter $|\lambda_1| > |\lambda_j|$ f"ur alle
$j > 1$. Ist $y_0 \in\Co$, so konvergiert die Folge
$(y_k)_{k\in\N}$ mit
\[
y_{(k+1)} = \frac{1}{\|A^{k+1} y_{(0)}\|} A^{k+1}y_{(0)}
\]
gegen einen normierten Eigenvektor zum Eigenwert $\lambda_1$
\end{thm}

\begin{proof}

\end{proof}

\begin{thm}[QR-Zerlegung]\label{thm:appTheorems:QR}

\end{thm}
