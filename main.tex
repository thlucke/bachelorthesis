\documentclass[11pt, twoside]{report} %no headers, no twosided layout by default

% -------------------------------- %
% ------- P A C K A G E S -------- %
% -------------------------------- %

\usepackage{style/myba}

% -------------------------------- %
% ------- D O C U M E N T -------- %
% -------------------------------- %

\begin{document}

% titlepage
\pagenumbering{roman}
\begin{titlepage}
  \begin{center}
    \vspace*{1cm}

    \Huge
    \textbf{\glqq Weniger ist mehr.\grqq}

    \vspace{0.5cm}
    \LARGE
    Eigenwertfilterung am Beispiel hermitescher Eigenwertprobleme

    \vspace{1.5cm}

    \textbf{Thorsten Matthias Lucke}\\
    Betreuer: Prof. Jörg Liesen, Prof. Christian Mehl

    \vfill

    Arbeit zur Erlangung des akademischen Grades\\
    Bachelor of Science

    \vspace{0.8cm}

  %  \includegraphics[width=0.4\textwidth]{images/logo_tu}

    \Large
    Department Name\\
    Technische Universität Berlin\\
    %Country\\
    \today

  \end{center}
\end{titlepage}


% preface
\thispagestyle{empty}
\cleardoublepage
\chapter*{}
\textit{Meiner Familie,\\
der Thesisselbsthilfegruppe,\\
der H"angemathe und dem 1 Caf\'e}

\chapter*{Erkl"arung}
Ich versichere, dass ich die vorliegende Arbeit selbst"andig verfasst und keine anderen als die
angegebenen Quellen und Hilfsmittel benutzt habe. Ich reiche sie erstmals als Pr"ufungsleistung ein.
Mir ist bekannt, dass ein Betrugsversuch mit der Note \glqq nicht ausreichend\grqq \ geahndet wird und im Wiederholungsfall zum Ausschluss
von der Erbringung weiterer Pr"ufungsleistungen f"uhren kann.

Name:

Vorname:

Matrikelnummer:

\textcolor{white}{filler}

Berlin, den

Signatur

\chapter*{Danksagung}
Obschon die vorliegende Arbeit der eigenen Feder entstammt, h"atte dieses Werk kaum vollendet werden k"onnen,
w"aren nicht einige H"urden beiseite geschafft worden.
So sehe ich es als meine Pflicht, den Helfern des \glqq R"aumungsdienstes\grqq\ ein paar Worte der Ehrung
zukommen zu lassen.\\

Der erste Dank gilt meinen Betreuern Prof. Dr. J"org Liesen und Prof. Dr. Christian Mehl deren Sprechstunden
stets eine Quelle der Inspiration waren. Immer hilfsbereit standen sie mir fachlich zur Seite und lie"sen mich von ihrer Erfahrung profitieren.\\

Ein weiterer Dank geht an die Mitarbeiterinnen und Mitarbeiter des studentischen Mathekaffees \glqq 1 Caf\'e\grqq. Welche Qualen blieben mir durch
die etlichen Teest"undchen, Schokoriegel und zum Schlafen einladenden Sofas erspart! Im selben Atemzug
danke ich den Mitgliedern der \glqq H"angemathe\grqq, deren kritisches Hinterfragen und Diskussionsfreudigkeit stets anregend war.\\

Schlie"slich richte ich meinen Dank an alle Lektoren, die akribisch jeden noch so hinterh"altig versteckten
Fehler entdeckt haben und damit das Lesen dieser Arbeit zu einem gr"o"seren Vergn"ugen machen.

% content
\tableofcontents

% chapters
\chapter{Pr"aludium}%{Introductio}
\pagenumbering{arabic}
Das L"osen von Eigenwertproblemen ist eine Standarddisziplin in der
numerischen linearen Algebra. Gleichungen der Gestalt

\begin{equation}\label{eq:eigenproblem}
Ax = \lambda Bx
\end{equation}

begegnet man in ganz unterschiedlichen Kontexten. So sind sie beispielsweise
bei der Bestimmung von Eigenfrequenzen oder dem Ermitteln von Fixpunkten beim
Rotieren eines Fußballs\footnote{Hier wird auf den bekannten
\emph{Satz vom Fußball} angespielt. Dieser besagt, dass auf einem Fußball
zwei Punkte existieren, die zu Spielbeginn und zur Halbzeit
an der gleichen Stelle liegen -- informell formuliert.} ebenso wie beim
Untersuchen des PageRanks einer Website von
Bedeutung. Entsprechend strotz der Kanon von angebotenen numerischen
Lösungsmethoden von Vielfalt und Virtuosität.\\

Nun mag der Fall eintreten, da es notwendig wird, lediglich eine Teilmenge
aus der Menge aller Eigenpaare zu untersuchen.\\
\textcolor{red}{Geeignetes Beispiel finden.}\\

Wie soll nun aber der eifrige Numeriker die gewünschte Menge an Eigenpaaren
finden?\\

Man könnte zunächst versuchen sämtliche Eigenpaare zu berechnen, die das Problem
hergibt und händisch die gewünschte Teilmenge auswählen. Bei einem Problem
dieser Größenordnung dürfte der Rechenknecht allerdings eine ordentliche Weile
beschäftigt sein und das Leben ist fürwahr zu knapp, um auf das Terminieren
von Algorithmen zu warten.\\

\section{Grundlagen}

\textcolor{red}{gauss-legendre quadratur, spektral projektoren (sp"ater?), rationale funktionen, spektralzerlegung f"ur allg eigenwertproblem}

Um das Lesen dieser Arbeit mehr zu einer Freude denn zu einer Schikane zu
machen, soll dieser Abschnitt einige Grundlagen der linearen Algebra und
der Funktionentheorie bereitstellen. Obschon sich der Autor bem"uht hat,
in der Literatur g"angige Notation zu benutzen, bittet er den
verst"andnissvollen Leser bei Unklarheiten im Anhang \glqq Notationen\grqq\
nachzuschlagen.\\

Den Anfang machen Definitionen und Resultaten aus der Matrizentheorie.
Eine Matrix $A\in\Cnn$ wird als \emph{hermitesch} bezeichnet, falls
sie die Identit"at $A=A^H$ erf"ullt. Sie ist \emph{positiv definit}, sofern
für alle Vektoren $x\in\Co$ die Abschätzung
\[
x^H A x > 0
\]
gilt. Folglich werden wir eine Matrix \emph{hermitesch positiv definit (HPD)}
nennen, wenn sie sowohl hermitesch als auch positiv definit ist.\\

Ist $A\in\Cnn$ eine solche HPD-Matrix und sind $x,y\in\Cn$, so werden wir anstelle von
$x^H A y$ die Notation $\sp{A}{x,y}$ verwenden. Im Falle $A=I_n$ schreiben wir
kurz $\sp{}{x,y}$.\\

ne anders: $\langle x,y \rangle = y^H x$ oder doch nicht anders? einfach eintr"age tauschen oder so

Im Rahmen dieser Arbeit werden "uberwiegend Hermitesche Eigenwertprobleme betrachtet.

\begin{prop}
Ist $p$ eine Projektion auf den Unterraum $U$, dann gilt $p(u) = u$ f"ur
alle $u \in U$.
\end{prop}
\begin{proof}
Sei zun"achst $u \in \Bild{p}$. Dann existiert $v \in V$ mit $p(v) - u = 0$.
Nach Definition gilt aber auch $p^2(v)-u = p(u)-u = 0$. Also folgt $p(u)=u$.
\textcolor{red}{gilt $\Bild(p) = U$?}
\end{proof}

Bilinearit"at von $\langle \cdot, \cdot \rangle_A$
Kontur (komplex), Jordankurven, Invarianter Unterraum


\chapter{Mathematische Grundlagen}
Dieses Kapitel widmet sich der Frage, ob und wie man aus der Menge aller Eigenpaare
des Problems
\[
Ax = \lambda Bx
\]
eine gew"unschte Teilmenge ausw"ahlen kann. Dazu werden im Rahmen dieser Arbeit aus dem Katalog der Verfahren
die zwei Folgenden ausgew"ahlt und vorgestellt: Rayleigh-Ritz-Verfahren\footnote{Die im Folgenden pr"asentierte Vorgehensweise wird in der
Literatur nicht einheitlich bezeichnet. Anstelle von Rayleigh-Ritz-Verfahren sind auch die Bezeichnungen Galerkin-Methoden oder ... "ublich (siehe da und da).
In dieser Arbeit wird durchg"angig vom Rayleigh-Ritz-Verfahren oder der Rayleigh-Ritz-Methode
gesprochen.} und Konturintegration. An einem abschlie"senden Beispiel wird au"serdem die Kompatibilit"at beider Konzepte dargelegt.

\section{Rayleigh-Ritz-Verfahren}
%\footnote{Dieser Abschnitt orientiert sich in Formulierung und Dramaturgie an Abschnitt 4.3.1 aus Y. Saad Solving large evp. Hier verallgemeinern wir aber die Konzepte direkt auf allg. ewp.}
Die Idee dieser Klasse von Verfahren ist das Approximieren des von
den gesuchten Eigenvektoren aufgespannten Unterraums. %Dazu sollen -- wie der Name vermuten l"asst --
%Projektionen bem"uht werden.\\
Bevor wir das allgemeine Eigenwertproblem untersuchen, wenden wir unsere Aufmerksamkeit
dem gew"ohnlichen Eigenwertproblem
\[
Ax = \lambda x% \textcolor{red}{variablen wieder neu einf"uhren? oder in der einleitung: im folgenden sei stets...}
\]
zu und betrachten f"ur eine Zahl $m\in\N$ mit $m\le n$
einen $m$-dimensionalen Unterraum $\U_m\subseteq\Cn$. Dieser zun"achst nicht
n"aher bestimmte \emph{Suchraum} wird als Grundlage f"ur das Verfahren gew"ahlt.
Gesucht sind nun Paare $(\widetilde{\lambda}, \widetilde{x})\in\C\times \U_m \setminus\{0\}$ --
die wir als approximierte L"osungen des Eigenproblems verstehen wollen --
welche die Eigenschaft
\begin{equation}\label{eq:orthogonal}
\langle u, A\widetilde{x} - \widetilde{\lambda}\widetilde{x} \rangle=0
\end{equation}
f"ur alle $u \in \U_m $ erf"ullen. Das Residuum $(A\widetilde{x} - \widetilde{\lambda}\widetilde{x})$
soll also orthogonal auf dem Suchraum stehen. Paare, die diesem Anliegen nachkommen,
werden auch \emph{Ritz-Paare} bez"uglich des Suchraums $\U_m$ genannt.\footnote{Ritz-Paare m"ussen nicht notwendigerweise eine Orthogonalit"atsbedingung erf"ullen... oder doch?}\\

Wir wollen nun annehmen, dass mit der Menge von Vektoren $\{u_i\}_{i=1:m}\subseteq \Cn$ eine Orthonormalbasis
des Unterraums $\U_m$ gegeben ist. Definieren wir dann die Matrix $U_m :=[u_i]_{i=1:m}\in\C^{n,m}$, so muss wegen $\widetilde{x}\in\U_m$
ein Vektor $y\in\C^m$ existieren, mit $U_m y = \widetilde{x}$. Die Forderung \eqref{eq:orthogonal}
ist dann zu der Gleichung
\[
U_m^H(AU_m y - \widetilde{\lambda} U_m y) = 0
\]
"aquivalent. Als direkte Konsequenz dieser Umformulierung erhalten wir unter Ausnutzung
der Orthogonalit"at der Spalten von $U_m$ mit
\begin{equation}\label{eq:transform}
(U_m^H A U_m) y = \widetilde{\lambda}y
\end{equation}
ein neues Eigenwertproblem. Jedes Eigenpaar $(\widetilde{\lambda},y)$ von \eqref{eq:transform}
liefert dann nach Konstruktion mit $(\widetilde{\lambda}, U_m y)$ ein Ritz-Paar des gew"ohnlichen
Eigenwertproblems bez"uglich des Suchraums $\U_m$. In Abh"angigkeit von der Wahl
des Suchraumes variiert nat"urlich G"ute der Approximation. Ist etwa $\U_m$
durch Eigenvektoren aufgespannt, so ist jedes Ritz-Paar schon ein Eigenpaar.
Wir werden sp"ater aus den Untersuchungen des allgemeinen Eigenwertproblems folgern,
dass bereits im Falle der Invarianz von $\U_m$ unter $A$ jedes Ritz-Paar von $A$ bez"uglich $\U_m$ bereits ein Eigenpaar von $A$ ist.
An dieser Stelle begn"ugen wir uns vorerst mit dieser erfreulichen Botschaft.\\

%Fassen wir die eben geschilderte Vorgehensweise algorithmisch zusammen, so ist
%der folgende Pseudocode denkbar:
Die eben skizzierte Vorgehensweise zur Berechnung von Ritz-Paaren des gew"ohnlichen Eigenwertproblems
l"asst sich algorithmisch wie folgt zusammenfassen.

\begin{algorithm}\label{alg:rp}
\caption{Berechnung von Ritz-Paaren}
\begin{algorithmic}[1]
\State Berechne ONB $\{u_i\}_{i=1:m}$ von $\mathcal{U}_m$ und setze $U_m\gets[u_i]_{i=1:m}$.
\State Setze $\widetilde{A}\gets U_m^H A U_m$ und
l"ose $\widetilde{A}Y = Y \Lambda$ in $Y$ und
$\Lambda$.
\State Setze $\widetilde{X} \gets U_m Y$ und gib Ritz-Paare $(\lambda_i, \widetilde{x}_i)_{i=1:m}$ aus.
\end{algorithmic}
\end{algorithm}

Hierbei entsprechen die Spalten von $Y\in\C^{n,m}$ gewissen Eigenvektoren von \eqref{eq:transform}, w"ahrend auf der Diagonalen von $\Lambda\in\C^{m,m}$
die zugeh"origen Eigenwerte zu finden sind.\\

Die Idee dieser Methode ist also simpel: Transformiere das Eigenwertproblem mit
einer gewissen Matrix in ein anderes Eigenwertproblem und benutze dessen L"osungen,
um Ritz-Paare des urspr"unglichen Problems zu erhalten.
Doch wozu die M"uhe, das urspr"ungliche Eigenwertproblem in ein anderes
Eigenwertproblem zu "uberf"uhren? Zwar gelingt es, aus dem transformierten Problem
\eqref{eq:transform} Ritz-Paare zu extrahieren, aber w"are nicht auch denkbar,
s"amtliche Eigenpaare von $A$ zu approximieren und die zum Unterraum $\U_m$
korrespondierende Teilmenge direkt auszuw"ahlen?\\

Dies mag in Einzelf"allen in der Tat sinnvoller sein. Sprechen wir allerdings
von Matrixdimensionen jenseits der Vorstellungskraft, ist eine vollst"andige
Berechnung aller Eigenpaare mitunter ein sehr zeitintensives Vergn"ugen. Bei
genauerer Betrachtung der Gleichung \eqref{eq:transform} f"allt auf, dass die
Matrix $(U_m^H A U_m) \in \C^{m,m}$ im Falle $n \gg m$ ein mitunter deutlich kleineres
Format hat, als die Matrix $A$ im urspr"unglichen Problem. Man darf hier also erwarten,
dass die ben"otigte Laufzeit zur Bestimmung der Ritz-Paare mit dem Algorithmus \ref{alg:rp}
geringer ist, als beim Approximieren s"amtlicher Eigenpaare von $A$. Dies
wird im vierten Kapitel anhand ausgew"ahlter Beispiele vorgef"uhrt.\\

Nun, da wir der Existenz und Funktionsweise der Rayleigh-Ritz-Methode gewahr wurden, erweitern wie die obige Theorie auf das
verallgemeinerte Eigenwertproblem
\begin{equation}\label{chap2:gevp}
Ax = \lambda Bx.
\end{equation}
Dabei gehen wir ganz analog zum gew"ohnlichen Eigenwertproblem vor und betrachten wieder
einen $m$-dimensionaler Suchraum $\U_m\subseteq \Cn$.
Gefunden werden sollen dieses Mal Paare $ (\widetilde{\lambda}, \widetilde{x}) \in \C
\times \U_m \setminus\{ 0\}$ die der Orthogonalit"atsbedingung
\begin{equation}\label{eq:borthogonal}
A\widetilde{x} - \widetilde{\lambda}B\widetilde{x} \ \bot \ \U_m
\end{equation}
gen"ugen. Durch die Wahl eines geeigneten Vektors $y\in\C^m$ kann die N"aherungsl"osung $\widetilde{x}$ wie zuvor durch das Produkt $U_m y$ ersetzt werden, wobei die Spalten von $U_m$ erneut eine ONB des Suchraums $\U_m$ bilden. Folglich l"asst sich die zu \eqref{eq:borthogonal} "aquivalente Forderung
\[
U_m^H(AU_m y - \widetilde{\lambda} BU_m y) = 0.
\]
aufstellen. Wie bereits beim gew"ohnlichen Eigenwertproblem, l"asst sich nun aus jeder L"osung
$(\widetilde{\lambda}, y)$ von
\begin{equation}\label{eq:transformedevp}
(U_m^H AU_m) y = \widetilde{\lambda} (U_m^H B U_m) y.
\end{equation}
mit $(\widetilde{\lambda}, U_m y)$ ein Ritz-Paar f"ur das verallgemeinerte Eigenwertproblem
gewinnen.\\

Da mit der Matrix $B$ ein weiterer Darsteller auf der Eigenproblemb"uhne ber"ucksichtigt werden muss, zieht dies als Konsequenz eine Anpassung des Algorithmus' \ref{alg:rp} nach sich. Wir k"onnen diesen in der folgenden Manier abwandeln.

\begin{algorithm}\label{alg:rp}
\caption{Berechnung von Ritz-Paaren}
\begin{algorithmic}[1]
\State Berechne ONB $\{u_i\}_{i=1:m}$ von $\mathcal{U}_m$ und setze $U_m\gets[u_i]_{i=1:m}$.
\State Setze $\widetilde{A}\gets U_m^H A U_m$,
$\widetilde{B} \gets U_m^H BU_m$ und
l"ose $\widetilde{A}Y = \widetilde{B}Y \Lambda$ in $Y$ und $\Lambda$.
\State Setze $\widetilde{X} \gets U_m Y$ und gib Ritz-Paare $(\lambda_i, \widetilde{x}_i)_{i=1:m}$ aus.
\end{algorithmic}
\end{algorithm}

Wir wollen uns nun vorerst von der Algorithmik verabschieden und einige Beobachtungen
festhalten. Eingangs wurde behauptet, dass die Invarianz des Suchraumes $\U_m$ unter
$A$ dazu f"uhrt, dass jedes Ritz-Paar bez"uglich $\U_m$ ein Eigenpaar von $A$ ist.
Doch warum ist das so? Dies zu beantworten verpflichtet sich der folgende Satz.

\begin{thm}\label{thm:invariant}
Neben zwei Matrizen $A,B\in\Cnn$ -- wobei $B$ eine HPD-Matrix ist -- sei f"ur
$m\in\N$ mit $m\le n$ ein $m$-dimensionaler Unterraum $\U_m \subseteq \C_n$ gegeben.
Ist dieser invariant unter $B^{-1}A$, so ist jedes Ritzpaar von $B^{-1}A$
bez"uglich $\U_m$ auch ein Eigenpaar von $(A,B)$.
\end{thm}

\begin{proof}
Beginnen wir mit einer ONB $\{u_i\}_{i=1:m}\subseteq\U_m$ des Unterraums $\U_m$
und setzen wie bisher $U_m := [u_i]_{i=1:m}\in\C^{n,m}$. Aufgrund der Invarianz
von $\U_m$ unter $B^{-1}A$ muss eine Matrix $V_m \in \C^{m,m}$ existieren, welche
die Gleichung
\begin{equation}\label{eq:thminvariant}
B^{-1}A U_m = U_m V_m
\end{equation}
erf"ullt. Insbesondere folgt unter Ausnutzung der Orthogonalit"at der Spalten
von $U_m$ die Identit"at $U_m^H B^{-1}A U_m = V_m$.
Sind nun $\lambda\in\C$ und $y\in\C^m\setminus\{0\}$ so gew"ahlt, dass $(\lambda, U_m y)$
ein Ritz-Paar von $B^{-1}A$ ist, so folgt aus
\[
U_m^H B^{-1}A U_m y = V_m y
\]
mit \eqref{eq:thminvariant} die Gleichung
\[
B^{-1}AU_m y = U_m V_m y = \lambda U_m y
\]
und schlie"slich auch die Behauptung durch Umstellen.
\end{proof}

Da das gew"ohnliche Eigenwertproblem ein Spezialfall des allgemeinen Eigenwertproblems
ist, k"onnen wir aus dem eben bewiesenen Resultat unmittelbar das folgende Korollar
ableiten.

\begin{kor}
Ist $A\in\Cnn$ und $\U_m\subseteq \Cn$ ein $m$-dimensionaler $A$-invarianter Unterraum, so ist
jedes Ritz-Paar von $A$ bez"uglich $\U_m$ ein Eigenpaar von $A$.
\end{kor}

\begin{proof}
Betrachte f"ur $B:=I_n$ das verallgemeinerte Eigenwertproblem
\[
Ax = \lambda Bx = \lambda x.
\]
Die Aussage folgt dann aus dem vorigen Satz.
\end{proof}

Es wird sich zu einem sp"ateren Zeitpunkt herausstellen, dass diese Erkenntnis aus algorithmischer Sicht h"ochst n"utzlich ist.
Die Rayleigh-Ritz-Methode l"asst sich nämlich in ein iteratives Verfahren umwandeln, welches in jedem Schritt den Suchraum "andert. Ist die Suchraumiterierte irgendwann einmal $A$-invariant, so kann der Algorithmus abgebrochen werden.\\

Wenden wir uns einer weiteren Kuriosit"at zu. Zwischen obigem Verfahren und dem Konzept der Projektion besteht ein enger Zusammenhang.
Um zu dieser Einsicht zu gelangen, bedarf es ein wenig
Vorbereitung.
%Das Rayleigh-Ritz-Verfahren steht in einem engen Zusammenhang zu Projektionsverfahren.
%Mit einer geeigneten Projektionsmatrix $P\in\Cnn$ ist ein Paar $(\lambda, x)\in\C\times\Co$
%\textcolor{red}{(oder $\in\C\times\U_m\setminus\{0\}$?)} genau dann eine L"osung
%der Gleichung \eqref{chap2:gevp}, wenn es eine L"osung von
%\[
%P^H APx = \lambda P^H BPx
%\]
%ist. Richtig definiert, besitzt $P$ noch weitere n"utzliche Eigenschaften, die
%der folgende Satz vorstellt.
\begin{thm}\label{thm:projektor}
Es sei $B\in\Cnn$ eine HPD-Matrix und f"ur $m\in\N$ mit $m\le n$ sei
ein $m$-dimensionaler Unterraum $\U_m \subseteq \Cn$ gegeben. Sei weiter $\{u_i\}_{i=1:m}\subseteq\U_m$ eine
Basis $B$-orthonormaler Vektoren, das hei"st, die Matrix $U_m := [u_i]_{i=1:m}
\in\C^{n,m}$ erf"ulle die Gleichung $U_m^H B U_m = I_m$. Dann ist die von der Matrix
 $P := U_m U_m^H B \in \Cnn$ induzierte lineare Abbildung
\[
p \colon \Cn \to \Cn, x\mapsto U_m U_m^H Bx
\]
eine $B$-orthogonale Projektion auf den Unterraum $\U_m$. Au"serdem gilt
f"ur alle $x\in\Cn$ die Identit"at
\[
\|x-p(x)\|_B = \min_{y\in\U_m} \|x - y\|_B.
\]
\end{thm}

\begin{proof}
Aus der $B$-Orthogonalit"at von $U_m$ folgt f"ur alle $x\in\Cn$
\[
p^2 (x) = U_m U_m^H B U_m U_m^H B x= U_m U_m^H Bx = p(x).
\]
%und damit gilt die Mengengleichheit $p(\Cn) = \U$ nach Konstruktion. %($\Rang{P_B} = \dim(\U)$).
Ist nun $y\in\U_m$ und $x\in\Cn$, dann folgt wegen $p(y) = y$ und $B=B^H$ auch
\begin{align*}
\langle y, x-p(x)\rangle_B &= y^H (Bx - B U_m U_m^H Bx) \\
&= y^H (Bx - B^H U_m U_m^H Bx) \\
&= y^H Bx - p(y)^H Bx = 0.
\end{align*}
Es gilt demnach $x-p(x) \ \bot_B \ \U_m$. Die Optimierungsaufgabe wird schlie"slich wegen
\begin{align*}
\|x-y\|_B^2 &= \|x-p(x) + p(x)-y\|_B^2 \\
&= \|x-p(x)\|_B^2 + \|p(x)-y\|_B^2\\
&\ge \|x-p(x)\|_B^2
\end{align*}
gel"ost. Dabei ist bei der zweiten Gleichheit zu ber"ucksichtigen, dass $x-p(x) \in \U_m^{\bot_B}$
und $p(x)-y \in \U_m$ gilt.
%denn aufgrund der positiven Definitheit von Normen, gilt Gleichheit genau dann,wenn $p(x)=x$ erf"ullt ist.
\end{proof}
Mit Hilfe der im Satz eingef"uhrten Projektionsmatrix $P$, l"asst sich die Gleichung
\eqref{eq:transformedevp} -- zu finden auf Seite \pageref{eq:transformedevp} -- umformulieren. Durch Linksmultiplikation
mit $B^H U_m$ und dem Einschub der Identit"at $U_m^H B U_m$ gilt n"amlich
\[
B^H U_m (U_m^H A U_m)(U_m^H B U_m)y = \widetilde{\lambda} B^H U_m (U_m^H B U_m)(U_m^H B U_m)y
\]
und folglich
\[
P^H A P U_m y = \widetilde{\lambda} P^H B P U_m y.
\]
Wenn wir uns an dieser Stelle erinnern, dass $\widetilde{x}$ durch $U_m y$ ersetzt wurde, so
erhalten wir
\[
P^H A P \widetilde{x} = \widetilde{\lambda} P^H B P \widetilde{x}.
\]
Im Fall der gew"ohnlichen Eigenwertgleichung erhalten wir speziell
\[
P A P \widetilde{x} = \widetilde{\lambda} \widetilde{x}.
\]
Wenden wir uns nun nochmal dem Rayleigh-Ritz-Verfahren zu und betrachten das
verallgemeinerte Eigenwertproblem f"ur eine hermitsche Matrix $A\in\Cnn$ sowie eine
HPD-Matrix $B\in\Cnn$. Angenommen, uns st"unde mit dem Suchraum $\U_m$ bereits ein von Eigenvektoren
aufgespannter Unterraum zur Verf"ugung. Dann existiert eine Matrix $U_m := [u_i]_{i=1:m}\in\C^{n,m}$ mit
$B$-orthonormalen Spalten aus Eigenvektoren von $(A,B)$ und $\Bild(U_m)=\U_m$, sowie eine reelle Diagonalmatrix
$\Lambda_m := \text{diag}(\lambda_i)_{i=1:m}\in\C^{m,m}$ mit
\[
AU_m = BU_m\Lambda_m.
\]
Dabei ensprechenden die Diagonaleintr"age von $\Lambda_m$ gerade den zu den Spalten
von $U_m$ korrespondierenden Eigenwerten von $(A,B)$. Folglich w"are das Bild
des Spektralprojektors $P=U_m U_m^H B$ nach Konstruktion wegen
\[
B^{-1}AP = B^{-1}(AU_m)U_m^H B = B^{-1}(BU_m \Lambda)U_m^H B
= U_m\Lambda U_m^H B = \left(\sum_{i=1}^m \lambda_i u_i u_i^H\right)\cdot B
\]
ein $(B^{-1}A)$-invarianter Unterraum. Daher erscheint es
wegen Satz \ref{thm:invariant} sinnvoll, den Spektralprojektor $P$ in das
Rayleigh-Ritz-Verfahren zu integrieren.
Da jedoch der Unterraum $\U_m$ im Allgemeinen nicht bekannt ist, stellt sich die Frage, ob es
"uberhaupt sinnvoll ist nach dem Projektor zu suchen.
Wir werden jedoch alsbald feststellen, dass %zumindest
die Matrix $U_m U_m^H$ analytisch berechnbar ist.


% nochmal ein wort verlieren, warum hier jetzt von filterung die rede sein kann
%folgender teil vielleicht eher was f"ur den anfang des kaptitels

%Im Allgemeinen ist es nicht ohne Weiteres m"oglich einen bestimmten Eigenraum gezielt
%zu approximieren, wenn keine weiteren Informationen bekannt sind. G"angige
%Wahl f"ur Suchr"aume sind Krylow-R"aume. Deren Struktur f"uhrt dazu, dass
%Unterr"aume approximiert werden, die von den Eigenvektoren mit den betragsm"a"sig
%gr"o"sten Eigenwerten aufgespannt werden. Es obliegt daher dem pers"onlichen
%Geschick einen passenden Suchraum aufzustellen.\\



%\textcolor{red}{anstelle einer basis von U kann auch beliebige matrix Q gew"ahlt
%werden und dann diese auf Um projeziert werden mit $U:=U_m^H U_m B Q$ Dann
%ist man fertig, aber Projektor ist i.A.nicht bekannt, daher approximieren. siehe
%kapitel 3}

\section{Konturintegration}\label{sec:kontur}

Die folgenden Zeilen pr"asentieren eine Vorgehensweise, die es gestattet, den im vorigen
Abschnitt eingef"uhrten Spektralprojektor analytisch zu konstruieren. Wir werden
zu einem sp"ateren Zeitpunkt untersuchen, in wie weit die Numerik in der Lage ist, dieses
Verfahren algorithmisch umzusetzen.\\

Zun"achst ist es erforderlich einige Begriffe und Notationen einzuf"uhren. F"ur zwei Zahlen $z_1, z_2 \in\C$ sei mit der Menge
\[
[z_1, z_2] := \{(1-t)z_1 + tz_2 \in\C \mid t\in[0,1]\}
\]
die direkte Verbindungsstrecke zwischen $z_1$ und $z_2$ bezeichnet. Im Spezialfall $z_1,z_2 \in\R$
stellen wir uns daher unter $[z_1,z_2]$ ein reellwertiges Intervall eingebettet in die komplexe Zahlenebene vor.
Des Weiteren ben"otigen wir sogenannte \emph{Jordan-Kurven}. Diese definieren wir im Kontext
dieser Arbeit wie folgt.

\begin{defn}
  Es sei $S:=\{z\in\C : |z| = 1\}$. Eine Menge
  $\Gamma\subset\C$ hei"st \emph{Jordan-Kurve}, falls
  ein Hom"oomorphismus $\gamma\colon S\to \C$ existiert mit $\Gamma = \Bild(\gamma)$. Dabei ist
  ein \emph{Hom"oomorphismus} eine stetige Bijektion mit stetiger Umkehrabbildung.
\end{defn}

Insbesondere sind Jordan-Kurven also geschlossene und "uberschneidungsfreie Kurven.
Ausgestattet mit diesen neuen Vokabeln kann die nun Pr"asentation des angek"undigten Verfahrens beginnen. Im weiteren Verlauf seien $A, B \in\Cnn$ hermitesche Matrizen und $B$ zus"atzlich
positiv definit.\\

Wir wollen annehmen, dass zwei Zahlen $\lambda_1, \lambda_2 \in \R$ mit $\lambda_1 < \lambda_2$
vorgegeben sind, die das abgeschlossene Intervall $I:=[\lambda_1, \lambda_2]$ definieren.
Wie bisher sollen Paare der Gestalt $(\lambda, x) \in I \times \Co$ ermittelt werden,
welche der verallgemeinerten Eigenwertgleichung
  \begin{equation}\label{eq:eigen} %eventuell einfaches ewp
  Ax = \lambda Bx
  \end{equation}

gen"ugen. Der Ausgangspunkt zur Bestimmung der gesuchten Eigenpaare %\footnote{Um der besseren Lesbarkeit Willen werden
%im Folgenden die verallgemeinerten Eigenvektoren und deren verallgemeinerte
%Eigenwerte kurz als Eigenvektor und Eigenwert bezeichnet.}
ist die durch%wird das Problem mit Hilfe einer
%gewissen Matrix $Q\in\Cnn$ in das "aquivalente Problem
%  \begin{equation}\label{eq:Qeigen}
%  A_Q \phi = \mu B_Q \phi
%  \end{equation}
%"uberf"uhrt. Hierbei sind $A_Q = Q^TAQ$, $B_Q = Q^TBQ$ und $(\mu, \phi)
%\in I\times\C^n$.
%Wird die Matrix $Q$ richtig gew"ahlt, so ist dann jeder zul"assige Eigenwert von
%\eqref{eq:Qeigen} auch ein zul"assiger Eigenwert von \eqref{eq:eigen} und
%umgekehrt.\footnote{Daher ist die Angabe $\mu\in I$ gerechtfertigt.}
%F"ur die Ermittlung der gesuchten Eigenvektoren bedarf es hingegen zus"atzlicher Arbeit.\\
%Zur Bestimmung der Transformationsmatrix $Q$ wird eine Konturintegration bem"uht.
%Ausgangspunkt dieser Integration ist die durch
  \begin{align*}
  G\colon\Omega &\to\Cnn\\
  \omega &\mapsto (\omega B - A)^{-1}
  \end{align*}

definierte \emph{Green-Funktion}, wobei $\Omega \subseteq \C$ eine vom Spektrum
von $B^{-1}A$ disjunkte Teilmenge der komplexen Zahlen ist.\footnote{W"are $\Omega$ nicht disjunkt vom Spektrum von $B^{-1}A$, so g"abe es ein $\omega\in\Omega$, welches zur Singularit"at von $(\omega B-A)$ f"uhrte.}
Diese Funktion $G$ wird nun "uber eine Jordan-Kurve $\Gamma$,
die um das vorgegebene Intervall $I$ \glqq heruml"auft\grqq, %Pr"aziser!%
in der Gestalt
\begin{equation}\label{eq:intgreen}
\frac{1}{2\pi\iota}\int_\Gamma G(\omega)\text{ d}\omega
\end{equation}
integriert.\footnote{Mit $\iota$ ist im Folgenden stets die imagin"are Einheit bezeichnet,
also $\iota = \sqrt{-1}$.} Das Integral ist hierbei eintragsweise zu verstehen.\\

Wir wollen nun annehmen, dass es genau $m\in\N$ Eigenwerte gibt,
die im Inneren von $I$ liegen. %\footnote{Diese Forderung ist wohldefiniert, da
%der Eigenschaften von $A$ und $B$ wegen alle %Eigenwerte von $B^{-1}A$ reell sind.}
%F"ur zu diesen Eigenwerten passende Eigenvektoren $\{x_i\}_{i=1:k} \subseteq \Cn$ sei au"serdem die Matrix
%$X_k := [x_i]_{i=1:k}$ gegeben.
Dann kann man nachweisen, dass sich \eqref{eq:intgreen} durch eine Matrix
$U_m = [u_i]_{i=1:k}\in\C^{n,m}$ mit $B$-orthonormalen Spalten in
\[
\frac{1}{2\pi\iota}\int_\gamma G(\omega)\text{ d}\omega = U_m U_m^H
\]
faktorisieren l"asst, wobei $u_i$ f"ur jedes $i=1:m$ ein zu einem in $I$ liegenden
Eigenwert korrespondierender Eigenvektor ist.\footnote{F"ur einen Beweis des Spezialfalls $B=I$ sei auf \cite[64f]{gohberg} und \cite[222-226]{liesen} verwiesen.}
Als direkte Konsequenz erhalten wir folglich mit dem Produkt
\[
\left( \frac{1}{2\pi\iota}\int_\gamma G(\omega)\text{ d}\omega \right) \cdot B
\]
den Spektralprojektor $U_m U_m^H B$. %Mit diesem Projektor
%k"onnen wir nun vorgehen wie im vorigen Kapitel und alle Eigenpaare des Problems
%finden.% Bedauerlicherweise bleibt die Dimension des Problems unver"andert. Daher
%begn"ugt man sich anstelle der Matrix $B$ mit einer vollrangigen Matrix $\widetilde{B}\in\C^{n,k}$
%um das Ausgangsproblem auf die Dimension $(k\times k)$ zu reduzieren.\\

%MEHR KOMPANA!
%B-Orthogonalit"at der x-vektoren
\newpage
\section{Illustration}\label{sec:bsp}

Bevor wir uns vom zweiten Kapitel verabschieden, sollen die Erkenntnisse der vorangegangenen
Abschnitte an einem Beispiel vorgef"uhrt werden. Dabei werden wir ausgehend von einem
verallgemeinerten Eigenwertproblem gewisse Eigenwerte filtern und die Konturintegration
mit der Ritz-Methode kombinieren.\\

\textcolor{red}{zu komplizert...}
Daf"ur wenden wir uns den beiden Matrizen
%Beipsiel: einmal mit irgendner matrix $\widetilde{B}$ und dann einmal $P^H AP$
%mit dem Spektralprojektor l"osen. (hei"st: einmal gro"ses Problem und einmal
%kleines Problem)
$A,B \in \C^{3,3}$ zu, welche durch
\[
A:= \begin{bmatrix} 3 & 3 & 0 \\ 3 & 9 & 0 \\ 0 & 0 & 1 \end{bmatrix} \text{ und }
B:= \begin{bmatrix} 3 & 0 & 0\\ 0 & 6 & 0 \\ 0 & 0 & 1 \end{bmatrix}
\]
gegeben sind. Dann ist $A$ hermitesch und $B$ eine HPD Matrix.
Durch einfaches Nachrechnen "uberpr"uft man, dass das
verallgemeinerte Eigenwertproblem \eqref{eq:eigen} durch Vektoren
$x_1 \in \spn_\C\{e_1+e_2\}$ mit zugeh"origem Eigenwert $\lambda_1 = 2$, Vektoren
$x_2 \in \spn_\C\{2e_1 - e_2\}$ mit zugeh"origem Eigenwert $\lambda_2 = 1/2$, sowie
Vektoren $x_3 \in\spn_\C\{e_3\}$ mit zugeh"origem Eigenwert $\lambda_3 = 1$
gel"ost wird.\footnote{Hier bezeichnen $e_1, e_2, e_3 \in\Cn$ die kanonischen Einheitsvektoren.}\\

Vergessen wir f"ur den Moment, dass uns die Eigenpaare bekannt sind und versuchen
mit den vorgestellten Methoden die Eigenpaare auf $I = [-3/2,3/2]$ zu bestimmen.
Gem"a"s Abschnitt \ref{sec:kontur} w"ahlen wir daher als Integrationskontur $\Gamma$ das Bild der Funktion
\begin{equation}\label{kurve}
\gamma\colon [0,2\pi]\to\Cn\text{, }\varphi\mapsto \frac{3}{2}e^{\iota \varphi}
\end{equation}
und integrieren die Green-Funktion
\[
G(\omega) = \begin{bmatrix}
3\omega-3 & -3 & 0 \\
-3 & 6\omega-9 & 0 \\
0 & 0 & \omega - 1
\end{bmatrix}^{-1} = \begin{bmatrix}
\frac{6\omega-9}{9(2\omega -1)(\omega - 2)} &
\frac{3}{9(2\omega -1)(\omega - 2)} & 0 \\
\frac{3}{9(2\omega -1)(\omega - 2)} &
\frac{3\omega-3}{9(2\omega -1)(\omega - 2)} & 0 \\
0 & 0 & \frac{1}{\omega - 1}
\end{bmatrix}
\]
dar"uber.
Als Skalierung der komplexen Einheitssph"are, ist $\Gamma$ eine Jordan-Kurve.
Da nach Konstruktion keiner der Eigenwerte im Bild von
$\gamma$ liegt, ist $G$ auf der gesamten Kontur wohldefiniert.
%\begin{figure}[h!]
%	\center
%	\begin{tikzpicture}
%	\draw[->] (-3.5cm,0cm) -- (3.5cm,0cm) node[right,fill=white] {Re};
%    \draw[->] (0cm,-2.5cm) -- (0cm,2.5cm) node[above,fill=white] {Im};
%    \draw[->] (0cm, 0cm) -- (.7, .7);
%	\draw[red](0cm,0cm)circle(1cm);
%	\foreach \x in {0,0.66,2.33} {
%	\filldraw[black] (\x cm,0) circle(2pt);
%	}
%	\draw (0,-0.3) node{$\lambda_1$};
%	\draw (0.66,-0.3) node{$\lambda_2$};
%	\draw (2.33,-0.3) node{$\lambda_3$};
	%\node[rotate=45] at (0.5, 1) {$r=1$};
	%\node at (2.3,-2.3) {$\C$};
	%\end{tikzpicture}
	%\caption{Skizze der Kurve \textcolor{red}{$\gamma$} in der komplexen Ebene.}
%\end{figure}
Folgen wir also dem Abschnitt \ref{sec:kontur}, erhalten wir

\[
\frac{1}{2\pi\iota} \int_\Gamma G \text{ d}s =
\frac{1}{2\pi\iota}\int_0^{2\pi} G(\gamma(\omega))\cdot \gamma'(\omega)
\text{ d}\omega
= \frac{1}{18} \begin{bmatrix}
4 & -2 & 0 \\
-2 & -1 & 0 \\
0 & 0 & 3
\end{bmatrix}
\]

Diese Matrix l"asst sich mit den $B$-orthonormalen Vektoren $x_1 = 1/\sqrt6 \cdot e_2$
und $x_2 = e_3$ wie gew"unscht in
\[
\begin{bmatrix} 0 & 0 & 0 \\ 0 & 1/6 & 0 \\ 0 & 0 & 1 \end{bmatrix}
= \begin{bmatrix} 0 & 0  \\ 1/\sqrt6 & 0  \\ 0 & 1  \end{bmatrix}
\begin{bmatrix} 0 & 1/\sqrt6 & 0 \\ 0 & 0 & 1 \end{bmatrix} =: X_2 X_2^T
\]
faktorisieren. Nachdem dies geschafft ist, lassen wir uns von der in Abschnitt
bla diskutierten Vorgehensweise inspirieren und reduzieren das Problem zun"achst mit
Hilfe einer vollrangigen Matrix $Y\in\C^{3,2}$ auf ein Problem kleinerer Dimension.
Zu diesem Zwecke w"ahlen wir
\[
Y:=\begin{bmatrix} 1 & 1 \\ 0 & 1 \\ 0 & 0\end{bmatrix}
\]
und transformieren mit der Matrix $Q:=X_2 X_2^T Y$ das Ausgangsproblem vem"oge
$\widetilde{A}:= Q^T A Q$ und $\widetilde{B}:=Q^T B Q$ auf das Problem
\[
\widetilde{A}y = \begin{bmatrix}0 & 0\\0 & 1/9 \end{bmatrix} y = \mu \begin{bmatrix}0 & 0\\0 & 1/6 \end{bmatrix} y = \mu \widetilde{B}y
\]
der Dimension $(2\times 2)$. Die Eigenwerte $0$ und $2/3$ lassen sich leicht ablesen
und stimmen wie erwartet mit den sich auf $[-1,1]$ befindlichen Eigenwerten des eingangs formulierten Problems "uberein.\\

\textcolor{red}{Der Mist funktioniert nich... Die R"ucktransformation liefert keinen Eigenvektor zum Eigenwert 0. Q muss vollen Rang haben. Mistst"uck.}\\


%dem Projektor $P:= X_2 X_2^T B$ transformieren.\\
%Zu l"osen ist daher das Eigenwertproblem
%\[
%P^T A P x = \text{diag}(0,4,0)x = \lambda \cdot \text{diag}(0,6,1)x = \lambda P^T B P x.
%\]


%Da wir bereits wissen, dass wir zwei Eigenpaare finden wollen, w"ahlen wir eine
%beliebige Matrix $\widetilde{B}\in\C^{3,2}$ vollen Ranges -- die wir in diesem
%Beispiel mit $\widetilde{B} = \begin{bmatrix}[5]$
\label{chap2}

%\chapter{Filter und Beschleuniger}
\chapter{Filtertechniken} %oder Filtern mit Projektionen
In diesem Kapitel soll demonstriert werden, dass Konturintegration und
Rayleigh-Ritz-Verfahren in einem engen Zusammenhang stehen \textcolor{red}{wird sich noch zeigen}.
Um dies zu Untermauern, orientieren sich die folgenden Abs"atze an den Ausf"uhrungen
von Ping Tak Peter Tang und Eric Polizzi in ~\cite{ptep}. Allerdings wird
die Notation im Sinne der Konsistenz an einigen Stellen abweichen.

\section{Beschleunigtes Rayleigh-Ritz Verfahren}\label{chap3:beschrr}

Betrachten wir also wie bisher das verallgemeinerte Eigenwertproblem mit zwei
komplexwertigen, hermiteschen $(n\times n)$-Matrizen $A$ und $B$ und fordern
zu"satzlich die positive Definitheit von $B$. Zu diesem Duo gesellt sich nun
mit $p(B^{-1}A)$ ein Polynom in $B^{-1}A$, welches wir benutzen um gem"a"s dem
oben zitierten Paper den Algorithmus () aus dem vorigen Kapitel wie folgt zu "andern.

\begin{algorithm}\label{alg:beschlrr}
\caption{Beschleunigtes iteratives Rayleigh-Ritz-Verfahren}\label{euclid}
\begin{algorithmic}[1]
\State W"ahle $m$ Zufallsvektoren $Q_{(0)} \gets [q_i]_{i=1:m} \in\C^{n,m}$.
Setze $k \gets 1$.
\State \textbf{repeat}
\State \ \ \ \ Approximiere den Unterraumprojektor: $Y_{(k)} \gets p(B^{-1}A)Q_{(k-1)}$
\State \ \ \ \ Reduziere die Dimension: $\widetilde{A}_{(k)} \gets Y_{(k)}^H A Y_{(k)}$,
$\widetilde{B}_{(k)} \gets Y_{(k)}^H B Y_{(k)}$.
\State \ \ \ \ L"ose das transformierte Problem $\widetilde{A}_{(k)}\widetilde{X}_{(k)}
= \widetilde{B}_{(k)}\widetilde{X}_{(k)}\widetilde{\Lambda}_{(k)}$ in
$\widetilde{X}_{(k)}$ und $\widetilde{\Lambda}_{(k)}$.
\State \ \ \ \ Setze $Q_{(k)} \gets Y_{(k)}\widetilde{X}_{(k)}$.
\State \ \ \ \ $k \gets k+1$.
\State \textbf{until} Abbruchkriterium ist erf"ullt.
\end{algorithmic}
\end{algorithm}

Das Polynom $p$ wird in diesem Kontext auch als \emph{Filter} oder \emph{Beschleuniger}
bezeichnet. Von dessen Wahl h"angt n"amlich ab, ob und wie gut Eigenpaare approximiert
werden. Es ist sogar m"oglich, gezielt solche Eigenpaare zu finden, wie sie im
Abschnitt \ref{sec:kontur} gesucht waren.\\

Um dies einzusehen, greifen wir erneut die Notationen aus besagtem Kapitel auf:
Es sei $[\lambda_1, \lambda_2]$ dasjenige Intervall, auf dem die Eigenwerte und
korresponierenden Eigenvektoren gefunden werden sollen und $X_k$ sei diejenige
Matrix dessen Spalten aus gerade diesen Eigenvektoren besteht. Es wurde bereits
diskutiert, dass die durch Konturintegration ermittelten Eigenvektoren
$B$-orthogonal sind. Damit l"asst sich also -- wie im Satz \ref{thm:projektor}
bewiesen -- der Spektralprojektor $P = X_k X_k^H B$ konstruieren. Falls nun
$p(B^{-1}A)$ mit diesem Projektor "ubereinstimmt, dann terminiert Algorithmus
\ref{alg:beschlrr} im Falle der Vollrangigkeit von $Y_{(1)}$ nach einer Iteration.\footnote{
Hierbei ist entscheidend, dass die Anzahl der Spalten von $Q$ mit der Anzahl der
Eigenpaare "ubereinstimmt, die auf dem Intervall zu finden sind (Vgl. ~\cite[356]{ptep}).}
Dies folgt unter Ausnutzung der Invarianz des Bildes von $PQ_{(0)}$ unter $B^{-1}A$
aus dem Satz $\ref{thm:invariant}$.\\

Da der Spektralprojektor in den meisten F"allen unbekannt sein d"urfte, liegt
die Idee nahe, ihn zumindest zu approximieren. Da in ~\cite[356]{ptep} bemerkt wird,
dass dies gut funktioniert, wenn $p$ eine durch \emph{Gau"s-Legendre-Quadratur}
konstruierte rationale Funktion ist, wird sich das folgende Intermezzo mit eben dieser
Klasse von Funktionen besch"aftigen, bevor wir mit der Konstruktion des Projektors
fortfahren.


\section{Rationale Funktionen}

Ausgehend von zwei Polynomen $p, q\in\C [t]$ mit
\[
p := \sum_{k=0}^n p_k t^k \text{ \ und\ } q := \sum_{k=0}^n q_k t^k
\]
definieren wir eine rationale Funktion $\rho\colon\C\setminus{N_q}\to\C$ verm"oge
\[
\rho(t) := \frac{p(t)}{q(t)}
\]
und identifizieren wie "ublich die Unbestimmte $t$ mit den Argumenten von $p$ und $q$.
Dabei ist $N_q := \{t\in\C \mid q(t) = 0\}$. Eine rationale Funktion wird als \emph{echt gebrochen} bezeichnet,
falls die Bedingung $\Grad(p) < \Grad(q)$ erf"ullt ist.



\section{Approximation des Spektralprojektors}

Nun da die f"ur den weiteren Verlauf der Arbeit wichtigen Eigenschaften rationaler
Funktionen wiederholt wurden, widmen wir uns der Approximation des Spektralprojektors
$P = X_k X_k^H B$. Daf"ur setzen wir den in Abschnitt \ref{chap3:beschrr} bereits begonnen Gedankengang aus ~\cite{ptep}
fort und "ubernehmen die zu letzt vereinbarten Voraussetzungen und Notationen.\\

Wenden wir uns also wieder dem reellen Intervall $I := [\lambda_1, \lambda_2]$ zu. Das Ziel ist
die Konstruktion einer rationalen Funktion $\rho\colon\C\to\C$ mit $\rho|_\R \subseteq \R$,
die auf $I$ n"aherungsweise der Indikatorfunktion von $I$ entspricht. Dazu
bem"uhen wir die Cauchy'sche Integraldarstellung der Indikatorfunktion und
wandeln diese mit Hilfe numerischer Quadraturformeln in die gew"unschte
rationale Funktion $\rho$ um.\\

Zu"achst zur Indikatorfunktion: Ist $c\in\R$ der Mittelpunkt des Intervalls $I$ und
$r$ der Abstand des Mittelpunktes zum Rand des Intervalls, dann entspricht die Menge
\[
\mathcal{C} := \{z\in\C : |z-c|\le r\}
\]
gerade einer Kreisscheibe mit Radius $r$ um $c$. F"ur eine auf $\mathcal{C}$
holomorphe Funktion $f\colon\C\to\C$ gilt dann gem"a"s der Cauchy'schen Integralformel
\[
f(z) = \frac{1}{2\pi\iota}\int_{\partial \mathcal{C}}\frac{f(z)}{\omega-z}\text{ d}\omega
\]
f"ur jedes $z$ im offenen Inneren von $\mathcal{C}$ (Vgl. ~\cite[20]{jaenich}).

\section{Der FEAST-Algorithmus}


%\chapter{Rationalfunktionelle Filter}
\chapter{Rationales Filtern}%rationale beschleuniger?
Nachdem das vorangegangene Kapitel Ideen zum Filtern von Eigenpaaren theoretisch beleuchtet hat, werden wir uns nun mit der Frage der praktischen Umsetzbarkeit besch"aftigen.
Im Mittelpunkt wird dabei die Konstruktion geeigneter Suchr"aume stehen, welche im Rayleigh-Ritz-Verfahren zum Einsatz kommen sollen.
Es wird sich zeigen, dass die Konturintegration hierbei ein n"utzliches Hilfsmittel darstellt.

\section{Rayleigh-Ritz Iteration}\label{chap4:beschrr}

%Diese Erkenntnis ist aus algorithmischer Sicht h"ochst interessant. In seiner iterativen Variante kann das Rayleigh-Ritz Verfahren abgebrochen werden, sobald die Suchraumiterierte Wandelt man das Rayleigh-Ritz Verfahren der Art ab, dass

%Es wird sich zu einem sp"ateren Zeitpunkt herausstellen, dass diese Erkenntnis aus algorithmischer Sicht h"ochst n"utzlich ist.
%Die Rayleigh-Ritz-Methode l"asst sich nämlich in ein iteratives Verfahren umwandeln, welches in jedem Schritt den Suchraum "andert. Ist die Suchraumiterierte irgendwann einmal $A$-invariant, so kann der Algorithmus abgebrochen werden. \textcolor{red}{Krylow Raum...}\\

Bei der Behandlung des Rayleigh-Ritz Verfahrens wurde angedeutet, dass das Einbinden einer geeigneten Iterationsvorschrift dabei helfen kann, die G"ute von errechneten Ritz-Paaren zu verbessern.
Hierbei ist \glqq G"ute\grqq\ nat"urlich in Abh"angigkeit vom Kontext zu bewerten. Im Folgenden wollen wir uns genauer mit dieser omin"osen Interationsvorschrift auseinander setzen und beginnen die Herleitung mit der Betrachtung eines sehr einfach umsetzbaren Verfahrens zur Bestimmung von Eigenpaaren.\\

%\footnote{Um besser zu verstehen orienteiren wir uns bei der Einf"uhrung des Algs an der Dramaturgie von Saad blabla 115ff}

Ausgangspunkt ist ein gew"ohnliches Eigenwertproblem mit einer von Null verschiedenen hermiteschen Matrix $A\in\Cnn$. Bei der sogenannten \emph{Potenzmethode} wird
ausgehend von einem Startvektor $y_{(0)}\in\Co$ in jeder Iteration der Vektor
\[
y_{(k+1)} = \frac{1}{\|A^{k+1} y_{(0)}\|} A^{k+1}y_{(0)}
\]
oder dazu "aquivalent
\[
y_{(k+1)} = \frac{1}{\|Ay_{(k)}\|} Ay_{(k)}
\]
berechnet. Dieser Vorgang wird wiederholt, bis gewisse Abbruchkriterien erf"ullt sind. Man kann zeigen, dass die Folge der Iterierten gegen einen zu einem betragsm"a"sig gr"o"sten Eigenwert geh"orenden Eigenvektor der L"ange 1 konvergiert.\footnote{Eine exaktere Formulierung bietet Satz \ref{thm:appTheorems:Potenzmethode} im Anhang \ref{appTheorems}.}
%, begn"ugen wir uns an dieser Stelle mit folgender Plausibilit"atsbetrachtung, welche in "ahnlicher Form in ~\cite[56]{stewart}
%zu finden ist.\\

%Seien $(\lambda_i, x_i)_{i=1:n}$ die Eigenpaare von $A$. Aufgrund der Hermitizit"at bilden dann die Eigenvektoren eine Basis des $\Cn$.
%Folglich existieren komplexwertige Skalare $\alpha_1,\ldots,\alpha_n$ mit
%\[
%y_{(0)} = \sum_{i=1}^n \alpha_i x_i.
%\]
%Wenden wir nun die $k$-te Potenz von $A$ auf $y_{(0)}$ an, ergibt sich daher wegen $A^k x_i = \lambda^k x_i$
%\begin{equation}\label{eq:chap3dominant}
%A^k y_{(0)} = \sum_{i=1}^n \alpha_i \lambda_i^k x_i.
%\end{equation}
%Wir wollen ohne Einschr"ankung annehmen, dass $|\lambda_1| \ge |\lambda_i|$ f"ur alle $i$ mit
%$1<i\le n$ gilt. Gegebenenfalls nummerieren wir die Eigenpaare und Skalare um. Dann wird f"ur gr"o"ser werdende $k$ die rechte Seite von \eqref{eq:chap3dominant} durch den Term $\alpha_1 \lambda_1^k x_1$ dominiert. In Kombination mit der Normierung f"uhrt dies letztlich
%zur behaupteten Approximierung.\footnote{Ein formaler Beweis ist im Anhang zu finden.}\\

\newpage
Da uns daran gelegen ist, nicht nur mit einzelnen Elementen des $\Cn$ zu arbeiten, sondern mit Matrizen zu hantieren, m"ussen wir eine allgemeinere Form der Potenzmethode betrachten. Dazu w"ahlen diesmal eine Startmatrix $Y_{(0)}\in\C^{n,m}$ vollen Ranges und berechnen die $k$-te Iterierte mit
\[
Y_{(k)} = A^k Y_{(0)}.
\]
Bei der Normierung ist allerdings Vorsicht geboten: In seinen Abhandlungen "uber Unterraumiterationen
merkt Y. Saad in ~\cite[Abschnitt 5.1]{saad} an,
dass ungeschicktes Normieren zum zunehmenden Verlust der linearen Unabh"angigkeit der Spalten von $Y_{(k)}$ f"uhren kann.\footnote{Dies ist unmittelbar einzusehen, wenn man bedenkt, dass jede Spalte gegen einen dominanten Eigenvektor konvergiert.}\\

Anstatt jede Spalte von $Y_{(k)}$ separat zu normieren, wird eine QR-Zerlegung\footnote{Eine Formulierung des Satzes "uber die Existenz der QR-Zerlegung ist im Anhang \ref{appTheorems} zu finden. Siehe hierzu Satz \ref{thm:appTheorems:QR}. F"ur weitere Ausf"uhrungen siehe auch \cite[S. 55 ff.]{stewart}.}
bem"uht. Diese f"uhrt in der Tat zu einer Normierung: Sind $Q := [q_i]_{i=1:n}\in\Cnn$ und $R\in\C^{n,m}$ so gew"ahlt, dass $Y_{(k)} = QR$ die QR-Zerlegung von $Y_{(k)}$ ist, so gilt
\[
\Bild(Y_{(k)}) = \Bild([q_i]_{i=1:m})
\]
und $\|q_i\|_2 = 1$ f"ur $i=1:m$. Daher stellt folgender Algorithmus eine Verallgemeinerung der Potenzmethode dar.

\begin{algorithm}
\caption{Verallgemeinerte Potenzmethode (Vgl. \cite[Algorithmus 5.1, S. 115]{saad})}\label{alg:chap4:potenzverfahrenMatrix}
\vspace{.15cm}
\textbf{Input:} Hermitesche Matrix $A$\\
\textbf{Output:} Matrix $Y_{(k)}$ mit approximierten Eigenvektoren
\begin{algorithmic}[1]
\State W"ahle linear unabh"angige Vektoren $\{y_i\}_{i=1:m}$ und setze $Y_{(0)}\gets[y_i]_{i=1:m}$ und $k\gets 1$.
\State \textbf{repeat}
\State \ \ \ \ Setze $Y_{(k)} \gets AY_{(k-1)}$ und berechne QR-Zerlegung $Y_{(k)} = QR$.
\State \ \ \ \ Setze $Y_{(k)} \gets Q$ und $k\gets k+1$.
\State \textbf{until} Verfahren konvergiert.
\end{algorithmic}
\end{algorithm}

Saad weist darauf hin, dass die Kosten der Berechnung der QR-Zerlegung sehr hoch werden k"onnen. Da der von den Spalten von $Y_{(k)}$ aufgespannte Unterraum gleich dem von den Spalten von $A^k Y_{(0)}$ aufgespannten Unterraum ist, schl"agt er daher folgende Abwandlung des eben vorgestellen Algorithmus vor.

\begin{algorithm}
\caption{Gebrauch variabler Exponenten (Vgl. ~\cite[Algorithmus 5.2, S. 116]{saad})}\label{alg:chap4:potentePotenz}
\vspace{.15cm}
\textbf{Input:} Hermitesche Matrix $A$\\
\textbf{Output:} Matrix $Y$ mit approximierten Eigenvektoren
\begin{algorithmic}[1]
\State W"ahle linear unabh"angige Vektoren $\{y_i\}_{i=1:m}$, setze $Y_\gets[y_i]_{i=1:m}$ und w"ahle initialen Exponenten $k$.
\State \textbf{repeat}
\State \ \ \ \ Setze $S \gets A^kY$ und orthonormalisiere $S$ zu $\widehat{S}$.
\State \ \ \ \ Setze $Y \gets \widehat{S}$ und w"ahle neuen Exponenten $k$.
\State \textbf{until} Verfahren konvergiert.
\end{algorithmic}
\end{algorithm}

\newpage

Auch hier ist zu beachten, dass im Falle der Wahl eines sehr gro"sen Exponenten die Unabh"angigkeit der Spalten von $S$ nicht mehr gew"ahrleistet werden kann.
Wir wollen an dieser Stelle auf Konvergenz- und Laufzeitanalysen der eben vorgestellten Algorithmen verzichten und kommen schlie"slich zur vielfach angek"undigten Iterationsvorschrift f"ur das Rayleigh-Ritz Verfahren, welche sich aus der Potenzmethode ableitet.

\begin{algorithm}
\caption{Iteratives Rayleigh-Ritz Verfahren (Vgl. \cite[Algorithmus 5.3, S. 118]{saad})}\label{alg:chap4:rrIteration}
\vspace{.15cm}
\textbf{Input:} Hermitesche Matrix $A$\\
\textbf{Output:} Matrix $Y$ mit approximierten Eigenvektoren
\begin{algorithmic}[1]
\State W"ahle linear unabh"angige Vektoren $\{y_i\}_{i=1:m}$, setze $Y \gets[y_i]_{i=1:m}$ und w"ahle initialen Exponenten $k$.
\State \textbf{repeat}
\State \ \ \ \ Setze $S \gets A^k Y$.
\State \ \ \ \ Orthonormalisiere die Spalten von $S$ und setze $\widetilde{A} \gets S^H A S$.
\State \ \ \ \ Berechne Eigenvektoren $\widetilde{X} \gets [\widetilde{x}_i]_{i=1:m}$ von $\widetilde{A}$.
\State \ \ \ \ Setze $Y \gets S \widetilde{X}$ und w"ahle neuen Exponenten $k$.
\State \textbf{until} Verfahren konvergiert.
\end{algorithmic}
\end{algorithm}

Zun"achst ein Wort zur f"unften Zeile. Hier wurde das Berechnen von Schurvektoren -- so wie es in der oben zitierten Quelle vorgeschlagen wird -- durch das Berechnen von Eigenvektoren ersetzt. Dies ist m"oglich, weil $A$ nach Vereinbarung ein hermitesche Matrix ist.\\% und somit unit"ar diagonalisiert werden kann. Es ist daher nicht n"otig, zwischen Eigenvektoren und Schurvektoren zu unterscheiden.\\ %Wie genau diese Eigenvektoren berechnet werden, wollen wir im Rahmen dieser Arbeit nicht genauer erl"autern.\\

Die Wurzeln des eben erarbeiteten Algorithmus sind deutlich zu erkennen. In den Zeilen vier bis sechs wird das Rayleigh-Ritz Verfahren benutzt. Anstelle von Ritz-Paaren werden allerdings lediglich Ritz-Vektoren berechnet. In jeder Iteration wird wie beim Potenzverfahren ein neuer Exponent festgelegt und somit ein neuer Suchraum $\S = \Bild(A^k Y)$ vorgegeben. Erl"auterungen zum Konvergenzverhalten sind in \cite[Abschnitt 5]{saad} zu finden. Zur Berechnung von Eigenpaaren eines HPD-Eigenwertproblems $(A,B)$, bei dem $B$ nicht mehr der Identit"at entspricht, m"ussen die Zeilen vier bis sechs gem"a"s Algorithmus \ref{alg:chap3:grp} angepasst werden.\\

Es ist m"oglich, den zuletzt eingef"uhren Algorithmus weiter abzuwandeln. Dazu betrachten wir wieder ein HPD-Eigenwertproblem $(A,B)$. Zu diesem Duo gesellt sich nun mit $\p(B^{-1}A)$ ein Polynom in $B^{-1}A$, welches wir benutzen, um das iterative Rayleigh-Ritz Verfahren, wie in Algorithmus \ref{alg:chap4:beschlRrIteration} formuliert, zu "andern.
Diese Methode "ahnelt stark dem Algorithmus \ref{alg:chap4:rrIteration}. Die Zeilen vier bis sechs entsprechen erneut dem Rayleigh-Ritz Verfahren, aber die Berechnung des Suchraums $\S := \Bild(P_{(k)})$ geht anders vonstatten.

\newpage

\begin{algorithm}
\caption{Beschleunigte Rayleigh-Ritz Iteration (Vgl. \cite[Algorithmus A]{ptep})}\label{alg:chap4:beschlRrIteration}
\vspace{.15cm}
\textbf{Input:} HPD-Eigenwertproblem $(A,B)$\\
\textbf{Output:} Matrix $Y_{(k)}$ mit approximierten Eigenvektoren
\begin{algorithmic}[1]
\State W"ahle $m$ linear unabh"angige Vektoren $Y_{(0)} \gets [y_i]_{i=1:m} \in\C^{n,m}$.
Setze $k \gets 1$.
\State \textbf{repeat}
\State \ \ \ \ Approximiere den Unterraumprojektor: $P_{(k)} \gets \p(B^{-1}A)Y_{(k-1)}$
\State \ \ \ \ Reduziere die Dimension: $\widetilde{A}_{(k)} \gets P_{(k)}^H A P_{(k)}$,
$\widetilde{B}_{(k)} \gets P_{(k)}^H B P_{(k)}$.
\State \ \ \ \ L"ose das transformierte Problem $\widetilde{A}_{(k)}\widetilde{X}_{(k)}
= \widetilde{B}_{(k)}\widetilde{X}_{(k)}\widetilde{\Lambda}_{(k)}$ in
$\widetilde{X}_{(k)}$ und $\widetilde{\Lambda}_{(k)}$.
\State \ \ \ \ Setze $Y_{(k)} \gets P_{(k)}\widetilde{X}_{(k)}$ und $k \gets k+1$.
\State \textbf{until} Abbruchkriterium ist erf"ullt.
\end{algorithmic}
\end{algorithm}

Im Kontext dieses Algorithmus wird $\p$ auch als \emph{Filter}\footnote{Dieser Filter ist nicht mit dem in Definition \ref{defn:chap3:filter} eingef"uhrten Begriff zu verwechseln.} oder \emph{Beschleuniger}
bezeichnet. Von dessen Wahl h"angt n"amlich ab, ob und wie gut Eigenpaare approximiert
werden: Sei $[\lambda_1,\lambda_2]$ ein reelles Intervall, in dessen Inneren $l\in\N$ Eigenwerte gefunden werden k"onnen. Ist nun $\p(B^{-1}A)$ der Spektralprojektor,
$m=l$ und hat die Matrix $P_{(1)} = \p(B^{-1}A) Y_{(0)}$ vollen Rang, so konvergiert der Algorithmus \ref{alg:chap4:beschlRrIteration} in einer Iteration
(Vgl. ~\cite[356]{ptep}).\\
%Dies folgt unter Ausnutzung der Invarianz des Bildes von $P_{(1)}$ unter $B^{-1}A$aus dem Satz $\ref{thm:chap3:invariant}$.\\

Da der Spektralprojektor in den meisten F"allen unbekannt sein d"urfte, liegt
die Idee nahe, ihn zu approximieren. Tang und Polizzi ~\cite[356]{ptep} merken an, dass dies gut funktioniert, falls $\p$ eine durch \emph{Gau"s-Legendre-Quadratur} konstruierte rationale Funktion ist.
Um den Gedankengang der Autoren nachvollziehen zu k"onnen, wird sich das folgende Intermezzo mit der Auffrischung des Konzeptes von Quadraturformeln besch"aftigen. Dabei sehen wir von Beweisen und ausufernden Erl"auterungen ab, da diese Thematik in den meisten Einf"uhrungsb"uchern zur numerischen Mathematik ausf"uhrlich besprochen wird.\footnote{Siehe zum Beispiel \cite[Abschnitt 6]{plato}.} Im Anschluss werden wir mit der
Konstruktion des Projektors fortfahren.

\section{Gau"s'sche Quadratur}

Um ein Integral numerisch zu approximieren, bedient man sich sogenannter \emph{Quadraturformeln}. Dazu betrachten wir eine stetige Funktion $f\colon\R\to\R$, welche wir auf einem gegebenen Intervall $I:=[a,b]\subset\R$ integrieren wollen.\footnote{Im Allgemeinen ist die Stetigkeit von $f$ nicht zwingend erforderlich. Dennoch werden wir uns hier der Einfachheit halber auf stetige Funktionen einschr"anken.}
Zu gegebenen St"utzpunkten $(x_i, f(x_i))_{i=0:n}$ auf $I\times\R$ sei $p_n$ das zugeh"orige \emph{Interpolationspolynom} vom Grad $n$, also ein Polynom, welches $p_n(x_i) = f(x_i)$ f"ur alle $i$ mit $0\le i\le n$ erf"ullt.
Dann bezeichnen wir die N"aherung
\begin{equation}\label{eq:quadratur}
Q_n(f) := \int_a^b p_n (x)\text{ d}x =
(b-a)\sum_{i=0}^n \omega_i f(x_i)
\end{equation}
als \emph{interpolatorische Quadraturformel}.
Dabei gilt
\[
\omega_k = \int_0^1 \prod_{j=0,j\neq k}^n
\frac{t-t_j}{t_k - t_j} \text{ d}t, \ t_j
= \frac{x_j-a}{b-a}.
\]

\newpage

Die Qualit"at der Approximation, also die Abweichung vom exakten Integral, h"angt ma"sgeblich von der Wahl und Anzahl der St"utzpunkte ab. Wollten wir
beispielsweise das Integral einer konstanten Funktion berechnen, so erschiene es wenig plausibel, anstelle der direkten Berechnung ein Polynom vom Grad 69 auf 70 St"utzstellen f"ur die Approximation zu bem"uhen.\\

Bei der Anwendung von Gau"s-Legendre-Quadraturen ergibt sich die Wahl der St"utzpunkte durch die Berechnung von Nullstellen von Polynomen, die in einer Orthogonalit"atsbeziehung zueinander stehen.
Wir werden in K"urze sauber formulieren, wie dies zu verstehen ist.\\

Ausgangspunkt f"ur die Integration ist nun eine stetige Funktion $f$, die eine Faktorisierung in zwei stetige Funktionen $\omega$ und $g$ der Art
\[
f = \omega \cdot g
\]
besitzt, wobei $\omega$ auf dem Integrationsintervall $[a,b]$ positiv sein soll. Die Funktion $\omega$ wird auch als \emph{Gewichtsfunktion} bezeichnet.
Ziel ist also nun die Berechnung von
\begin{equation}\label{eq:gintegral}
\int_a^b \omega(x)g(x) \text{ d}x,
\end{equation}
wobei wir zus"atzlich fordern, dass \eqref{eq:quadratur} und \eqref{eq:gintegral} f"ur alle Polynome bis zum Grad $(2n-1)$ "ubereinstimmen.\\

Dazu betrachten wir die Monombasis $\{x^k\}_{k=0:(2n-1)}$ auf dem Raum der Polynome vom
Grad $(2n-1)$. Dann landen wir unweigerlich bei dem
Gleichungssystem
\[
\sum_{j=1}^n \omega_j x_j^k = \int_a^b \omega(x)\cdot x^k \text{ d}x \text{ mit } k = 0:2n-1.
\]
Man kann zeigen, dass die L"osung dieses Systems durch Nullstellen eines Polynoms gegeben ist, welches durch
ein Gram-Schmidt-Orthogonalisierungsverfahren bez"uglich des Skalarproduktes
\[
\langle p,q\rangle_\omega := \int_a^b p(x) q(x)\omega(x) \text{ d}x
\]
konstruiert wurde. Das hei"st konkret: Ausgehend vom Polynom $p_0 \equiv 1$ ist
\[
p_n(x) := x^n - \sum_{j=0}^{n-1} \frac{\langle x^n, p_j \rangle_\omega}{\langle p_j, p_j\rangle_\omega} p_j (x)
\]
gerade dasjenige Polynom $n$-ten Grades, durch dessen Nullstellen das obige Gleichungssystem gel"ost wird.

\newpage

Sind nun $\{x_j\}_{j=1:n}$ die Nullstellen dieses $n$-ten Orthogonalit"atspolynoms, so hei"st die numerische Integrationsformel
\[
Q_n(f) = \sum_{j=1}^n \omega_j f(x_j) \text{ mit }
\omega_j = \langle L_j, 1 \rangle_\omega
= \int_a^b L_j(x)\p(x)\text{ d}x
\]
\emph{Gau"s'sche Quadraturformel der $n$-ten Ordnung}.
Dabei ist $L_j (x)$ durch
\[
L_j(x) = \prod_{k\neq j, k=1}^n \frac{x-x_k}{x_j - x_k}
\]
gegeben. \textcolor{red}{NOCHMAL NACHLESEN. IRGENDWAS STIMMT HIER NICHT.}

\section{Approximation des Spektralprojektors}

Sei $(A,B)$ ein HPD-Eigenwertproblem und $I:=[\lambda_1, \lambda_2]$ ein reelles Intervall. Sei weiter $\mathcal{X}\subseteq\Cn$ derjenige Unterraum, welcher von den Eigenvektoren aufgespannt wird, die zu den im Inneren von $I$ befindlichen Eigenwerten korrespondieren. Wir wollen in diesem Abschnitt den Spektralprojektor approximieren, welcher den Vektorraum $\Cn$ auf $\mathcal{X}$ projiziert.\\

Die erste Ingredienz, die zum Gelingen dieser Approximation beitr"agt, ist eine rationale Funktion $\r\colon\C\to\C$ mit $\r(\R) \subseteq \R$, welche auf $I$ n"aherungsweise der konstanten $1$-Funktion enstpricht, also $\r(I) \subseteq\ ]1-\varepsilon,1+\varepsilon[$ f"ur ein $\varepsilon > 0$.
Daf"ur bem"uhen wir die Cauchy'sche Integraldarstellung der Indikatorfunktion und
wandeln diese mit Hilfe numerischer Quadraturformeln in die gew"unschte
rationale Funktion $\r$ um.\\

Zun"achst zur Indikatorfunktion: Ist $c$ der Mittelpunkt des Intervalls $I$ und
$r$ der Abstand des Mittelpunktes zum Rand des Intervalls, dann entspricht die Menge
\[
\mathcal{C} := \{z\in\C : |z-c| = r\}
\]
der Sph"are mit Radius $r$ um $c$. Mit dem Cauchy'schen Integralsatz
l"asst sich zeigen, dass im Falle $z\notin \mathcal{C}$
\[
\frac{1}{2\pi\iota}\int_{ \mathcal{C}}\frac{1}{t-z}\text{ d}t
= \begin{cases}1 &\text{ falls }|z-c| < r \\ 0 &\text{ falls }|z-c| > r \end{cases}
\]
gilt. Um dieses Integral mit einer Gau"s'schen Quadraturformel ann"ahern zu k"onnen, ben"otigen wir eine Parametrisierung von $\mathcal{C}$.
Wie in \cite{ptep} vorgeschlagen w"ahlen wir die Funktion
\[
\gamma\colon[-1,3]\to\C\text{, }
x\mapsto c+re^{\iota \frac{\pi}{2}(1+x)}
\]
als Parametrisierung.
%Die Ableitung von $\gamma$ ist dann f"ur jedes $t\in[-1,3]$ durch
%\[
%\gamma'(t)=\iota \frac{\pi}{2}re^{\iota \frac{\pi}{2}(1+t)}
%\]
%gegeben.
\newpage

Wir erhalten dann f"ur alle $z\notin\mathcal{C}$ die Gleichung
\begin{align*}
\frac{1}{2\pi\iota}\int_{ \mathcal{C}}\frac{1}{t-z}\text{ d}t
&= \frac{1}{2\pi\iota} \int_{-1}^3 \frac{\gamma'(x)}{\gamma(x)-z}\text{ d}x \\
&= \frac{1}{2\pi\iota} \left( \int_{-1}^1 \frac{\gamma'(x)}{\gamma(x)-z} \text{ d}x +
\int_{1}^3\frac{\gamma'(x)}{\gamma(x)-z}\text{ d}x \right) \\
&= \frac{1}{2\pi\iota} \left( \int_{-1}^1 \frac{\gamma'(x)}{\gamma(x)-z} \text{ d}x +
\int_{-1}^1\frac{\gamma'(2-x)}{\gamma(2-x)-z}\text{ d}x \right) \\
&= \frac{1}{2\pi\iota} \int_{-1}^1 \left( \frac{\gamma'(x)}{\gamma(x)-z} +
\frac{\overline{\gamma'(x)}}{\overline{\gamma(x)}-z}\right)\text{d}x
\end{align*}
wobei $\overline{\gamma(x)}$ und $\overline{\gamma'(x)}$ die komplexen Konjugationen
von $\gamma(x)$ beziehungsweise $\gamma'(x)$ bezeichnen.\\

Es seien nun $(\omega_j, x_j)_{j=1:m}$
die f"ur die Gau"s'sche Quadraturformel ben"otigten Gewichte und St"utzstellen.
Dann setzen wir
\[
\rho(z) := \frac{1}{2\pi\iota}\sum_{j=1}^m \left(
\frac{\omega_j \cdot \gamma'(x_j)}{\gamma(x_j)-z} - \frac{\omega_j \cdot \overline{\gamma'(x_j)}}{\overline{\gamma(x_j)}-z}
\right)
\]
und erhalten nach der Substitution $\gamma(x_j) := \gamma_j$ und
$\sigma_j := \omega_j \gamma'(x_j) / (2\pi\iota)$ die gew"unschte rationale
Funktion
\[
\r\colon\C\to\C, z\mapsto\sum_{j=1}^m\left(\frac{\sigma_j}{\gamma_j - z} +
\frac{\overline{\sigma_j}}{\overline{\gamma_j} - z}\right)
\]
als Approximation der Indikatorfunktion. Hierbei ist bemerkenswert, dass die
rationale Funktion bereits in Partialbruchzerlegung vorliegt.
Setzen wir schlie"slich $B^{-1}A$ in die
rationale Funktion ein, so erhalten wir
\begin{align*}
\r(B^{-1}A) &= \sum_{j=1}^m \sigma_j (\gamma_j I - B^{-1}A)^{-1} +
\sum_{j=1}^m \overline{\sigma_j} (\overline{\gamma_j} I - B^{-1}A)^{-1}\\
&= \sum_{j=1}^m \sigma_j (\gamma_j B - A)^{-1} B +
\sum_{j=1}^m \overline{\sigma_j} (\overline{\gamma_j} B - A)^{-1} B
\end{align*}
und folglich
\[
\r(B^{-1}A)Y =
\sum_{j=1}^m \sigma_j (\gamma_j B - A)^{-1} BY +
\sum_{j=1}^m \overline{\sigma_j} (\overline{\gamma_j} B - A)^{-1} BY
\]
f"ur eine Matrix $Y\in\C^{n,m}$. Sind $A, B$ und $Y$ reellwertig, so l"asst sich dies zu
\begin{equation}\label{eq:chap4:realteilsumme}
\r(B^{-1}A)Y =
2\sum_{j=1}^m \mathfrak{Re}\left(\sigma_j (\gamma_j B - A)^{-1} BY\right)
\end{equation}
vereinfachen. Wie man diese rationale Funktion konkret in einem Algorithmus verwenden kann, besprechen wir im anschlie"senden Abschnitt.

\newpage

\section{FEAST-Algorithmus}

Im Jahr 2009 stellte E. Polizzi in \cite{polizzi} ein Verfahren vor, welches sich die Theorie der vorigen Abschnitte zu Nutze macht. Dieser \glqq[\ldots] \emph{fast and stable algorithm for solving the symmetric eigenvalue problem} [\ldots]\grqq\ \cite[Abstract]{polizzi} wurde seither weiterentwickelt und zahlreichen Analysen unterzogen.\footnote{Siehe etwa \cite{lzp},\cite{kpt} und \cite{ptep}.}\\
%Wir wollen in diesem Abschnitt die wichtigen Punkte des oben zitierten Papers skizzieren und damit das vierte Kapitel abschlie"sen.\\

Polizzi beginnt seine Ausf"uhrungen mit einem Blick in die Physik und motiviert das L"osen von Eigenwertproblemen anhand der Schr"odinger-Eigenwertgleichung
\[
\textbf{\text{H}}\Psi = E\textbf{\text{S}}\Psi
\]
welche die Fragestellung modelliert, ob gewisse Quantenobjekte, die von einer Wellenfunktion $\Psi$ beschrieben werden, kinetische Energie besitzen oder nicht. Hierbei ist \textbf{H} eine hermitesche Matrix und \textbf{S} eine symmetrisch positiv definite Matrix.\\

Der Autor merkt an, dass das L"osen solcher Systeme, insbesondere dann, wenn sie sehr gro"s sind, enorme Anforderungen an die Numeriker stellt. Die Frage ist stets, wie Eigenpaare effizient berechnet werden k"onnen und welche Genauigkeit man erwarten darf. Polizzi bewirbt seinen Algorithmus mit hoher Geschwindigkeit, Robustheit und guter Skalierbarkeit.\\ %Seinem Urteil nach, werden Methoden, wie \glqq\emph{Rayleigh-quotient multigrid}\grqq\ oder \glqq\emph{parallel Chebyshev subspace iteration}\grqq\ als weniger effizient eingesch"atzt.\\

Ausgehend von einer Ausf"uhrung "uber das Konzept der Konturintegration zur Bestimmung von Eigenpaaren, baut Polizzi seinen Algorithmus auf, welcher auf $n$-dimensionale Eigenwertprobleme $(A,B)$ mit hermiteschem oder reell symmetrischem $A$ und symmetrisch positiv definitem $B$ ausgelegt ist und im Inneren eines reellen Intervalls $[\lambda_1,\lambda_2]$ die $m\in\N$ Eigenpaare finden soll.
Sind $(\omega_j, x_j)_{j=1:k}$ die f"ur die Gau"s'sche Quadraturformel ben"otigten Gewichte und St"utzstellen, so l"asst sich der FEAST-Algorithmus wie in Algorithmus \ref{alg:chap4:feast} notieren.\\

Dieser Algorithmus stellt eine Umsetzung der beschleunigten Rayleigh-Ritz Iteration dar. Die Zeilen drei bis acht erf"ullen den Zweck des Berechnens von $\p(B^{-1}A)Y$, wobei man sich hier einer "ahnlichen Umformulierung bedient wie in Gleichung \eqref{eq:chap4:realteilsumme}.\footnote{Die entsprechende Anpassung von Algorithmus \ref{alg:chap4:beschlRrIteration} ist in \cite[365]{ptep} zu finden.}\\

Es ist zu beachten, dass sich hier durch die Verwendung einer negativ orientierten Kurve, was aus der vierten und f"unften Zeile herauszulesen ist, Vorzeichenwechsel ergeben. Da bei einer entsprechenden Parametrisierung auf der rechten Seite der Identit"at \eqref{eq:chap3:kontur} der zus"atzliche Faktor $-1$ ben"otigt wird, muss dies entsprechend bei der Konstruktion der rationalen Funktion $\p$ ber"ucksichtigt werden.

\newpage

\begin{algorithm}
\caption{FEAST-Algorithmus (Vgl. \cite[Abschnitt III]{polizzi})}\label{alg:chap4:feast}
\textbf{Input:} Eigenwertproblem $(A,B)$ wie oben beschrieben, Intervall $[\lambda_1,\lambda_2]$, Gewichte und St"utzstellen $(\omega_j, x_j)_{j=1:k}$
\textbf{Output:} Approximierte Eigenpaare $(\widetilde{\lambda_i},\xi_i)_{i=1:m}$
\begin{algorithmic}[1]
\State W"ahle $M > m$ linear unabh"angige Vektoren $Y \gets [y_i]_{i=1:M} \in\C^{n,M}$, setze $Q\gets 0_{n,M}$ und $r\gets (\lambda_1 - \lambda_2)/2$.
\State \textbf{repeat}
\State \ \ \ \ \textbf{for j=1:k}

\State \ \ \ \ \ \ \ \ Berechne $\theta_{(j)} \gets -(\pi/2)(x_j-1)$.
\State \ \ \ \ \ \ \ \ Berechne $t_{(j)} \gets (\lambda_2 + \lambda_1)/2 + re^{\iota\theta_{(j)}}$.
\State \ \ \ \ \ \ \ \ L"ose $(t_{(j)} B-A)Q_{(j)} = Y$ in $Q_{(j)}$.
\State \ \ \ \ \ \ \ \ Setze $Q\gets Q - (\omega_j/2)\cdot\mathfrak{Re}\left(re^{\iota \theta_{(j)}} Q_{(j)}\right)$.
\State \ \ \ \ \textbf{endfor}
\State \ \ \ \ Reduziere die Dimension: $\widetilde{A} \gets Q^H A Q$,
$\widetilde{B} \gets Q^H B Q$.
\State \ \ \ \ L"ose das transformierte Problem $\widetilde{A}\widetilde{X}
= \widetilde{B}\widetilde{X}\widetilde{\Lambda}$ in
$\widetilde{X}$ und $\widetilde{\Lambda} = \text{diag}(\widetilde{\lambda}_1,\ldots,\widetilde{\lambda}_M)$.
\State \ \ \ \ Setze $X \gets [\xi_i]_{i=1:M} = Q\widetilde{X}$.
\State \ \ \ \ Gilt $\widetilde{\lambda_i} \in [\lambda_1,\lambda_2]$, so gib Eigenpaar $(\widetilde{\lambda_i},\xi_i)$ aus.

\State \textbf{until} Abbruchkriterium ist erf"ullt.
\end{algorithmic}
\end{algorithm}

Die weiteren Abschnitte des Papers besch"aftigen sich vornehmlich mit numerischen Experimenten. Polizzi demonstriert anhand elektronenstrukturell bezogenen Berechnungen auf Kohlenstoffnanor"ohren die numerische Stabilit"at, Robustheit und Skalierbarkeit des FEAST-Algorithmus. Dabei vergleicht er FEAST mit ARPACK -- einer Bibliothek auf FORTAN77 basierenden Methoden zum L"osen hochdimensionaler Eigenwertprobleme.\\

Abschlie"send hebt der Autor einige Merkmale seines Algorithmus heraus, von denen an dieser Stelle eine bescheidene Auswahl vorgestellt wird. Zun"achst einmal kommt FEAST g"anzlich ohne Orthogonalisierungsmethoden aus, wie es etwa bei der unbeschleunigten Rayleigh-Ritz Iteration n"otig ist. Des Weiteren weist Polizzi auf das Potential bez"uglich paralleler Implementierbarkeit hin.\\

In Papern aus den vergangenen zwei Jahren, wie etwa in \cite[Abschnitt 4.1]{kpt}, wird auf dieses Potential genauer eingegangen. Inzwischen wurden auch Versuche unternommen, FEAST auf nicht hermitesche Eigenwertprobleme zu verallgemeinern.\footnote{Siehe \cite{kpt}.}

\newpage
\textcolor{white}{blind}

%Im dritten Kapitel wurde mit Satz \ref{thm:chap3:invariant} gezeigt, dass Ritz-Paare im Falle der $(B^{-1}A)$-Invarianz des Suchraumes $\S$ schon Eigenpaare sind. Wir k"onnten folglich die Iteration in Algorithmus \ref{alg:chap4:rrIteration} beenden, sobald eine Potenz


%\chapter{Implementation und numerische Experimente}
\chapter{Numerische Experimente}
Die folgende Proposition weist nach, dass der eben vorgestellte Algorithmus tats"achlich
ein Projektionsverfahren ist.

\begin{prop}\label{prop:projektor}
Die durch die Matrix $V_m V_m^H \in \Cnn$ induzierte lineare Abbildung
\[
P\colon \Cn \to \Cn, x\mapsto V_m V_m^H x
\]
ist eine orthogonale Projektion auf den Unterraum V und l"ost das Minimierungsproblem
\[
\min_{y\in V}\|x-y\|, x\in\Cn (\|x-P(x)\| = ...)
\]
\end{prop}

\begin{proof}
Sei $x\in\Cn$ vorgegeben. Dann gilt $P(x) = V_m V_m^H x \in \Bild(V_m) = V$ nach Konstruktion.
Da aus der Orthogonalit"at der Spalten von $V_m$
\[
P^2 = V_m V_m^H V_m V_m^H = V_m V_m^H = P
\]
folgt, ist $P$ somit eine Projektion auf $V$. Dar"uber hinaus gilt f"ur alle
Vektoren $y\in V$
\[
\langle y, x-P(x)\rangle = y^H (x - V_m V_m^H x) = y^H x - P(y)^H x = 0
\]
Also ist $x-P(x) \in V^{\bot}$ und $P$ tats"achlich eine orthogonale Projektion.
Wegen $P(x)-y \in V$ folgt die Minimalit"at aus der Absch"atzung
\begin{align*}
\|x-y\|^2 &= \|x-P(x) + P(x)-y\|^2 \\
&= \langle x-P(x) + P(x)-y, x-P(x) + P(x)-y \rangle \\
&= \|x-P(x)\|^2 + \langle x-P(x), P(x)-y \rangle + \langle P(x)-y, x-P(x)\rangle + \|P(x)-y\|^2 \\
&= \|x - P(x)\|^2 + \| P(x)-y \|^2\\
&\ge \|x-P(x)\|^2.
\end{align*}
Damit ist die Behauptung bewiesen, da Gleichheit nur mit $y=P(x)$ folgen kann.
\end{proof}

Wir wollen nun die eben erarbeitete Theorie auf das verallgemeinerte Eigenwertproblem
\[
Ax = \lambda Bx
\]
"ubertragen.


%\chapter{}%{Conclusio}
%Die folgende Proposition weist nach, dass der eben vorgestellte Algorithmus tats"achlich
ein Projektionsverfahren ist.

\begin{prop}\label{prop:projektor}
Die durch die Matrix $V_m V_m^H \in \Cnn$ induzierte lineare Abbildung
\[
P\colon \Cn \to \Cn, x\mapsto V_m V_m^H x
\]
ist eine orthogonale Projektion auf den Unterraum V und l"ost das Minimierungsproblem
\[
\min_{y\in V}\|x-y\|, x\in\Cn (\|x-P(x)\| = ...)
\]
\end{prop}

\begin{proof}
Sei $x\in\Cn$ vorgegeben. Dann gilt $P(x) = V_m V_m^H x \in \Bild(V_m) = V$ nach Konstruktion.
Da aus der Orthogonalit"at der Spalten von $V_m$
\[
P^2 = V_m V_m^H V_m V_m^H = V_m V_m^H = P
\]
folgt, ist $P$ somit eine Projektion auf $V$. Dar"uber hinaus gilt f"ur alle
Vektoren $y\in V$
\[
\langle y, x-P(x)\rangle = y^H (x - V_m V_m^H x) = y^H x - P(y)^H x = 0
\]
Also ist $x-P(x) \in V^{\bot}$ und $P$ tats"achlich eine orthogonale Projektion.
Wegen $P(x)-y \in V$ folgt die Minimalit"at aus der Absch"atzung
\begin{align*}
\|x-y\|^2 &= \|x-P(x) + P(x)-y\|^2 \\
&= \langle x-P(x) + P(x)-y, x-P(x) + P(x)-y \rangle \\
&= \|x-P(x)\|^2 + \langle x-P(x), P(x)-y \rangle + \langle P(x)-y, x-P(x)\rangle + \|P(x)-y\|^2 \\
&= \|x - P(x)\|^2 + \| P(x)-y \|^2\\
&\ge \|x-P(x)\|^2.
\end{align*}
Damit ist die Behauptung bewiesen, da Gleichheit nur mit $y=P(x)$ folgen kann.
\end{proof}

Wir wollen nun die eben erarbeitete Theorie auf das verallgemeinerte Eigenwertproblem
\[
Ax = \lambda Bx
\]
"ubertragen.


\appendix
\chapter{Notationen}\label{appNotation}
\begin{tabular}{ll}
$\N$ & Menge der nat"urlichen Zahlen \{1,2,3,\ldots\}\\
$\N_{0}$ & $\N\cup\{0\}$\\
$\R$ & Menge der reellen Zahlen\\
$\C$ & Menge der komplexen Zahlen\\
$\R^{m,n}$ & Menge der reellen $(m\times n)$-Matrizen\\
$\C^{m,n}$ & Menge der komplexen $(m\times n)$-Matrizen\\
$\R^n$ & Elemente aus $\R^{n,1}$\\
$\Cn$ & Elemente aus $\C^{n,1}$\\
$\spn{x}$ & Lineare H"ulle $\{\lambda x \mid \lambda\in\C\}$ des Vektors $x$\\
$\delta_{i,j}$ & Kronecker-Delta $\delta_{i,j} = 1$, f"ur $i=j$ und $\delta_{i,j} = 0$, f"ur $i\neq j$\\
$I_n$ & Einheitsmatrix $[\delta_{i,j}]_{i,j=1:n}$\\
$0_n$ & Nullmatrix des $\Cnn$\\
$0_{m,n}$ & Nullmatrix des $\C^{m,n}$\\
$A^H$ & Komplex konjugiert Transponierte der Matrix $A$\\
$\gg$ & Erheblich gr"o"ser als\\
$\bot$ & Steht orthogonal / unit"ar auf\\
$\bot_B$ & Steht $B$-orthogonal / $B$-unit"ar auf\\
$\U^{\bot_B}$ & $B$-orthogonales Komplement des Unterraums $\U$\\
$\langle x,y\rangle_B$ & Schreibweise f"ur $x^H B y$, f"ur eine HPD-Matrix $B$\\
$\|\cdot\|_B$ & Von der Matrix $B$ induzierte Norm: $\|x\|_B := \sqrt{x^H B x}$\\
$\|x\|_2$ & Die euklidische Norm $\sqrt{x^H x}$\\
$\|A\|_2$ & Induzierte $2$-Norm f"ur Matrizen (genauer)\\
$\mathfrak{Re}$ & Realteil komplexer Zahlen
\end{tabular}


\chapter{S"atze und Beweise}
\begin{thm}[Singul"arwertzerlegung]\label{thm:svd}
Ist $A\in\C^{m,n}$ eine $r$-rangige Matrix, so existieren unit"are Matrizen $U\in\C^{m,m}, V\in\Cnn$
und $\Sigma^2_+ =\text{diag}(\sigma^2_1,\ldots,\sigma^2_r)\in\C^{r,r}$ mit der Eigenschaft $\sigma_1 \ge \ldots \ge \sigma_r > 0$, die
eine Faktorisierung der Art
\[
A = U \Sigma V^H
\]
erm"oglichen. Dabei ist
\[
\Sigma = \begin{bmatrix} \Sigma_+ & 0_{r,n-r} \\
0_{m-r,r} & 0_{m-r,n-r} \end{bmatrix}.
\]
\textcolor{red}{nochmal bearbeiten mit den quadraten}

\end{thm}

\begin{thm}[Spektralsatz f"ur hermitesche Matrizen]\label{thm:appTheorems:Spektralsatz}
bla
\end{thm}

\begin{thm}[Existenz der Cholesky-Zerlegung]\label{thm:appTheorems:Cholesky}
bla
\end{thm}

\begin{thm}[Jordan'sche Normalform]\label{thm:appTheorems:Jordan}
blubb
\end{thm}

\begin{thm}[Cauchy'scher Integralsatz]\label{thm:appTheorems:Cauchy}
bla
\end{thm}

\begin{thm}[Potenzmethode]\label{thm:appTheorems:Potenzmethode} Es sei $A\in\Cnn$ eine hermitesche Matrix mit Eigenpaaren $(\lambda_i,x_i)_{i=1:n} \in \R\times\Co$ und $n\ge 2$.
Es gelte weiter $|\lambda_1| > |\lambda_j|$ f"ur alle
$j > 1$. Ist $y_0 \in\Co$, so konvergiert die Folge
$(y_k)_{k\in\N}$ mit
\[
y_{(k+1)} = \frac{1}{\|A^{k+1} y_{(0)}\|} A^{k+1}y_{(0)}
\]
gegen einen normierten Eigenvektor zum Eigenwert $\lambda_1$
\end{thm}

\begin{proof}

\end{proof}


\chapter{Nebenrechnungen}
\begin{comp}[Zu Satz \ref{thm:chap3:projector}]\label{comp:app:projector}

\end{comp}


%\chapter{Quellcode}



\nocite{*}
\printbibliography

\end{document}
