\documentclass[11pt, twoside]{report}
%no headers, no twosided layout by default

% -------------------------------- %
% ------- P A C K A G E S -------- %
% -------------------------------- %

\usepackage{style/myba}

% -------------------------------- %
% ------- D O C U M E N T -------- %
% -------------------------------- %

\begin{document}

% titlepage
\pagenumbering{roman}
\begin{titlepage}
  \begin{center}
    \vspace*{1cm}

    \Huge
    \textbf{Der FEAST-Algorithmus}

    \vspace{0.5cm}
    \LARGE
    Ein Verfahren zum Lösen allgemeiner Eigenwertprobleme

    \vspace{1.5cm}

    \textbf{Thorsten Matthias Lucke}\\
    Betreuer

    \vfill

    Arbeit zur Erlangung des akademischen Grades\\
    Bachelor of Science

    \vspace{0.8cm}

    \includegraphics[width=0.4\textwidth]{images/logo_tu}

    \Large
    Department Name\\
    Technische Universität Berlin\\
    %Country\\
    \today

  \end{center}
\end{titlepage}


% preface
\thispagestyle{empty}
\textcolor{white}{bla} \\
\vfill
\textit{Dem Institut f"ur Mathematik\\
der Technischen Universit"at Berlin\\
vorgelegte Bachelorarbeit von} \\
Thorsten Matthias Lucke.\\

\textit{Betreuer und Erstgutachter}:\\
Prof. Dr. J"org Liesen\\

\emph{Zweitgutachter}:\\ Prof. Dr. Christian Mehl\\

\emph{Datum der Einreichung}:\\
Freitag, der 13. Oktober 2017


\chapter*{Erkl"arung}
Die selbst"andige und eigenh"andige Anfertigung versichert an Eides statt\\

Berlin, den 13. Oktober 2017\\

\rule{5cm}{1pt}\\
Unterschrift


%Ich versichere, dass ich die vorliegende Arbeit selbst"andig verfasst und keine anderen als die
%angegebenen Quellen und Hilfsmittel benutzt habe. Ich reiche sie erstmals als Pr"ufungsleistung ein.
%Mir ist bekannt, dass ein Betrugsversuch mit der Note \glqq nicht ausreichend\grqq \ geahndet wird und im Wiederholungsfall zum Ausschluss
%von der Erbringung weiterer Pr"ufungsleistungen f"uhren kann.

%Name:

%Vorname:

%Matrikelnummer:

%\textcolor{white}{filler}

%Berlin, den

%Signatur
\newpage

\chapter*{}

\newpage

\chapter*{}

\textit{Meiner Familie,\\
der Thesisselbsthilfegruppe,\\
der H"angemathe und dem 1 Caf\'e}

\chapter*{Danksagung}
Obschon die vorliegende Arbeit der eigenen Feder entstammt, h"atte dieses Werk kaum vollendet werden k"onnen,
w"aren nicht einige H"urden beiseite geschafft worden.
So sehe ich es als meine Pflicht, den Helferinnen und Helfern des \glqq R"aumungsdienstes\grqq\ ein paar Worte der Ehrung
zukommen zu lassen.\\

Der erste Dank gilt meinen Betreuern Prof. Dr. J"org Liesen und Prof. Dr. Christian Mehl deren Sprechstunden
stets eine Quelle der Inspiration waren. Immer hilfsbereit standen sie mir fachlich zur Seite und lie"sen mich von ihrer Erfahrung profitieren.\\

Ein weiterer Dank geht an die Mitarbeiterinnen und Mitarbeiter des studentischen Mathekaffees \glqq 1 Caf\'e\grqq. Welche Qualen blieben mir durch
die etlichen Teest"undchen, Schokoriegel und zum Schlafen einladenden Sofas erspart! Im selben Atemzug
danke ich den Mitgliedern der \glqq H"angemathe\grqq, deren kritisches Hinterfragen und Diskussionsfreudigkeit stets anregend war.\\

Schlie"slich richte ich meinen Dank an alle Lektoren, die akribisch jeden noch so hinterh"altig versteckten
Fehler entdeckt haben und damit das Lesen dieser Arbeit zu einem gr"o"seren Vergn"ugen machen.

% content
\tableofcontents
\chapter*{}
% chapters
\chapter{Einleitung}%{Introductio}
\pagenumbering{arabic}
Das L"osen von Eigenwertproblemen ist eine Standarddisziplin in der
numerischen linearen Algebra. Die konkrete Problemstellung dabei lautet: Zu gegebenen Matrizen $A,B\in\Cnn$ sollen Paare $(\lambda, x)$ mit $\lambda\in\C$ und $x\in\Co$ gefunden werden, welche
der Gleichung
\begin{equation}\label{chap1:eq:eigenproblem}
Ax = \lambda Bx
\end{equation}

gen"ugen. Solchen \emph{Eigenwertgleichungen} begegnet man in ganz unterschiedlichen Kontexten.
So sind sie beispielsweise bei der Bestimmung von Eigenfrequenzen oder dem Ermitteln von Fixpunkten beim
Rotieren eines Fu"sballs\footnote{Hier wird auf den bekannten
\emph{Satz vom Fu"sball} angespielt. Dieser besagt, dass auf einem Fußball
zwei Punkte existieren, die zu Spielbeginn und zur Halbzeit
an der gleichen Stelle liegen -- informell formuliert.} ebenso wie beim
Untersuchen des PageRanks\footnote{
Siehe Abschnitt zwei in ~\cite{page}.
} einer Website von
Bedeutung. Entsprechend strotz der Kanon von angebotenen numerischen
L"osungsmethoden von Vielfalt und Virtuosit"at.\footnote{Dies best"atigt sich beispielsweise bei einem Blick in das Inhaltsverzeichnis von ~\cite{stewart}.}\\

Nun mag der Fall eintreten, da es notwendig wird, lediglich eine Teilmenge
aus der Menge aller Eigenpaare zu untersuchen. Betrachten wir zur Illustration folgende Abbildung.

\begin{figure}[h!]
  \centering
  \resizebox{.4\linewidth}{!}{\includegraphics{images/Cat}}
  \caption{\glqq Mind Games of Ozzy\grqq\ von Andy Prokh.}\label{chap1:im:cat}
\end{figure}

%Diese Foto ben"otigt in der Originalgr"o"se einen Speicherplatz von \textcolor{red}{9000mb}.%\footnote{Das Foto aus Abbildung \ref{chap1:im:cat} wurde dabei mit einer \textcolor{red}{tollen Kamera} gemacht.}
Man stelle sich vor, man wolle dieses Bild komprimieren, um den ben"otigten Speicherplatz zu reduzieren.
Dies l"asst sich beispielsweise mit einer sogenannten \emph{Singul"arwertzerlegung} -- von Numerikern liebevoll als das \emph{Schweizer Taschenmesser der linearen Algebra} bezeichnet -- bewerkstelligen.\\

Da es sich bei Abbildung \ref{chap1:im:cat} um ein Schwarz-Wei"s-Bild handelt, l"asst sie sich als eine Matrix $M$ auffassen, deren Eintr"age die Grauwerte der einzelnen Pixel repr"asentieren.
Unter Anwendung des Satzes von der Singul"arwertzerlegung\footnote{Eine Formulierung ohne Beweis ist mit Satz \ref{app:thm:svd} -- zu finden im Anhang -- gegeben.} l"asst sich nun das Bild der Katze durch eine Folge von Matrizen niedrigeren Ranges approximieren.
Man berechnet hierf"ur die Wurzeln der von Null verschiedenen Eigenwerte von $M^H M$ und verwendet dann diese sogenannten \emph{Singul"arwerte} um die Matrix $M$ zu rekonstruieren.
Auf weitere Details "uber den hinter diesem Verfahren stehenden Algorithmus wollen wir an dieser Stelle verzichten.\\

Die Matrix $M$ hat in unserem Beispiel mehr als 700 Singul"arwerte. Die Frage ist nun, ob wir uns von einigen Singul"arwerten trennen k"onnen und dennoch eine akzeptable Bildqualit"at aufrecht erhalten k"onnen.
Insbesondere ist wichtig zu wissen, welche dieser Werte entscheidend f"ur die Rekonstruktion sind.\\

Dazu nehmen wir uns die $n\in\N$ Singul"arwerte $\sigma_1,\ldots,\sigma_n$ her. Ohne Einschr"ankung seien diese so nummeriert, dass $\sigma_1 \ge \ldots \ge \sigma_n$ gilt. Sollte dies nicht der Fall sein, nummerieren wir einfach um.
Nun filtern wir nach eigenen Kriterien Teilmengen aus den Singul"arwerten, beziehungsweise den Eigenwerten von $M^H M$ heraus und rekonstruieren mit deren Hilfe die Abbildung \ref{chap1:im:cat}.\\

Es sei an dieser Stelle darauf hingewiesen, dass die folgendenen Beurteilungen der Bildqualit"at nicht mathematisch begr"undet sind.
Die Eindr"ucke entstammen dem pers"onlichen Empfinden des Autors, sowie den Aussagen einer kleinen Testgruppe von Studierenden des Fachs Mathematik.\\

Beginnen wir mit einer mehr oder weniger willk"urlichen Auswahl von Singul"arwerten (SW).

\begin{figure}[h!]
\center
\begin{subfigure}[c]{.3\textwidth}
\includegraphics[width=.9\linewidth]{images/Cat10-30}
\subcaption{SW $\sigma_{10},\ldots,\sigma_{30}$.}
\end{subfigure}
\begin{subfigure}[c]{.3\textwidth}
\includegraphics[width=.9\linewidth]{images/Cat40-90}
\subcaption{SW $\sigma_{40},\ldots,\sigma_{90}$.}
\end{subfigure}
\begin{subfigure}[c]{.3\textwidth}
\includegraphics[width=.9\linewidth]{images/Cat100-400}
\subcaption{SW $\sigma_{100},\ldots,\sigma_{400}$.}
\end{subfigure}

\caption{Rekonstruktionsversuch von Abbildung \ref{chap1:im:cat}.}\label{chap1:im:catinterval}
\end{figure}

Die in Abbildung \ref{chap1:im:catinterval} gew"ahlte Filtrierung erscheint etwas ungl"ucklich, da das urspr"ungliche Bild kaum wieder zu erkennen ist.
Versuchen wir unser Gl"uck daher mit einer anderen Wahl von Singul"arwerten.

\newpage

\begin{figure}[h!]
\center
\begin{subfigure}[c]{.3\textwidth}
\includegraphics[width=.9\linewidth]{images/Cat1}
\subcaption{SW $\sigma_1$.}
\end{subfigure}
\begin{subfigure}[c]{.3\textwidth}
\includegraphics[width=.9\linewidth]{images/Cat5}
\subcaption{SW $\sigma_1,\ldots, \sigma_5$.}
\end{subfigure}
\begin{subfigure}[c]{.3\textwidth}
\includegraphics[width=.9\linewidth]{images/Cat20}
\subcaption{SW $\sigma_1,\ldots, \sigma_{20}$.}
\end{subfigure}

\vspace{0.4cm}
\begin{subfigure}[c]{.3\textwidth}
\includegraphics[width=.9\linewidth]{images/Cat50}
\subcaption{SW $\sigma_1,\ldots, \sigma_{50}$.}
\end{subfigure}
\begin{subfigure}[c]{.3\textwidth}
\includegraphics[width=.9\linewidth]{images/Cat100}
\subcaption{SW $\sigma_1,\ldots, \sigma_{100}$.}\label{im:chap1:subE}
\end{subfigure}
\begin{subfigure}[c]{.3\textwidth}
\includegraphics[width=.9\linewidth]{images/Cat300}
\subcaption{SW $\sigma_1,\ldots, \sigma_{300}$.}\label{im:chap1:subF}
\end{subfigure}

\caption{Approximation von Abbildung \ref{chap1:im:cat} mittels Singul"arwertzerlegung.}
\end{figure}

Dieser Versuch wirkt "uberzeugender. Bereits bei Figur \ref{im:chap1:subE} ist ein genauer Blick erforderlich, um die Makel der Komprimierung zu erkennen.
Bei der Verwendung der ersten 300 SW ist es nahezu unm"oglich einen Unterschied zum Original festzustellen.
Es ist also zu "uberlegen, ob man sich mit der Qualit"at der Figur \ref{im:chap1:subF} zufrieden geben m"ochte und statt Abbildung \ref{chap1:im:cat} nicht einfach dieses abspeichert.\\

Nun ist das Approximieren von Bildern ein sehr spezieller Fall des Filterns von Eigenwerten und die Berechnung der Singul"arwertzerlegung nicht immer zweckm"a"sig oder m"oglich.
Daher setzt sich diese Arbeit mit Alternativen auseinander, die dem Filtern dienlich sind. Dabei werden neben den mathematischen Ideen dieser Alternativen auch M"oglichkeiten der Implementation vorgestellt.\\

Im dritten Kapitel werden zun"achst zwei Methoden pr"asentiert, die als Werkzeuge zum Filtern von Eigenpaaren dienlich sind.
Daran wird sich nach einer kurzen analytischen Illustration dieser Verfahren eine Diskussion anschlie"sen, bei der untersucht wird, wie es um die praktische Umsetzbarkeit der Verfahren bestellt ist.
Wir werden feststellen, dass sich die Algorithmen kombinieren und modifizieren lassen und zu einem Verfahren f"uhren, welches in der Literatur als FEAST-Algorithmus gehandelt wird. Zum Abschluss begeben wir uns in das \glqq Numerik-Labor\grqq\ und werden die Grenzen der Algorithmen austesten.\\

Bevor es konkreter wird, erinnert der folgende Abschnitt an einige mathematische Grundlagen, die als helfende Handreichung das Lesen dieser Schrift mehr zur Freude, denn zur Schikane machen soll.


\chapter{Mathematische Grundlagen}
Ein M"oglicheit das eingangs geschilderte Eigenwertproblem zu l"osen, ist
das Anwenden einer Klasse von Verfahren, die sich \emph{Rayleigh-Ritz-Verfahren}
nennen. Die Idee besteht darin die Eigenr"aume/ den Eigenraum \textcolor{red}{(was nun genau? bekomme ich alle
Eigenr"aume? nur einen? was genau approximiere ich?)} durch eine Folge von Unterr"aumen
zu approximieren, die durch orthogonale Projektionen konstruiert werden.\textcolor{red}{was ist genau
mit approximieren gemeint? in welchem sinne? und ist das "uberhaupt korrekt?} In diesem
Abschnitt wird die mathematische Idee dieser Projektionsverfahren vorgestellt und
die Verwendung von Filtern motiviert \textcolor{red}{kam die motivation nicht schon
in der einleitung?}

\section{Orthogonale Projektionsverfahren}
\footnote{Dieser Abschnitt orientiert sich in Formulierung und Dramaturgie an Abschnitt 4.3.1 aus Y. Saad Solving large evp. Hier verallgemeinern wir aber die Konzepte direkt auf allg. ewp.}
Bevor wir das allgemeine Eigenproblem untersuchen, konzentrieren wir uns zun"achst auf das gew"ohnliche Eigenproblem
\[
Ax = \lambda x% \textcolor{red}{variablen wieder neu einf"uhren? oder in der einleitung: im folgenden sei stets...}
\]
und betrachten f"ur eine Zahl $m\in\N$ mit $m\le n$
einen $m$-dimensionalen Unterraum $V\subset\Cn$. Dieser zun"achst nicht
n"aher bestimmte \emph{Suchraum} wird als Grundlage f"ur das Projektionsverfahren gew"ahlt.
Gesucht sind nun Paare $(\widetilde{\lambda}, \widetilde{x})\in\C\times V$ -- die wir als approximierte L"osungen des Eigenproblems verstehen wollen --
welche die Eigenschaft
\begin{equation}\label{eq:orthogonal}
\langle A\widetilde{x} - \widetilde{\lambda}\widetilde{x}, v\rangle=0
\end{equation}
f"ur alle $v\in V$ erf"ullen.\footnote{Paare $(\widetilde{\lambda}, \widetilde{x})$ der
eben beschriebenen Art werden auch \emph{Ritz-Paare} genannt.}
Das Residuum $A\widetilde{x} - \widetilde{\lambda}\widetilde{x}$
soll also orthogonal auf dem Suchraum stehen.\\%residuum bzgl was???

Angenommen, eine Orthonormalbasis $\{v_i\}_{i=1:m}\subset V$ ist gegeben.
Dann l"asst sich die Forderung \eqref{eq:orthogonal} mit Hilfe der Matrix $\pmb{V_m} :=[v_i]_{i=1:m}\in\C^{n,m}$, einem geeigneten Vektor
$y\in\C^m$ und der Substitution $V_m y=\widetilde{x}$ in das Gleichungssystem
\[
V_m^H(AV_m y - \widetilde{\lambda} V_m y) = 0
\]
"uberf"uhren. Als direkte Konsequenz dieser Substitution erhalten wir unter Ausnutzung der Orthogonalit"at der Spalten von $V_m$ mit
\begin{equation}\label{eq:transform}
V_m^H A V_m y = \widetilde{\lambda}y.
\end{equation}
ein neues Eigenproblem. Jedes Eigenpaar $(\widetilde{\lambda},y)$ von \eqref{eq:transform}
liefert dann ein Ritz-Paar $(\widetilde{\lambda}, V_m y)$ des gew"ohnlichen
Eigenproblems bez"uglich des Suchraums $V$. Man kann zeigen, dass im Falle der Invarianz
von $V$ unter $A$ jedes Ritz-Paar von $A$ sogar ein Eigenpaar ist. \textcolor{red}{beweis eventuell dazu nehmen}\\

Ausgehend von diesen "Uberlegungen l"asst sich das Rayleigh-Ritz-Verfahren wie folgt
beschreiben:

\begin{lstlisting}[caption = Rayleigh-Ritz-Verfahren (Vgl. Saad Algo. 4.5), captionpos=b]
Berechne eine ONB {v1,...,vm} von V und setze Vm = [v1,...,vm]
Berechne Eigenwerte von Vm^H * A * Vm und wähle die gewünschten aus
Berechne die zu den eben ermittelten Eigenwerte zugehörigen EVektoren
Berechne die Ritz Paare
\end{lstlisting}

Die folgende Proposition weist nach, dass der eben vorgestellte Algorithmus tats"achlich
ein Projektionsverfahren ist.

\begin{prop}
Die durch die Matrix $V_m V_m^H \in \Cnn$ induzierte lineare Abbildung
\[
P\colon \Cn \to \Cn, x\mapsto V_m V_m^H x
\]
ist eine orthogonale Projektion auf den Unterraum V und l"ost das Minimierungsproblem
\[
\min_{y\in V}\|x-y\|.
\]
\end{prop}

\begin{proof}
Sei $x\in\Cn$ vorgegeben. Dann gilt $P(x) = V_m V_m^H x \in \Bild(V_m) = V$ nach Konstruktion.
Da aus der Orthogonalit"at der Spalten von $V_m$
\[
P^2 = V_m V_m^H V_m V_m^H = V_m V_m^H = P
\]
folgt, ist $P$ somit eine Projektion auf $V$. Dar"uber hinaus gilt f"ur alle
Vektoren $y\in V$
\[
\langle x-P(x), y\rangle = y^H (x - V_m V_m^H x) = y^H (I - V_m V_m^H)x
= y^H x - y^H V_m V_m^H x = y^H x - y^H x = 0
\]
\textcolor{red}{Bem.: $y^H V_m V_m^H = P(y)^H = y^H$, da $y \in V$.}
Also ist $x-P(x) \in V^{\bot}$ und $P$ tats"achlich eine orthogonale Projektion.
Wegen $P(x)-y \in V$ folgt die Minimalit"at aus der Absch"atzung
\begin{align*}
\|x-y\|^2 &= \|x-P(x) + P(x)-y\|^2 \\
&= \langle x-P(x) + P(x)-y, x-P(x) + P(x)-y \rangle \\
&= \|x-P(x)\|^2 + \langle x-P(x), P(x)-y \rangle + \langle P(x)-y, x-P(x)\rangle + \|P(x)-y\|^2 \\
&= \|x - P(x)\|^2 + \| P(x)-y \|^2\\
&\ge \|x-P(x)\|^2.
\end{align*}
Damit ist die Behauptung bewiesen, da Gleichheit nur mit $y=P(x)$ folgen kann.
\end{proof}
Projektio $P=V$... siehe skript liesen

Auf diesen "Uberlegungen aufbauend, l"asst sich nun das folgende Verfahren konstruieren.
blabla saad... s. 98.

\textcolor{red}{WIE DAS ALLGEMEINE PROBLEM EINBRINGEN? WO IST DANN B? Direkt allgemein machen?}
\label{chap2}

%\chapter{Filter und Beschleuniger}
\chapter{Filtertechniken} %oder Filtern mit Projektionen
In diesem Kapitel soll demonstriert werden, dass Konturintegration und
Rayleigh-Ritz-Verfahren in einem engen Zusammenhang stehen \textcolor{red}{wird sich noch zeigen}.
Um dies zu Untermauern, orientieren sich die folgenden Abs"atze an den Ausf"uhrungen
von Ping Tak Peter Tang und Eric Polizzi in ~\cite{ptep}. Allerdings wird
die Notation im Sinne der Konsistenz an einigen Stellen abweichen.

\section{Beschleunigtes Rayleigh-Ritz Verfahren}\label{chap3:beschrr}

Betrachten wir also wie bisher das verallgemeinerte Eigenwertproblem mit zwei
komplexwertigen, hermiteschen $(n\times n)$-Matrizen $A$ und $B$ und fordern
zu"satzlich die positive Definitheit von $B$. Zu diesem Duo gesellt sich nun
mit $p(B^{-1}A)$ ein Polynom in $B^{-1}A$, welches wir benutzen um gem"a"s dem
oben zitierten Paper den Algorithmus () aus dem vorigen Kapitel wie folgt zu "andern.

\begin{algorithm}\label{alg:beschlrr}
\caption{Beschleunigtes iteratives Rayleigh-Ritz-Verfahren}\label{euclid}
\begin{algorithmic}[1]
\State W"ahle $m$ Zufallsvektoren $Q_{(0)} \gets [q_i]_{i=1:m} \in\C^{n,m}$.
Setze $k \gets 1$.
\State \textbf{repeat}
\State \ \ \ \ Approximiere den Unterraumprojektor: $Y_{(k)} \gets p(B^{-1}A)Q_{(k-1)}$
\State \ \ \ \ Reduziere die Dimension: $\widetilde{A}_{(k)} \gets Y_{(k)}^H A Y_{(k)}$,
$\widetilde{B}_{(k)} \gets Y_{(k)}^H B Y_{(k)}$.
\State \ \ \ \ L"ose das transformierte Problem $\widetilde{A}_{(k)}\widetilde{X}_{(k)}
= \widetilde{B}_{(k)}\widetilde{X}_{(k)}\widetilde{\Lambda}_{(k)}$ in
$\widetilde{X}_{(k)}$ und $\widetilde{\Lambda}_{(k)}$.
\State \ \ \ \ Setze $Q_{(k)} \gets Y_{(k)}\widetilde{X}_{(k)}$.
\State \ \ \ \ $k \gets k+1$.
\State \textbf{until} Abbruchkriterium ist erf"ullt.
\end{algorithmic}
\end{algorithm}

Das Polynom $p$ wird in diesem Kontext auch als \emph{Filter} oder \emph{Beschleuniger}
bezeichnet. Von dessen Wahl h"angt n"amlich ab, ob und wie gut Eigenpaare approximiert
werden. Es ist sogar m"oglich, gezielt solche Eigenpaare zu finden, wie sie im
Abschnitt \ref{sec:kontur} gesucht waren.\\

Um dies einzusehen, greifen wir erneut die Notationen aus besagtem Kapitel auf:
Es sei $[\lambda_1, \lambda_2]$ dasjenige Intervall, auf dem die Eigenwerte und
korresponierenden Eigenvektoren gefunden werden sollen und $X_k$ sei diejenige
Matrix dessen Spalten aus gerade diesen Eigenvektoren besteht. Es wurde bereits
diskutiert, dass die durch Konturintegration ermittelten Eigenvektoren
$B$-orthogonal sind. Damit l"asst sich also -- wie im Satz \ref{thm:projektor}
bewiesen -- der Spektralprojektor $P = X_k X_k^H B$ konstruieren. Falls nun
$p(B^{-1}A)$ mit diesem Projektor "ubereinstimmt, dann terminiert Algorithmus
\ref{alg:beschlrr} im Falle der Vollrangigkeit von $Y_{(1)}$ nach einer Iteration.\footnote{
Hierbei ist entscheidend, dass die Anzahl der Spalten von $Q$ mit der Anzahl der
Eigenpaare "ubereinstimmt, die auf dem Intervall zu finden sind (Vgl. ~\cite[356]{ptep}).}
Dies folgt unter Ausnutzung der Invarianz des Bildes von $PQ_{(0)}$ unter $B^{-1}A$
aus dem Satz $\ref{thm:invariant}$.\\

Da der Spektralprojektor in den meisten F"allen unbekannt sein d"urfte, liegt
die Idee nahe, ihn zumindest zu approximieren. Da in ~\cite[356]{ptep} bemerkt wird,
dass dies gut funktioniert, wenn $p$ eine durch \emph{Gau"s-Legendre-Quadratur}
konstruierte rationale Funktion ist, wird sich das folgende Intermezzo mit eben dieser
Klasse von Funktionen besch"aftigen, bevor wir mit der Konstruktion des Projektors
fortfahren.


\section{Rationale Funktionen}

Ausgehend von zwei Polynomen $p, q\in\C [t]$ mit
\[
p := \sum_{k=0}^n p_k t^k \text{ \ und\ } q := \sum_{k=0}^n q_k t^k
\]
definieren wir eine rationale Funktion $\rho\colon\C\setminus{N_q}\to\C$ verm"oge
\[
\rho(t) := \frac{p(t)}{q(t)}
\]
und identifizieren wie "ublich die Unbestimmte $t$ mit den Argumenten von $p$ und $q$.
Dabei ist $N_q := \{t\in\C \mid q(t) = 0\}$. Eine rationale Funktion wird als \emph{echt gebrochen} bezeichnet,
falls die Bedingung $\Grad(p) < \Grad(q)$ erf"ullt ist.



\section{Approximation des Spektralprojektors}

Nun da die f"ur den weiteren Verlauf der Arbeit wichtigen Eigenschaften rationaler
Funktionen wiederholt wurden, widmen wir uns der Approximation des Spektralprojektors
$P = X_k X_k^H B$. Daf"ur setzen wir den in Abschnitt \ref{chap3:beschrr} bereits begonnen Gedankengang aus ~\cite{ptep}
fort und "ubernehmen die zu letzt vereinbarten Voraussetzungen und Notationen.\\

Wenden wir uns also wieder dem reellen Intervall $I := [\lambda_1, \lambda_2]$ zu. Das Ziel ist
die Konstruktion einer rationalen Funktion $\rho\colon\C\to\C$ mit $\rho|_\R \subseteq \R$,
die auf $I$ n"aherungsweise der Indikatorfunktion von $I$ entspricht. Dazu
bem"uhen wir die Cauchy'sche Integraldarstellung der Indikatorfunktion und
wandeln diese mit Hilfe numerischer Quadraturformeln in die gew"unschte
rationale Funktion $\rho$ um.\\

Zu"achst zur Indikatorfunktion: Ist $c\in\R$ der Mittelpunkt des Intervalls $I$ und
$r$ der Abstand des Mittelpunktes zum Rand des Intervalls, dann entspricht die Menge
\[
\mathcal{C} := \{z\in\C : |z-c|\le r\}
\]
gerade einer Kreisscheibe mit Radius $r$ um $c$. F"ur eine auf $\mathcal{C}$
holomorphe Funktion $f\colon\C\to\C$ gilt dann gem"a"s der Cauchy'schen Integralformel
\[
f(z) = \frac{1}{2\pi\iota}\int_{\partial \mathcal{C}}\frac{f(z)}{\omega-z}\text{ d}\omega
\]
f"ur jedes $z$ im offenen Inneren von $\mathcal{C}$ (Vgl. ~\cite[20]{jaenich}).

\section{Der FEAST-Algorithmus}


%\chapter{Rationalfunktionelle Filter}
\chapter{Hilfsmittel: Rationale Funktionen}%rationale beschleuniger?
Nachdem das vorangegangene Kapitel Ideen zum Filtern von Eigenpaaren theoretisch beleuchtet hat, werden wir uns nun mit der Frage der praktischen Umsetzbarkeit besch"aftigen.
Im Mittelpunkt wird dabei die Konstruktion geeigneter Suchr"aume stehen, welche im Rayleigh-Ritz-Verfahren zum Einsatz kommen sollen.
Es wird sich zeigen, dass die Konturintegration hierbei ein n"utzliches Hilfsmittel darstellt.

\section{Rayleigh-Ritz Iteration}\label{chap4:beschrr}

%Diese Erkenntnis ist aus algorithmischer Sicht h"ochst interessant. In seiner iterativen Variante kann das Rayleigh-Ritz Verfahren abgebrochen werden, sobald die Suchraumiterierte Wandelt man das Rayleigh-Ritz Verfahren der Art ab, dass

%Es wird sich zu einem sp"ateren Zeitpunkt herausstellen, dass diese Erkenntnis aus algorithmischer Sicht h"ochst n"utzlich ist.
%Die Rayleigh-Ritz-Methode l"asst sich nämlich in ein iteratives Verfahren umwandeln, welches in jedem Schritt den Suchraum "andert. Ist die Suchraumiterierte irgendwann einmal $A$-invariant, so kann der Algorithmus abgebrochen werden. \textcolor{red}{Krylow Raum...}\\

Bei der Behandlung des Rayleigh-Ritz Verfahrens wurde angedeutet, dass das Einbinden einer geeigneten Iterationsvorschrift dabei helfen kann, die G"ute von errechneten Ritz-Paaren zu verbessern.
Hierbei ist \glqq G"ute\grqq\ nat"urlich in Abh"angigkeit vom Kontext zu bewerten. Im Folgenden wollen wir uns genauer mit dieser omin"osen Interationsvorschrift auseinander setzen und beginnen die Herleitung bei einem sehr einfach umsetzbaren Verfahren zur Bestimmung von Eigenpaaren.\\

%\footnote{Um besser zu verstehen orienteiren wir uns bei der Einf"uhrung des Algs an der Dramaturgie von Saad blabla 115ff}

Ausgangspunkt f"ur unsere Betrachtungen ist ein gew"ohnliches Eigenwertproblem mit einer von Null verschiedenen hermiteschen Matrix $A\in\Cnn$. Bei der sogenannten \emph{Potenzmethode} wird
ausgehend von einem Startvektor $y_{(0)}\in\Co$ in jeder Iteration der Vektor
\[
y_{(k+1)} = \frac{1}{\|A^{k+1} y_{(0)}\|} A^{k+1}y_{(0)}
\]
oder dazu "aquivalent
\[
y_{(k+1)} = \frac{1}{\|Ay_{(k)}\|} Ay_{(k)}
\]
berechnet. Dieser Vorgang wird wiederholt, bis gewisse Abbruchkriterien erf"ullt sind. Man kann zeigen, dass die Folge der Iterierten gegen einen zum betragsm"a"sig gr"o"sten Eigenwert geh"orenden Eigenvektor konvergiert.\footnote{Eine exaktere Formulierung bietet Satz \ref{thm:appTheorems:Potenzmethode} im Anhang \ref{appTheorems}.}
%, begn"ugen wir uns an dieser Stelle mit folgender Plausibilit"atsbetrachtung, welche in "ahnlicher Form in ~\cite[56]{stewart}
%zu finden ist.\\

%Seien $(\lambda_i, x_i)_{i=1:n}$ die Eigenpaare von $A$. Aufgrund der Hermitizit"at bilden dann die Eigenvektoren eine Basis des $\Cn$.
%Folglich existieren komplexwertige Skalare $\alpha_1,\ldots,\alpha_n$ mit
%\[
%y_{(0)} = \sum_{i=1}^n \alpha_i x_i.
%\]
%Wenden wir nun die $k$-te Potenz von $A$ auf $y_{(0)}$ an, ergibt sich daher wegen $A^k x_i = \lambda^k x_i$
%\begin{equation}\label{eq:chap3dominant}
%A^k y_{(0)} = \sum_{i=1}^n \alpha_i \lambda_i^k x_i.
%\end{equation}
%Wir wollen ohne Einschr"ankung annehmen, dass $|\lambda_1| \ge |\lambda_i|$ f"ur alle $i$ mit
%$1<i\le n$ gilt. Gegebenenfalls nummerieren wir die Eigenpaare und Skalare um. Dann wird f"ur gr"o"ser werdende $k$ die rechte Seite von \eqref{eq:chap3dominant} durch den Term $\alpha_1 \lambda_1^k x_1$ dominiert. In Kombination mit der Normierung f"uhrt dies letztlich
%zur behaupteten Approximierung.\footnote{Ein formaler Beweis ist im Anhang zu finden.}\\

\newpage
Da uns daran gelegen ist, nicht nur mit einzelnen Elementen des $\Cn$ zu arbeiten sondern mit Matrizen zu hantieren, m"ussen wir ein allgemeinere Form der Potenzmethode betrachten. Dazu w"ahlen diesmal eine Startmatrix $Y_{0}\in\C^{n,m}$ vollen Ranges und berechnen die $k$-te Iterierte mit
\[
Y_{(k)} = A^k Y_{(0)}.
\]
Bei der Normalisierung ist allerdings Vorsicht geboten: In seinen Abhandlungen "uber Unterraumiterationen
merkt Y. Saad in ~\cite[Abschnitt 5.1]{saad} an,
dass es beim ungeschickten Normalisieren vorkommen kann, dass die Spalten von $Y_{(k)}$ zunehmend ihre lineare Unabh"angigkeit verlieren.\footnote{Dies ist unmittelbar einzusehen, wenn man bedenkt, dass jede Spalte gegen einen dominanten Eigenvektor konvergiert.}\\

Anstatt jede Spalte von $Y_{(k)}$ separat zu normalisieren, wird eine QR-Zerlegung\footnote{Eine Formulierung des Satzes "uber die Existenz der QR-Zerlegung ist im Anhang \ref{appTheorems} zu finden. Siehe hierzu Satz \ref{thm:appTheorems:QR}. F"ur weitere Ausf"uhrungen siehe auch \cite[S. 55 ff.]{stewart}.}
bem"uht. Diese f"uhrt in der Tat zu einer Normalisierung. Ist n"amlich $Y_{(k)} = QR$ mit
$Q = [q_i]_{i=1:m}$ die QR-Zerlegung, so gilt $\Bild(Y_{(k)}) = \Bild(Q)$ und $\|q_i\|_2 = 1$ f"ur $i=1:m$. Folglich stellt folgender Algorithmus eine Verallgemeinerung der Potenzmethode dar.

\begin{algorithm}
\caption{Verallgemeinerte Potenzmethode (Vgl. \cite[Algorithmus 5.1, 115]{saad})}\label{alg:chap4:potenzverfahrenMatrix}
\begin{algorithmic}[1]
\State W"ahle linear unabh"angige Vektoren $\{y_i\}_{i=1:m}$ und setze $Y_{(0)}\gets[y_i]_{i=1:m}$ und $k\gets 1$
\State \textbf{repeat}
\State \ \ \ \ Setze $Y_{(k)} \gets AY_{(k-1)}$ und berechne QR-Zerlegung $Y_{(k)} = QR$.
\State \ \ \ \ Setze $Y_{(k)} \gets Q$ und $k\gets k+1$.
\State \textbf{until} Verfahren konvergiert.
\end{algorithmic}
\end{algorithm}

Saad weist darauf hin, dass die Kosten der Berechnung der QR-Zerlegung sehr hoch werden k"onnen. Da der von den Spalten von $Y_{(k)}$ aufgespannte Unterraum gleich dem von den Spalten von $A^k Y_{(0)}$ aufgespannten Unterraum ist, schl"agt Saad daher folgende Abwandlung des eben vorgestellen Algorithmus' vor.

\begin{algorithm}
\caption{Gebrauch variabler Exponenten (Vgl. ~\cite[Algorithmus 5.2, 116]{saad})}\label{alg:chap4:potentePotenz}
\begin{algorithmic}[1]
\State W"ahle linear unabh"angige Vektoren $\{y_i\}_{i=1:m}$, setze $Y_\gets[y_i]_{i=1:m}$ und w"ahle initialen Exponenten $k$.
\State \textbf{repeat}
\State \ \ \ \ Setze $S \gets A^kY$ und orthonormalisiere $S$ zu $\widehat{S}$.
\State \ \ \ \ Setze $Y \gets \widehat{S}$.
\State \ \ \ \ W"ahle neuen Exponenten $k$.
\State \textbf{until} Verfahren konvergiert.
\end{algorithmic}
\end{algorithm}

Auch hier ist zu beachten, dass im Falle der Wahl eines sehr gro"sen Exponenten die Unabh"angigkeit der Spalten von $S$ nicht mehr gew"ahrleistet werden kann.
Wir wollen an dieser Stelle auf Konvergenz- und Laufzeitanalysen der eben vorgestellten Algorithmen verzichten und kommen schlie"slich zur vielfach angek"undigten Iterationsvorschrift, welche sich aus der Potenzmethode ableitet.

\newpage

\begin{algorithm}
\caption{Iteratives Rayleigh-Ritz Verfahren (Vgl. \cite[Algorithmus 5.3, 118]{saad})}\label{alg:chap4:rrIteration}
\begin{algorithmic}[1]
\State W"ahle linear unabh"angige Vektoren $\{y_i\}_{i=1:m}$, setze $Y_\gets[y_i]_{i=1:m}$ und w"ahle initialen Exponenten $k$.
\State \textbf{repeat}
\State \ \ \ \ Setze $S \gets A^k Y$.
\State \ \ \ \ Orthonormalisiere die Spalten von $S$ und setze $\widetilde{A} \gets S^H A S$.
\State \ \ \ \ Berechne Eigenvektoren $\widetilde{X} \gets [\widetilde{x}_i]_{i=1:m}$ von $\widetilde{A}$.
\State \ \ \ \ Setze $Y \gets S \widetilde{X}$.
\State \ \ \ \ W"ahle neuen Exponenten $k$.
\State \textbf{until} Verfahren konvergiert.
\end{algorithmic}
\end{algorithm}

Zun"achst ein Wort zur f"unften Zeile. Hier wurde das Berechnen von Schurvektoren -- so wie es in der oben zitierten Quelle vorgeschlagen wird -- durch das Berechnen von Eigenvektoren ersetzt. Dies ist m"oglich, weil $A$ nach Vereinbarung ein hermitesche Matrix ist und somit unit"ar diagonalisiert werden kann. Es ist daher nicht n"otig zwischen Eigenvektoren und Schurvektoren zu unterscheiden. Wie genau diese Eigenvektoren berechnet werden, wollen wir im Rahmen dieser Arbeit nicht genauer erl"autern.\\

Die Wurzeln des eben erarbeiteten Algorithmus' sind deutlich zu erkennen. In den Zeilen vier bis sechs wird das Rayleigh-Ritz Verfahren benutzt. Anstelle von Ritz-Paaren werden allerdings lediglich Ritz-Vektoren berechnet. In jeder Iteration wird wie beim Potenzverfahren ein neuer Exponent festgelegt und somit ein neuer Suchraum $\S = \Bild(A^k Y)$ vorgegeben. Erl"auterungen zum Konvergenzverhalten sind in \cite[Abschnitt 5]{saad} zu finden. Zur Berechnung von Eigenpaaren eines HPD-Eigenwertproblems $(A,B)$ bei dem $B$ nicht mehr der Identit"at entspricht, m"ussen die Zeilen vier bis sechs gem"a"s Algorithmus \ref{alg:chap3:grp} angepasst werden.\\

Es ist m"oglich, den zuletzt eingef"uhren Algorithmus weiter abzuwandeln. Dazu betrachten wir wieder ein HPD-Eigenwertproblem $(A,B)$. Zu diesem Duo gesellt sich nun mit $\p(B^{-1}A)$ ein Polynom in $B^{-1}A$, welches wir benutzen, um das iterative Rayleigh-Ritz Verfahren wie folgt zu "andern.

\begin{algorithm}
\caption{Beschleunigte Rayleigh-Ritz Iteration (Vgl. \cite[Algorithmus A]{ptep})}\label{alg:chap4:beschlRrIteration}
\begin{algorithmic}[1]
\State W"ahle $m$ linear unabh"angige Vektoren $Y_{(0)} \gets [y_i]_{i=1:m} \in\C^{n,m}$.
Setze $k \gets 1$.
\State \textbf{repeat}
\State \ \ \ \ Approximiere den Unterraumprojektor: $P_{(k)} \gets \p(B^{-1}A)Y_{(k-1)}$
\State \ \ \ \ Reduziere die Dimension: $\widetilde{A}_{(k)} \gets P_{(k)}^H A P_{(k)}$,
$\widetilde{B}_{(k)} \gets P_{(k)}^H B P_{(k)}$.
\State \ \ \ \ L"ose das transformierte Problem $\widetilde{A}_{(k)}\widetilde{X}_{(k)}
= \widetilde{B}_{(k)}\widetilde{X}_{(k)}\widetilde{\Lambda}_{(k)}$ in
$\widetilde{X}_{(k)}$ und $\widetilde{\Lambda}_{(k)}$.
\State \ \ \ \ Setze $Y_{(k)} \gets P_{(k)}\widetilde{X}_{(k)}$.
\State \ \ \ \ $k \gets k+1$.
\State \textbf{until} Abbruchkriterium ist erf"ullt.
\end{algorithmic}
\end{algorithm}

Diese Methode "ahnelt stark dem Algorithmus \ref{alg:chap4:rrIteration}. Die Zeilen vier bis sechs entsprechen erneut dem Rayleigh-Ritz Verfahren, aber die Berechnung des Suchraums $\S := \Bild(P_{(k)})$ geht anders vonstatten.

\newpage

Im Kontext dieses Algorithmus' wird $\p$ auch als \emph{Filter}\footnote{Dieser Filter ist nicht mit dem in Definition \ref{defn:chap3:filter} eingef"uhrten Begriff zu verwechseln.} oder \emph{Beschleuniger}
bezeichnet. Von dessen Wahl h"angt n"amlich ab, ob und wie gut Eigenpaare approximiert
werden: Sei $[\lambda_1,\lambda_2]$ ein reelles Intervall, auf dem $l\in\N$ Eigenwerte und die korrespondierenden Eigenvektoren gefunden werden k"onnen. Ist nun $\p(B^{-1}A)$ der Spektralprojektor,
$m=l$ und hat die Matrix $Y_{(1)} = \p(B^{-1}A) Y_{(0)}$ vollen Rang, so konvergiert der Algorithmus \ref{alg:chap4:beschlRrIteration} in einer Iteration
(Vgl. ~\cite[356]{ptep}).
Dies folgt unter Ausnutzung der Invarianz des Bildes von $P_{(1)}$ unter $B^{-1}A$
aus dem Satz $\ref{thm:chap3:invariant}$.\\

Da der Spektralprojektor in den meisten F"allen unbekannt sein d"urfte, liegt
die Idee nahe, ihn zu approximieren. Tang und Polizzi ~\cite[356]{ptep} merken an, dass dies gut funktioniert, falls $\p$ eine durch \emph{Gau"s-Legendre-Quadratur} konstruierte rationale Funktion ist.
Um den Gedankengang der Autoren nachvollziehen zu k"onnen, wird sich das folgende Intermezzo mit der Auffrischung des Konzeptes von Quadraturformeln und rationalen Funktionen besch"aftigen. Dabei sehen wir von Beweisen und ausufernden Erl"auterungen ab, da diese Thematik in den meisten Einf"uhrungsb"uchern zur numerischen Mathematik ausf"uhrlich besprochen wird.\footnote{Siehe zum Beispiel \cite[Abschnitt 6]{plato}} Im Anschluss fahren wir mit der
Konstruktion des Projektors fortfahren.

\section{Gau"s'sche Quadratur}

Um ein Integral numerisch zu approximieren, bedient man sich sogenannter \emph{Quadraturformeln}. Dazu betrachten wir eine stetige Funktion $f\colon\R\to\R$, welche wir auf einem gegebenen Intervall $I:=[a,b]\subset\R$ integrieren wollen.\footnote{Im Allgemeinen ist die Stetigkeit von $f$ nicht zwingend erforderlich. Wir werden uns hier der Einfachheit halber auf stetige Funktionen einschr"anken.}
Zu gegebenen St"utzpunkten $(x_i, f(x_i))_{i=0:n}$ auf $I\times\R$ sei $p_n$ das zugeh"orige \emph{Interpolationspolynom} vom Grad $n$, also ein Polynom, welches $p(x_i) = f(x_i)$ f"ur alle $i$ mit $0\le i\le n$ erf"ullt.\\

Dann bezeichnen wir die N"aherung
\begin{equation}\label{eq:quadratur}
Q_n(f) := \int_a^b p_n (x)\text{ d}x =
(b-a)\sum_{i=0}^n \omega_i f(x_i)
\end{equation}
als \emph{interpolatorische Quadraturformel}. Dabei gilt
\[
\omega_k = \int_0^1 \prod_{j=0,j\neq k}^n
\frac{t-t_j}{t_k - t_j} \text{ d}t, \ t_j
= \frac{x_j-a}{b-a}.
\]
Die Qualit"at der Approximation, also die Abweichung vom exakten Integral, h"angt ma"sgeblich von der Wahl und Anzahl der St"utzpunkte ab. Wollten wir
beispielsweise das Integral einer konstanten Funktion berechnen, so erschiene es wenig plausibel, anstelle der direkten Berechnung ein Polynom vom Grad 69 auf 70 St"utzstellen f"ur die Approximation zu bem"uhen.

\newpage

Bei der Anwendung von Gau"s-Legendre-Quadraturen ergibt sich die Wahl der St"utzpunkte durch die Berechnung von Nullstellen von Polynomen, die in einer Orthogonalit"atsbeziehung zueinander stehen.
Wir werden in K"urze formal formulieren, wie dies zu verstehen ist.\\

Ausgangspunkt f"ur die Integration ist nun eine stetige Funktion $f$, die eine Faktorisierung in zwei stetige Funktionen $g$ und $\omega$ der Art
\[
f = \omega \cdot g
\]
besitzt, wobei $\omega$ auf dem Integrationsintervall $[a,b]$ positiv sein soll. Die Funktion $\omega$ wird auch als \emph{Gewichtsfunktion} bezeichnet.
Ziel ist also nun die Berechnung von
\begin{equation}\label{eq:gintegral}
\int_a^b \omega(x)g(x) \text{ d}x,
\end{equation}
wobei wir zus"atzlich fordern, dass \eqref{eq:quadratur} und \eqref{eq:gintegral} f"ur alle Polynome bis zum Grad $(2n-1)$ "ubereinstimmen.\\

Dazu betrachten wir die Standardbasis $\{x^k\}_{k=0:(2n-1)}$ auf dem Raum der Polynome vom
Grad $(2n-1)$. Dann landen wir unweigerlich bei dem
Gleichungssystem
\[
\sum_{j=1}^n \omega_j x_j^k = \int_a^b \omega(x)\cdot x^k \text{ d}x \text{ mit } k = 0:2n-1.
\]
Man kann zeigen, dass die L"osung dieses Systems durch Nullstellen eines Polynoms gegeben ist, welches durch
ein Gram-Schmidt-Orthogonalisierungsverfahren bez"uglich des Skalarproduktes
\[
\langle p,q\rangle_\omega := \int_a^b p(x) q(x)\omega(x) \text{ d}x
\]
konstruiert wurde. Das hei"st konkret: Ausgehend vom Polynom $p_0 \equiv 1$ ist
\[
p_n(x) := x^n - \sum_{j=0}^{n-1} \frac{\langle x^n, p_j \rangle_\omega}{\langle p_j, p_j\rangle_\omega} p_j (x)
\]
gerade dasjenige Polynom, durch dessen Nullstellen das obige Gleichungssystem gel"ost wird. Sind nun
$\{x_j\}_{j=1:n}$ die Nullstellen dieses $n$-ten Orthogonalit"atspolynoms, so hei"st die numerische
Integrationsformel
\[
Q_n(f) = \sum_{j=1}^n \omega_j f(x_j) \text{ mit }
\omega_j = \langle L_j, 1 \rangle_\omega
= \int_a^b L_j(x)\p(x\text{ d}x
\]
\emph{Gau"s'sche Quadraturformel der $n$-ten Ordnung}.
Dabei ist
\[
L_j(x) = \prod_{k\neq j=1}^n \frac{x-x_j}{x_k - x_j}.
\]

\newpage

\section{Approximation des Spektralprojektors}

Sei $(A,B)$ ein HPD-Eigenwertproblem und $I:=[\lambda_1, \lambda_2]$ ein reelles Intervall. Sei weiter $\mathcal{X}\subseteq\Cn$ derjenige Unterraum, welcher von den Eigenvektoren aufgespannt wird, die zu den im Inneren von $I$ befindlichen Eigenwerten korrespondieren. Wir wollen in diesem Abschnitt den Spektralprojektor approximieren, welcher den Vektorraum $\Cn$ auf $\mathcal{X}$ projeziert.\\

Die erste Ingredienz, die zum Gelingen dieser Approximation beitr"agt, ist eine rationale Funktion $\r\colon\C\to\C$ mit $\r(\R) \subseteq \R$, welche auf $I$ n"aherungsweise der Indikatorfunktion von $I$ entspricht.
Daf"ur bem"uhen wir die Cauchy'sche Integraldarstellung von der Indikatorfunktion und
wandeln diese mit Hilfe numerischer Quadraturformeln in die gew"unschte
rationale Funktion $\r$ um.\\

Zu"achst zur Indikatorfunktion: Ist $c$ der Mittelpunkt des Intervalls $I$ und
$r$ der Abstand des Mittelpunktes zum Rand des Intervalls, dann entspricht die Menge
\[
\mathcal{C} := \{z\in\C : |z-c| = r\}
\]
der Sph"are mit Radius $r$ um $c$. Mit dem Cauchy'schen Integralsatz
l"asst sich zeigen, dass im Falle $z\notin \mathcal{C}$
\[
\frac{1}{2\pi\iota}\int_{ \mathcal{C}}\frac{1}{t-z}\text{ d}t
= \begin{cases}1 &\text{ falls }|z-c| < r \\ 0 &\text{ falls }|z-c| > 0 \end{cases}
\]
gilt. Um dieses Integral mit einer Gau"s'schen Quadraturformel ann"ahern zu k"onnen, ben"otigen wir eine Parametrisierung von $\mathcal{C}$. Hierf"ur ziehen wir \cite{ptep} zurate und w"ahlen demzufolge
die Funktion
\[
\gamma\colon[-1,3]\to\C\text{, }
x\mapsto c+re^{\iota \frac{\pi}{2}(1+x)}
\]
als Parametrisierung der Sph"are $\mathcal{C}$.
%Die Ableitung von $\gamma$ ist dann f"ur jedes $t\in[-1,3]$ durch
%\[
%\gamma'(t)=\iota \frac{\pi}{2}re^{\iota \frac{\pi}{2}(1+t)}
%\]
%gegeben.
Wir erhalten dann f"ur alle $z\notin\mathcal{C}$ die Gleichung
\begin{align*}
\frac{1}{2\pi\iota}\int_{ \mathcal{C}}\frac{1}{t-z}\text{ d}t
&= \frac{1}{2\pi\iota} \int_{-1}^3 \frac{\gamma'(x)}{\gamma(x)-z}\text{ d}x \\
&= \frac{1}{2\pi\iota} \left( \int_{-1}^1 \frac{\gamma'(x)}{\gamma(x)-z} \text{ d}x +
\int_{1}^3\frac{\gamma'(x)}{\gamma(x)-z}\text{ d}x \right) \\
&= \frac{1}{2\pi\iota} \left( \int_{-1}^1 \frac{\gamma'(x)}{\gamma(x)-z} \text{ d}x +
\int_{-1}^1\frac{\gamma'(2-x)}{\gamma(2-x)-z}\text{ d}x \right) \\
&= \frac{1}{2\pi\iota} \int_{-1}^1 \left( \frac{\gamma'(x)}{\gamma(x)-z} +
\frac{\overline{\gamma'(x)}}{\overline{\gamma(x)}-z}\right)\text{d}x
\end{align*}
wobei $\overline{\gamma(x)}$ und $\overline{\gamma'(x)}$ die komplexen Konjugationen
von $\gamma(x)$ beziehungsweise $\gamma'(x)$ bezeichnen.

\newpage

Es seien nun $(w_j, x_j)_{j=1:m}$
die f"ur die Gau"s'sche Quadraturformel ben"otigten Gewichte und Diskretisierungspunkte.
Dann setzen wir
\[
\rho(z) := \frac{1}{2\pi\iota}\sum_{j=1}^m \left(
\frac{w_j \cdot \gamma'(x_j)}{\gamma(x_j)-z} - \frac{w_j \cdot \overline{\gamma'(x_j)}}{\overline{\gamma(x_j)}-z}
\right)
\]
und erhalten nach der Substitution $\gamma(x_j) := \gamma_j$ und
$\sigma_j := w_j \gamma'(x_j) / (2\pi\iota)$ die gew"unschte rationale
Funktion
\[
\r\colon\C\to\C, z\mapsto\sum_{j=1}^m\left(\frac{\sigma_j}{\gamma_j - z} +
\frac{\overline{\sigma_j}}{\overline{\gamma_j} - z}\right)
\]
zur Approximation der Indikatorfunktion. Hierbei ist bemerkenswert, dass die
rationale Funktion bereits in Partialbruchzerlegung vorliegt.
Setzen wir schlie"slich $B^{-1}A$ in die
rationale Funktion ein, so erhalten wir
\begin{align*}
\r(B^{-1}A) &= \sum_{k=1}^m \sigma_k (\gamma_k I - B^{-1}A)^{-1} +
\sum_{k=1}^m \overline{\sigma_k} (\overline{\gamma_k} I - B^{-1}A)^{-1}\\
&= \sum_{k=1}^m \sigma_k (\gamma_k B - A)^{-1} B +
\sum_{k=1}^m \overline{\sigma_k} (\overline{\gamma_k} B - A)^{-1} B
\end{align*}
und folglich
\[
\r(B^{-1}A)Y =
\sum_{k=1}^q \sigma_k (\gamma_k B - A)^{-1} BY +
\sum_{k=1}^q \overline{\sigma_k} (\overline{\gamma_k} B - A)^{-1} BY
\]
f"ur eine Matrix $Y\in\C^{n,m}$. Sind $A, B$ und $Y$ reellwertig, so l"asst sich dies zu
\begin{equation}\label{eq:chap4:realteilsumme}
\r(B^{-1}A)Y =
2\sum_{k=1}^q \mathfrak{Re}\left(\sigma_k (\gamma_k B - A)^{-1} BY\right)
\end{equation}
vereinfachen. Dabei gebe die Funktion $\mathfrak{Re}$ den Realteil ihres Inputs aus. Wie man nun konkret diese Werte berechnet und wie ein Algorithmus aussehen kann, der dies umsetzt, besprechen wir im anschlie"senden Abschnitt. In diesem werden wir den sogenannten \emph{FEAST-Algorithmus} kennenlernen.

\newpage

\section{FEAST-Algorithmus}

Im Jahr 2009 stellte Eric Polizzi in \cite{polizzi} ein Verfahren vor, welches sich die Theorien der vorigen Abschnitte zu Nutze macht. Dieser \glqq[\ldots] \emph{fast and stable algorithm for solving the symmetric eigenvalue problem} [\ldots]\grqq\ \cite[Abstract]{polizzi} wurde seither weiterentwickelt und zahlreichen Analysen unterzogen.\footnote{Siehe etwa \cite{lzp},\cite{kpt} und \cite{ptep}.}\\
%Wir wollen in diesem Abschnitt die wichtigen Punkte des oben zitierten Papers skizzieren und damit das vierte Kapitel abschlie"sen.\\

Polizzi beginnt seine Ausf"uhrungen mit einem Blick in die Physik und motiviert das L"osen von Eigenwertproblemen anhand der Schr"odinger Eigenwertgleichung
\[
\textbf{\text{H}}\Psi = E\textbf{\text{S}}\Psi
\]
welche die Fragestellung modelliert, ob gewisse Quantenobjekte, die von einer Wellenfunktion $\Psi$ beschrieben werden, kinetische Energie besitzen oder nicht. Hierbei ist \textbf{H} eine hermitesche Matrix und \textbf{S} eine symmetrisch positiv definite Matrix.\\

Der Autor merkt an, dass das L"osen solcher Systeme, insbesondere dann, wenn sie sehr gro"s sind, enorme Anforderungen an die Numeriker stellt. Die Frage ist stets, wie Eigenpaare effizient berechnet werden k"onnen und welche Genauigkeit man erwarten darf. Polizzi bewirbt seinen Algorithmus mit hoher Geschwindigkeit, Robustheit und guter Skalierbarkeit.\\ %Seinem Urteil nach, werden Methoden, wie \glqq\emph{Rayleigh-quotient multigrid}\grqq\ oder \glqq\emph{parallel Chebyshev subspace iteration}\grqq\ als weniger effizient eingesch"atzt.\\

Ausgehend von einer Ausf"uhrung "uber das Konzept der Konturintegration zur Bestimmung von Eigenpaaren, baut Polizzi seinen Algorithmus auf, welcher auf $N$-dimensionale Eigenwertprobleme $(A,B)$ mit hermiteschem oder reell symmetrischem $A$ und symmetrisch positiv definitem $B$ ausgelegt ist und auf einem reellen Intervall $[\lambda_1,\lambda_2]$ die $m\in\N$ Eigenpaare finden soll.
Sind $(\omega_j, s_j)_{j=1:k}$ die f"ur die Gau"s'sche Quadraturformel ben"otigten Gewichte und St"utzstellen, so l"asst sich der vorgestellte FEAST-Algorithmus wie folgt notieren.

\begin{algorithm}
\caption{FEAST-Algorithmus (Vgl. \cite[Abschnitt III]{polizzi})}\label{alg:chap4:feast}
\begin{algorithmic}[1]
\State W"ahle $M > m$ linear unabh"angige Vektoren $Y \gets [y_i]_{i=1:M} \in\C^{n,M}$, setze $Q\gets 0_{n,M}$ und $r\gets (\lambda_1 - \lambda_2)/2$.
\State \textbf{repeat}
\State \ \ \ \ \textbf{for j=1:k}

\State \ \ \ \ \ \ \ \ Berechne $\theta_{(j)} \gets -(\pi/2)(s_j-1)$.
\State \ \ \ \ \ \ \ \ Berechne $t_{(j)} \gets (\lambda_2 + \lambda_1)/2 + re^{\iota\theta_{(j)}}$.
\State \ \ \ \ \ \ \ \ L"ose $(t_{(j)} B-A)Q_{(j)} = Y$ in $Q_{(j)}$.
\State \ \ \ \ \ \ \ \ Setze $Q\gets Q - (\omega_j/2)\cdot\mathfrak{Re}\left(re^{\iota \theta_{(j)}} Q_{(j)}\right)$.
\State \ \ \ \ \textbf{endfor}
\State \ \ \ \ Reduziere die Dimension: $\widetilde{A} \gets Q^H A Q$,
$\widetilde{B} \gets Q^H B Q$.
\State \ \ \ \ L"ose das transformierte Problem $\widetilde{A}\widetilde{X}
= \widetilde{B}\widetilde{X}\widetilde{\Lambda}$ in
$\widetilde{X}$ und $\widetilde{\Lambda} = \text{diag}(\widetilde{\lambda}_1,\ldots,\widetilde{\lambda}_M)$.
\State \ \ \ \ Setze $X \gets [x_i]_{i=1:M} = Q\widetilde{X}$.
\State \ \ \ \ Gilt $\widetilde{\lambda_i} \in [\lambda_1,\lambda_2]$, so gibt Eigenpaar $(\widetilde{\lambda_i},x_i)$ aus.

\State \textbf{until} Abbruchkriterium ist erf"ullt.
\end{algorithmic}
\end{algorithm}

Dieser Algorithmus stellt eine Umsetzung der beschleunigten Rayleigh-Ritz Iteration dar. Die Zeilen drei bis acht erf"ullen den Zweck des Berechnens von $\p(B^{-1}A)Y$, wobei man sich hier einer "ahnlichen Umformulierung bedient, wie in Gleichung \eqref{eq:chap4:realteilsumme}.\footnote{Die entsprechende Anpassung von Algorithmus \ref{alg:chap4:beschlRrIteration} ist in \cite[365]{ptep} zu finden.}\\

Es ist zu beachten, dass sich hier durch die Verwendung einer negativ orientierten Kurve, was aus der vierten und f"unften Zeile herauszulesen ist, Vorzeichenwechsel ergeben. Da bei einer entsprechenden Parametrisierung auf der rechten Seite der Identit"at \eqref{eq:chap3:kontur} der zus"atzliche Faktor $(-1)$ ben"otigt wird, muss dies entsprechend bei der Konstruktion der rationalen Funktion $\p$ ber"ucksichtigt werden.\\

Die weiteren Abschnitte des Papers besch"aftigen sich vornehmlich mit numerischen Experimenten. Polizzi demonstriert anhand elektronenstrukturell bezogenen Berechnungen auf Kohlenstoffnanor"ohren die numerische Stabilit"at, Robustheit und Skalierbarkeit des FEAST-Algorithmus. Dabei vergleicht er FEAST mit ARPACK -- einer Bibliothek auf FORTAN77 basierenden Methoden zum L"osen hochdimensionaler Eigenwertprobleme.\\

Abschlie"send hebt der Autor einige Merkmale seines Algorithmus' heraus, von denen an dieser Stelle eine bescheidene Auswahl vorgestellt wird. Zun"achst einmal kommt FEAST g"anzlich ohne Orthogonalisierungsmethoden aus, wie es etwa bei der unbeschleunigten Rayleigh-Ritz Iteration n"otig ist. Des Weiteren weist Polizzi auf das Potential bez"uglich paralleler Implementierbarkeit hin.\\

In Papern aus den vergangenen zwei Jahren, wie etwa in \cite[Abschnitt 4.1]{kpt}, wird auf dieses Potential genauer eingegangen. Inzwischen wurden auch Versuche unternommen, FEAST auf nicht hermitesche Eigenwertprobleme auszuweiten.\footnote{Siehe \cite{kpt}.}

%Im dritten Kapitel wurde mit Satz \ref{thm:chap3:invariant} gezeigt, dass Ritz-Paare im Falle der $(B^{-1}A)$-Invarianz des Suchraumes $\S$ schon Eigenpaare sind. Wir k"onnten folglich die Iteration in Algorithmus \ref{alg:chap4:rrIteration} beenden, sobald eine Potenz


%\chapter{Implementation und numerische Experimente}
\chapter{Numerische Experimente}
Die folgende Proposition weist nach, dass der eben vorgestellte Algorithmus tats"achlich
ein Projektionsverfahren ist.

\begin{prop}\label{prop:projektor}
Die durch die Matrix $V_m V_m^H \in \Cnn$ induzierte lineare Abbildung
\[
P\colon \Cn \to \Cn, x\mapsto V_m V_m^H x
\]
ist eine orthogonale Projektion auf den Unterraum V und l"ost das Minimierungsproblem
\[
\min_{y\in V}\|x-y\|, x\in\Cn (\|x-P(x)\| = ...)
\]
\end{prop}

\begin{proof}
Sei $x\in\Cn$ vorgegeben. Dann gilt $P(x) = V_m V_m^H x \in \Bild(V_m) = V$ nach Konstruktion.
Da aus der Orthogonalit"at der Spalten von $V_m$
\[
P^2 = V_m V_m^H V_m V_m^H = V_m V_m^H = P
\]
folgt, ist $P$ somit eine Projektion auf $V$. Dar"uber hinaus gilt f"ur alle
Vektoren $y\in V$
\[
\langle y, x-P(x)\rangle = y^H (x - V_m V_m^H x) = y^H x - P(y)^H x = 0
\]
Also ist $x-P(x) \in V^{\bot}$ und $P$ tats"achlich eine orthogonale Projektion.
Wegen $P(x)-y \in V$ folgt die Minimalit"at aus der Absch"atzung
\begin{align*}
\|x-y\|^2 &= \|x-P(x) + P(x)-y\|^2 \\
&= \langle x-P(x) + P(x)-y, x-P(x) + P(x)-y \rangle \\
&= \|x-P(x)\|^2 + \langle x-P(x), P(x)-y \rangle + \langle P(x)-y, x-P(x)\rangle + \|P(x)-y\|^2 \\
&= \|x - P(x)\|^2 + \| P(x)-y \|^2\\
&\ge \|x-P(x)\|^2.
\end{align*}
Damit ist die Behauptung bewiesen, da Gleichheit nur mit $y=P(x)$ folgen kann.
\end{proof}

Wir wollen nun die eben erarbeitete Theorie auf das verallgemeinerte Eigenwertproblem
\[
Ax = \lambda Bx
\]
"ubertragen.


\appendix
\chapter{Symbolverzeichnis}\label{appNotation}
\begin{tabular}{ll}
$\N$ & Menge der nat"urlichen Zahlen \{1,2,3,\ldots\}\\
$\N_{0}$ & $\N\cup\{0\}$\\
$\R$ & Menge der reellen Zahlen\\
$\C$ & Menge der komplexen Zahlen\\
$\R^{m,n}$ & Menge der reellen $(m\times n)$-Matrizen\\
$\C^{m,n}$ & Menge der komplexen $(m\times n)$-Matrizen\\
$\R^n$ & Elemente aus $\R^{n,1}$\\
$\Cn$ & Elemente aus $\C^{n,1}$\\
$\spn{x}$ & Lineare H"ulle $\{\lambda x \mid \lambda\in\C\}$ des Vektors $x$\\
$\delta_{i,j}$ & Kronecker-Delta $\delta_{i,j} = 1$, f"ur $i=j$ und $\delta_{i,j} = 0$, f"ur $i\neq j$\\
$I_n$ & Einheitsmatrix $[\delta_{i,j}]_{i,j=1:n}$\\
$0_n$ & Nullmatrix des $\Cnn$\\
$0_{m,n}$ & Nullmatrix des $\C^{m,n}$\\
$A^H$ & Komplex konjugiert Transponierte der Matrix $A$\\
$\gg$ & Erheblich gr"o"ser als\\
$\bot$ & Steht orthogonal / unit"ar auf\\
$\bot_B$ & Steht $B$-orthogonal / $B$-unit"ar auf\\
$\U^{\bot_B}$ & $B$-orthogonales Komplement des Unterraums $\U$\\
$\langle x,y\rangle_B$ & Schreibweise f"ur $x^H B y$, f"ur eine HPD-Matrix $B$\\
$\|\cdot\|_B$ & Von der Matrix $B$ induzierte Norm: $\|x\|_B := \sqrt{x^H B x}$\\
$\|x\|_2$ & Die euklidische Norm $\sqrt{x^H x}$\\
$\|A\|_2$ & Induzierte $2$-Norm f"ur Matrizen (genauer)\\
$\mathfrak{Re}$ & Realteil komplexer Zahlen
\end{tabular}


\chapter{S"atze}\label{appTheorems}
\begin{thm}[Singul"arwertzerlegung]\label{thm:appTheorems:svd}
Ist $A\in\C^{m,n}$ eine $r$-rangige Matrix, so existieren unit"are Matrizen $U\in\C^{m,m}, V\in\Cnn$
und eine Diagonalmatrix $\Sigma^2_+ =\text{diag}(\sigma^2_1,\ldots,\sigma^2_r)\in\C^{r,r}$ mit der Eigenschaft $\sigma_1 \ge \ldots \ge \sigma_r > 0$, die
eine Faktorisierung der Art
\[
A = U \Sigma V^H
\]
erm"oglichen. Dabei ist
\[
\Sigma = \begin{bmatrix} \Sigma_+ & 0_{r,n-r} \\
0_{m-r,r} & 0_{m-r,n-r} \end{bmatrix}.
\]
\end{thm}

\begin{thm}[Beste Rang-$k$ Approximierung]\label{thm:appTheorems:rang}
Ist $A\in\C^{m,n}$ eine $r$-rangige Matrix und $A = U\Sigma V^H$ mit $U=[u_i]_{i=1:m}$ und $V=[v_i]_{i=1:n}$ die Singul"arwertzerlegung von $A$ wie in Satz \ref{thm:appTheorems:svd}, dann gilt
\[
\left\|A-\sum_{i=1}^k \sigma_i u_i v_i^H\right\|_2 = \min_{\substack{B\in\C^{m,n} \\ \text{Rang}(B)=k}} \|A-B\|_2
\]
f"ur alle $k=1:r$.
\end{thm}

\begin{thm}[Spektralsatz f"ur hermitesche Matrizen]\label{thm:appTheorems:Spektralsatz}
Eine Matrix $A\in\Cnn$ ist genau dann hermitesch, wenn sie unit"ar diagonalisierbar ist und alle Eigenwerte von $A$ reell sind.
\end{thm}

\begin{thm}[Existenz der Cholesky-Zerlegung]\label{thm:appTheorems:Cholesky}
Ist $A\in\Cnn$ eine HPD-Matrix, so existiert eine eindeutig bestimmte untere Dreiecksmatrix $L\in\Cnn$ mit positiven Diagonalelementen, mit $A=LL^H$.
\end{thm}

%\begin{thm}[Jordan'sche Normalform]\label{thm:appTheorems:Jordan}
%blubb
%\end{thm}

\begin{thm}[Cauchy'scher Integralsatz]\label{thm:appTheorems:Cauchy}
Ist $\Omega\subseteq\C$ ein einfach zusammenh"angendes Gebiet, $f\colon\Omega\to\C$ eine holomorphe Funktion, sowie $\Gamma$ eine Jordan-Kurve, so gilt
\[
\oint_\Gamma f(z) \text{ d}z = 0.
\]
\end{thm}

\newpage

\begin{thm}[Potenzmethode]\label{thm:appTheorems:Potenzmethode} Es sei $A\in\Cnn$ eine hermitesche Matrix mit Eigenpaaren $(\lambda_i,x_i)_{i=1:n} \in \R\times\Co$ und $n\ge 2$.
Es gelte weiter $|\lambda_1| > |\lambda_j|$ f"ur alle
$j > 1$. Ist $y_0 \in\Co$, so konvergiert die Folge
$(y_k)_{k\in\N}$ mit
\[
y_{(k+1)} = \frac{1}{\|A^{k+1} y_{(0)}\|} A^{k+1}y_{(0)}
\]
gegen einen normierten Eigenvektor zum Eigenwert $\lambda_1$
\end{thm}

\begin{thm}[QR-Zerlegung]\label{thm:appTheorems:QR}
Sei $A\in\C^{n,m}$ mit $n\ge m$. Dann exisitert eine unit"are Matrix $Q\in\Cnn$ und eine obere Dreiecksmatrix $R = [r_{ij}]_{i,j=1:m}\in\C^{m,m}$ mit $r_{ii}\ge 0$ f"ur $i=1:m$ mit
\[
A = Q\begin{bmatrix} R \\ 0_{n-m,m} \end{bmatrix}.
\]
\end{thm}


\chapter{Algorithmen}\label{appAlgorithms}
\begin{lstlisting}[caption=Eigenpaarberechnung mit Zeiterfassung, captionpos=b, label=alg:appAlgorithm:eigenPlot]
% Eingabe: -; Ausgabe: -
% Funktionsaufruf: eigenPlot

function eigenPlot

% Anpassung an Latexfont
set(0,'defaulttextinterpreter','latex');

% Beschriftung des Plots
figure; hold on;
xlabel('Dimension'); ylabel('Ben\"{o}tigte Rechenzeit in Sekunden');

% Initialisierung der x,y-Achsenwerte
xDimension = 10:10:1500; yTime = 10:10:1500;

  % Beginn der Eigenpaarberechnung
  for i=1:3
    for N=10:10:1500  % zehnstufiger Dimensionszuwachs
      tMatrix = tic; tEigenpairs = tic; % Timerinitialisierung
      A = rand(N); A = A'*A; % Erzeugung einer symmetrischen Matrix
      endMatrix = toc(tMatrix); % Dauer Matrixinitialisierung
      [X, D] = eig(A);  % Berechnung der Eigenpaare
      endEigenpairs = toc(tEigenpairs);
      yTime(N/10) = endEigenpairs-endMatrix;  % Dauer Eigenpaarberechnung
    end%for
    plot(xDimension, yTime);  % Erstellung des Diagramms
  end%for
  print -depsc eigLaufzeit;  % Umwandlung in .eps Datei

end%function
\end{lstlisting}

\newpage
\begin{lstlisting}[caption=Rayleigh-Ritz Verfahren mit Zeiterfassung, captionpos=b, label=alg:appAlgorithm:rayleighRitz]
% Eingabe: -; Ausgabe: -
% Funktionsaufruf: rayleighRitz

function rayleighRitz

% Anpassung an Latexfont
set(0,'defaulttextinterpreter','latex');

% Beschriftung des Plots
figure; hold on;
xlabel('Dimension $N$ der untersuchten Matrix');
ylabel({'Ben\"{o}tigte Rechenzeit in Sekunden'; ...
  'f\"{u}r $N/5$-dimensionalen Suchraum'});

% Initialisierung der x,y-Achsenwerte
xDimension = 10:10:1500; yTime = 10:10:1500;

  % Beginn der Eigenpaarberechnung
  for i=1:3
    for N=10:10:1500  % zehnstufiger Dimensionszuwachs
      tMatrix = tic; tRitzpairs = tic; % Timerinitialisierung
      A = rand(N); A = A'*A; % Erzeugung einer symmetrischen Matrix
      S = rand(N, N/5); % Suchraum der Dimension N/5
      endMatrix = toc(tMatrix); % Dauer Matrixinitialisierung

      % Rayleigh-Ritz Verfahren
      S = orth(S);  % Orthogonalisierung der Basisvektoren
      A = S'*A*S; % Verringerung der Dimension
      [X, D] = eig(A); S*X; % Berechnung der Ritz-Paare
      endRitzpairs = toc(tRitzpairs);
      yTime(N/10) = endRitzpairs-endMatrix;  % Dauer Eigenpaarberechnung
    end%for
    plot(xDimension, yTime);  % Erstellung des Diagramms
  end%for
  print -depsc rayleighRitzLaufzeit;  % Umwandlung in .eps Datei

end%function
\end{lstlisting}

\newpage
\begin{lstlisting}[caption=Berechnung des Winkels zwischen Ritz- und Eigenvektoren, captionpos=b, label=alg:appAlgorithm:ritzVecAngle]
% Eingabe: -; Ausgabe: -
% Funktionsaufruf: ritzPairAngle

function ritzPairAngle

% Anpassung an Latexfont
set(0,'defaulttextinterpreter','latex');

% Beschriftung des Plots
figure; hold on;
xlabel('Dimension der untersuchten Matrix');
ylabel({'Winkel in Bogenma{\ss}'});

% Initialisierung der x-Achse
xDimension = 10:10:500;

  % Beginn der Eigenpaarberechnung
  for i = 1:3
    for N = 10:10:500  % zehnstufiger Dimensionszuwachs
      %clearvars -except i N xDimension yMaxDist
      [i, N]
      A = rand(N); A = A'*A; % Erzeuge symmetrische Matrix
      [X,D] = eig(A);
      S = rand(N, 2); % Suchraum der Dimension 2

      % Rayleigh-Ritz Verfahren
      S = orth(S);  % Orthogonalisierung der Basisvektoren
      A2 = S'*A*S; % Verringerung der Dimension
      [X2, D2] = eig(A2); X2 = S*X2; % Berechnung der Ritz-Paare

      % Berechnung des minimalen Winkels
      p = 1:N; P = nchoosek(p,2);
      for k = 1:length(P) % = #Zeilen von P, wegen N >= 10
          Theta(k) = subspace(X2, X(:,P(k,:)));
      end
      yTheta(N/10) = min(Theta);
    end%for
    plot(xDimension, yTheta);  % Erstellung des Diagramms
  end%for
  print -depsc ritzPairAngle;  % Umwandlung in .eps Datei

end%function
\end{lstlisting}

\newpage

\begin{lstlisting}[caption=Berechnung des Winkels zwischen Ritz- und Eigenvektoren bei iterativem Rayleigh-Ritz Verfahren, captionpos=b, label=alg:appAlgorithm:iterRitzVecAngle]
% Eingabe: -; Ausgabe: -
% Funktionsaufruf: iterRayleighRitz

function iterRayleighRitz

% Anpassung an Latexfont
set(0,'defaulttextinterpreter','latex');

% Beschriftung des Plots
figure; hold on; xIteration = 1:50;
xlabel('Iteration'); ylabel('Winkel in Bogenma{\ss}');

%for i=1:3
A = rand(500); A=A'*A; [X, D] = eig(A);
Y1 = rand(500,2);
k1 = 1;

for j=1:100

    % Rayleigh-Ritz Prozedur
    S1 = A^k1*Y1; S1 = orth(S1);
    A1 = S1'*A*S1;
    [X1, D1] = eig(A1); Y1 = S1*X1;

    % Berechnung des Winkels
    p = 1:500; P = nchoosek(p,2);
    for k = 1:length(P) % = #Zeilen von P
        Theta(k) = subspace(Y1, X(:,P(k,:)));
    end
    yTheta(j) = min(Theta);
end%for
plot(xIteration, yTheta);

%print -depsc iterRitzAngle;

end%function
\end{lstlisting}

\newpage

\begin{lstlisting}[caption=FEAST, captionpos=b, label=alg:appAlgorithm:FEAST]
% Eingabe: -; % Ausgabe: -
% Funktionsaufruf: feast

function feast

% Initialisierungen
n = 500; iter = 10; lmin = 1; lmax = 2.5;

% Initialisierung des Plots
set(0,'defaulttextinterpreter','latex');
figure; hold on;
xlabel('Durchgef\"{u}hrte Iterationen');
ylabel('Winkel $\theta_{\min}(\mathcal{X},\mathcal{R})$');

% Erzeuge und transformiere die Filterfunktionen r auf
% das Intervall [lmin,lmax].
x = rkfun(); t = 2/(lmax-lmin)*x - (lmin+lmax)/(lmax-lmin);
s = rkfun('step', 5); r = s(t);

for j=1:3
    % Erzeuge HPD-Eigenwertproblem und skaliere die Matrizen
    A = 10*rand(n); A=A'*A - 1;
    B = 20*rand(n); B=B'*B;

    % Berechne Eigenpaare auf ]lmin,lmax[ mit eig(A,B) als Referenz
    [X, D] = eig(A,B);

    for i=1:n
        if lmin < D(i,i) < lmax
           eigVecInterval(:,i) = X(:,i);
           eigValInterval(i) = D(i,i);
        end%if
    end%for

    Yk = rand(n, length(eigValInterval));
    xAxis = 1:iter; M = B\A;

    for k=1:iter
        % Beschleunnigtes Rayleigh-Ritz Verfahren
        Pk = r(M, Yk); % Approximierung des Projektors
        Ak = Pk'*A*Pk; Bk = Pk'*B*Pk;
        [Xk, Dk] = eig(Ak, Bk);
        Yk = Pk*Xk;
        % Berechnung des Unterraumwinkels
        theta(k) = subspace(Yk, eigVecInterval);
    end%for
    semilogy(xAxis, theta);
end%for
end%function

\end{lstlisting}


\nocite{*}
%\addcontentsline{toc}{chapter}{Literatur}
\printbibliography[heading=bibintoc]

\end{document}
