\documentclass[a4paper, 11pt]{article}

\usepackage[ngerman]{babel}
\usepackage{amsmath, amsfonts, amssymb, amsthm}
\usepackage{geometry}
\usepackage[onehalfspacing]{setspace}

\usepackage{fontspec}
\setmainfont[
	BoldFont=MinionPro-Bold.otf,
	ItalicFont=Minion Pro Italic.ttf
	]{MinionPro-Regular.otf}

\setlength{\parindent}{0pt}

\newcommand{\Cnn}{\mathbb{C}^{n,n}}
\newcommand{\Cn}{\mathbb{C}^n}
\newcommand{\C}{\mathbb{C}}
\newcommand{\Co}{\mathbb{\C}\setminus\{0\}}
\newcommand{\R}{\mathbb{R}}
\newcommand{\GL}{\text{GL}_n (\C)}

\newtheorem{prop}{Proposition}

\begin{document}

\begin{prop}
Es seien $A, B \in \Cnn$ hermitesch und $B$ invertierbar. Ist $(\lambda, x)\in\C\times\Co$ ein
Paar, welches der verallgemeinerten Eigenwertgleichung
\[
Ax = \lambda Bx
\]
gen"ugt, so ist $\lambda$ reell.
\end{prop}

\begin{proof}
Wir bem"uhen das euklidische Skalarprodukt: Ist $(\lambda, x)\in\C\times\Co$ ein zul"assiges
Eigenpaar, so gilt
\[
\lambda(x,Bx) = (x,Ax) = (Ax,x) = \bar{\lambda}(Bx,x)
\]
nach Voraussetzung. Der Hermitizit"at und der Invertierbarkeit von $B$ wegen, gilt 
\[
(x,Bx)=(Bx,x) \neq 0
\]
und daher folgt aus $\lambda = \bar{\lambda}$ die Behauptung.
\end{proof}

\end{document}