\documentclass[a4paper, 11pt]{article}

\usepackage[ngerman]{babel}
\usepackage{amsmath, amsfonts, amssymb, amsthm}
\usepackage{geometry}
\usepackage[onehalfspacing]{setspace}

\usepackage{fontspec}
\setmainfont[
	BoldFont=MinionPro-Bold.otf,
	ItalicFont=Minion Pro Italic.ttf
	]{MinionPro-Regular.otf}

\setlength{\parindent}{0pt}

\newcommand{\Cnn}{\mathbb{C}^{n,n}}
\newcommand{\C}{\mathbb{C}}
\newtheorem{prop}{Proposition}

\begin{document}
\title{Der FEAST-Algorithmus}
\author{Thorsten Lucke}

\maketitle

\tableofcontents

\vfill
\section{Einleitung} %fuer bA noch: wie filtere ich eigenwerte einer matrix auf einem intervall
Das L"osen von Eigenwertproblemen ist eine Standarddisziplin in der
numerischen linearen Algebra. Gleichungen der Gestalt
\[
Ax = \lambda Bx
\]
begegnet man in ganz unterschiedlichen Kontexten. So sind sie beispielsweise bei der Bestimmung von Eigenfrequenzen oder dem Ermitteln von Fixpunkten beim Rotieren eines Fu"sballs\footnote{Hier wird auf den bekannten \emph{Satz vom Fu"sball} angespielt. Dieser
besagt, dass auf einem Fu"sball zwei Punkte existieren, die zu Spielbeginn und zur Halbzeit
an der gleichen Stelle liegen -- informell formuliert.} ebenso wie beim Untersuchen des PageRanks einer Website von
Bedeutung.
Entsprechend strotz der Kanon von angebotenen numerischen
L"osungsmethoden von Vielfalt und Virtuosit"at.

\section{Mathematische Idee}
Es seien nun zwei  Zahlen $\lambda_1, \lambda_2 \in \mathbb{R}$ vorgegeben, die
das reelle Intervall $I:=[\lambda_1, \lambda_2] \subseteq \mathbb{R}$ definieren.
F"ur zwei hermitesche Matrizen $A, B \in \Cnn$ mit der zus"atzlichen
Forderung, dass $B$ positiv definit ist, sollen Paare der Gestalt
$(\lambda, x) \in I \times \mathbb{C}^n$ ermittelt werden, welche der verallgemeinerten
Eigenwertgleichung
\begin{equation}
Ax = \lambda Bx
\end{equation}
gen"ugen.\\

Zur Bestimmung der Transformationsmatrix $Q$ wird eine Konturintegration bem"uht.
Ausgangspunkt dieser Integration ist die durch
\begin{align*}
G\colon\Omega &\to\mathbb{C}^{n,n}\\
\omega &\mapsto (\omega B - A)^{-1}
\end{align*}

definierte  \textsc{Green}-Funktion, wobei $\Omega \subseteq \C$ eine passende
Teilmenge der komplexen Zahlen ist. Insbesondere m"ussen $\Omega$ und das
Spektrum von $B^{-1}A$ disjunkt sein, da $G$ andernfalls nicht wohldefiniert ist.\\

Die Funktion $G$ wird nun "uber eine geschlossene komplexe Kurve $\gamma$,
die um das vorgegebene Intervall $I$ heruml"auft, in der Gestalt
\[
-\frac{1}{2\pi\iota}\int_\gamma G(\omega)\text{ d}\omega
\]
integriert.\footnote{Mit $\iota$ ist im Folgenden stets die imagin"are Einheit bezeichnet,
also $\iota = \sqrt{-1}$.}

\end{document}